%&12pt
\pdfpagewidth=297mm
\pdfpageheight=210mm
\pdfhorigin=1in
\pdfvorigin=0pt
\input quire
\shhtotal=\pdfpagewidth
\htotal=.5\shhtotal
\vtotal=\pdfpageheight
\shoutline=0pt
\shstaplewidth=0pt
\shcrop=0pt
\shfootline={}
\shthickness=0pt % we want to use staples, but do not use .27mm (and do not do corresponding manipulations with \horigin and \hsize) until you do TODO in ~/term/стихи.tex
                 % (this document consists only of 3 sheets, so protrusions are almot imperceptible)
\quire{12}

\horigin=9mm
\hoffset=-9mm
\hsize=\htotal \advance\hsize by-2\horigin \advance\hsize by\hoffset
\advance\horigin by-\hoffset
\output={\ifodd\pageno\hoffset=0pt\fi \plainoutput}

\vorigin=15mm
\vsize=\topskip \advance\vsize by35\baselineskip

\font\TENSL=OMSL10
\headline={\rlap{\vbox to0pt{\vss\hbox{\raise5pt\line{\TENSL
  \ifodd\pageno\hfil Разговор у костра\kern.5pt\else
  Летний интенсив по тайцзицюань\hfil\fi}}}}\line{\hrulefill}}
\footline={\raise3pt\line{\hss\tenrm\folio\hss}}

\font\speakerF=omssbx10
\def\A{\item{\speakerF А.}}
\def\I{\item{\speakerF И.}}
\def\L{\item{\speakerF Л.}}
\def\N{\item{\speakerF Н.}}
\setbox0=\hbox{\speakerF А.\enskip}
\parindent=\wd0

\hyphenation{тай-цзи-цю-ань}

\A
... со знанием программ развития, реабилитации и так далее.
То есть нормальное советское образование.
И, соответственно, когда это всё появилось,
резко в Америке появились клубы, появились методики,
появилось понятие индивидуального тренера.
Потому что стало ясно, что можно обучать людей на условно-научной основе.
То есть в этом смысле в качалке тоже не всё так просто.

\I
Мне в качалке как-то меньше понравилось.
Я если днём устаю --- раз, кусочек комплекса сделал --- уже размялся.

\A
Это понятно, я с тобой согласен.
Мне тоже качалка не очень интересна.
Я дома иногда гири тягаю, по своим соображениям:
в какой-то момент не хватает физухи, и надо её добавлять.
У нас все, кто занимается больше 10--15 лет, в какой-то момент добавляют физуху.
Я для себя выбрал гири. И иногда работу с тяжёлым оружием.
Алебарды тяжёлые, ещё что-то.

\I
Тут основной момент в чём для меня лично заключается:
что мозг, он как бы...
То есть, занятия должны быть регулярными.
И я пришёл к тому, что регулярность, она появляется только...
Допустим, один человек может просто 5 минут просидеть,
а другой человек за 5 минут может какой-то комплекс сделать.
То есть, нужен какой-то фактор, который как-бы страгивает с места.
Чтобы это в привычку вошло.
А чтобы это вошло в привычку, нужно
чтобы оно с одной стороны не вызывало противодействия внутри мозга.
``Вот опять вставать, там, тягать.''
Чтобы это и чуть-чуть нагружало, и при этом давало бы некий приятный момент.
И вот как раз эти комплексы не вызывают такого внутреннего подсознательного протеста.
И моему мозгу легче их принять на регулярной основе.
Я поэтому к этому вот способу и пришёл, вот эти комплексы, вот это всё.
А качалка, мне её тяжело сделать регулярной.
Это надо конкретно, прям регулярно.
Если не будет регулярности, процесса не будет, толку не будет.
То есть надо прям,
чтобы мозг знал, что вот в этот день, у тебя в это время вот это,
чтобы уже прям на подкорке всё это было.
Тогда это для меня работает.
А иначе я буду какие-то искать оправдания: что-то, куда-то, чего-то.
И именно я пришёл к тому, что вот эти комплексы, они как раз для моего мозга
проще дают как бы разбег вот этот, чтобы в привычку всё это вошло.
И мы вот каждый день занимаемся и с одной стороны,
как бы утром неохота вставать, а потом
пришёл, цигунчик поделал,
и у тебя по телу всё приятно, и тебе приятно,
и тебе на следующий день уже охота
идти туда.
То есть, как бы
сильно не нагружать, и при этом
чуть-чуть прорабатывать. А потом уже
в конце и нагрузочки маленько можно
дать. И оно постепенно,
постепенно идёт.
Вот так.
Может просто, что
тут ещё аура добавляется некая.
То есть, вот эти все
фишечки восточные всякие,
такие мелочи. А спортзал
он как бы, ну, не
имеет некой
оболочки, да, упаковки
такой красивой.
Культуры, скажем так. Для меня,
по крайней мере. Может, если углубиться, да.

\A
Ну, или просто ты не нашёл
те залы, где это есть.
То есть, тут, как говорится, кому
с чем повезло
встретиться,
тот туда и пошёл.
Мне вот повезло встретиться
с боевыми искусствами.
А при этом, вот, например, мне
не повезло встретиться со скалолазами.
По крайней мере,
в раннем возрасте.
И когда я уже попробовал
скалолазы, я уже был достаточно взрослый.
Мне уже лениво было переучиваться.
Или, например, бокс.
Вот не повезло мне.
Не было интересной
секции бокса.
В наше время вообще бокса было мало.
Где-то там существовал,
как некое такое, что-то
спортивное-спортивное совсем.
Вот.
Зато рукопашный бой повезло.
Десантники у нас
в Туле, где я рос,
создали военно-патриотический
клуб, и я там занимался
рукопашным боем.
Ну, короче, вот
такие моменты.
Один из наших
преподавателей, Роман Романов,
говорит ``Судьба.''
С чем тебя судьба свела,
вот туда ты идёшь.

\I
Да. И когда ты уже
чем-то занимаешься, неважно чем,
чем больше занимаешься, тем больше
это раскрывается для тебя.
А для меня, например,
вот сейчас в тайцзи
понятие столба, я вообще
это не понимал, а сейчас для меня
это уже как бы вошло... Ну, то есть,
постепенно вот эти все...
Я когда сюда попал, для меня это,
ну, как бы с одной стороны странно,
что люди, которые
понимают, то есть, на
одном языке можешь говорить,
потому что в нашем городе это бесполезно.
Даже просто там вот эти слова
там какие-то, тебя не поймут просто.

\A
Ну, понятно.
Ну, и второй момент это то, что
в разном возрасте разные
как бы ограничения
существуют и разные задачи.
То есть, одна и та же
общая задача
развивать физическое
тело и вот этот храм души
поддерживать.
Когда тебе
20, это
один момент. Когда тебе 40 --- другой.
Когда тебе 60 -- третий.
Известный пример это
Бола Йенг, который
приятель
Вандама по фильму
«Кровавый спорт». Значит,
он до
45 лет
был президентом
Гонконговской федерации бодибилдинга.
Сам постоянно качался.
Значит,
ну, и снимался в кино.
В основном занимался
такими жёсткими видами
единоборств, типа Хунгара
какого-нибудь там, школа тигра.
Вот.
Ну, такой гигантский мужик был.
Вот. Потом
после 45-и,
ну, как бы здоровье-то уже не то,
он
остался почётным
президентом
Федерации бодибилдинга. Тренироваться
перестал. Сейчас
занимается тайцзи.
Фигура совершенно поменялась. Фигура,
ну, вот, как у Никиты.
Даже не как у Никиты,
как у нашего
Сергея «чёрненького».
То есть, он щупленький вообще-то.
Как его на этом бодибилдинге разнесло,
вообще непонятно.

\L
Химия.

\A
Ну, наверное,
химия. Ну, хотя в те годы
ещё химия не очень использовалась.
Вот. И всё. И занимается тайцзи.
Прекрасно себя чувствует. Ему сейчас уже
под 80. Вот. Он спортсмен.
Он... Даже в каких-то фильмах, по-моему, снимается ещё.
А его сын, опять-таки,
фигачит бодибилдинг.
Опять-таки, это как бы
по стопам папы идёт.
Ему сейчас где-то под
сорокет, сыну.
Ну, так же качается.
Раздутый весь.

\I
Тайцзи ещё плюс, что можно заниматься
в разных местах. То есть,
с теми же палками, там,
на природе. То есть...\
в парке, просто вот
минута свободная --- раз, ты позанимался.
Тебе не надо инвентаря.
Ну, и опять же, это как часть
культуры. То есть, если бы
вот у нас в городе, допустим,
организовалось бы что-нибудь или я не знаю.
Вообще, в принципе, общение...\
как некий интерес общий.

\A
Главное, чтобы тебе самому
интересно было. Будет интересно,
либо ты
дойдёшь до того
уровня, когда сможешь
группу собирать.
Ну, до уровня понимания,
до уровня, там, какой-то
трансляции. Я же вот, например,
никогда не собирался преподавать.
У меня... Ну,
преподавать именно боевые искусства.
У меня цели в жизни
совершенно другие были. Я психолог.
У меня есть профессия.
Я в Новосибирск приехал...
Вообще никак не связано
было с тайцзицюань.
Из Москвы приехал. Вот.
Ну, так получилось, что остался здесь
и начал искать учителя.
Учителя нету.
То есть, как бы ездить к учителю в Москву,
ну, такое себе
удовольствие из Новосибирска.
Значит, ну,
что делать? Пришлось... А!
Значит, начал обзванивать
секции айкидо, потому что я раньше
ещё айкидо занимался. Думаю, ну,
хотя бы в айкидо буду ходить.
Значит,
позвонил по объявлению,
а мне говорят,
``А у вас опыт есть
занятий айкидо?'' Я говорю, ``Есть. Я год
занимался.'' ``О! Нам инструктор
нужен.'' Я говорю, ``Всё понятно с вами.''
Если годовалый
человек вам как инструктор сойдёт.
Ну, нет, спасибо.
Вот. Ну, и
пришлось свою группу собирать.
То есть, просто чтобы хоть с кем-то
тренироваться. Потом
уже и учитель появился.
Был такой
период. Вот.
Либо второй вариант. Например,
ну, соберём какой-нибудь семинар,
приедем к вам, проведём занятия.
Таким способом
иногда группа
собирается. Ну, то есть,
вариантов прям много.

\I
Я ещё вообще
как бы имею...
Да, это тоже. И
дополнительно как бы
к этому: есть
некая
сфера интересов, где
на одном языке люди говорят
и на определённых
понятиях, то есть, уже
друг друга
понимают. И тайцзи
оно вот это всё в себе собирает:
тут и туйшоу, и
все эти комплексы, и все эти движения,
и с палками, и так, и так.
Это уже как бы формирует
определенный
коллектив. То есть,
уже саму по себе культуру какую-то
создаёт.

\A Да, не просто коллектив,
а поле.

\I Да, поле, поле.
Как вот среди негров
баскетбол, да, к примеру. Ну, это мем, да.
И здесь вот нечто такое,
которое достаточно ёмкое, чтобы
в себе всё это собрать.

\A
Я понимаю.
Я сейчас с младшим сыном
в баскетбол играю. Я в молодости
баскетболистом был.
Потом боевые искусства победили.
То есть, у меня в какой-то момент серьезный
выбор был, где-то на уровне
четвёртого курса.
Меня звали в сборную
играть в баскетбол.
И при этом я интенсивно
тренировался по ушу. В какой-то момент
я прям понимаю, что меня
разрывает. Я такой
всё --- ушу это серьезно, а баскетбол
ладно. По остаточному принципу.
Вот.
Но тут вот с младшим сыном
начали ходить на площадку.
Я тут же прям
начал вспоминать все эти
там термины, приёмы,
упражнения.
А там другие ребята приходят.
То есть, вот тут
культура общения, она очень много
значит.

\I
Да, да, да. Это поле, да.

\A
Да, поле общее. Там интересы,
ценности. Когда есть
о чём поговорить.

\I
Да, именно принципы вот эти все
базовые, что там...
Для тайцзи которые характерны.

\A
Учитывая, что ты человек
такой методичный,
постепенно разберёшься.
Ну, а насчёт видео --- надо
прям аккуратнее, потому что...
Ну
опять, на начальных этапах это не страшно.
На начальных этапах
мы такое смотрели, что
по сравнению с этим любой
современный преподаватель
просто огонь.
То есть, мы учились по каким-то жутко корявым
записям.
По принципу, кто-то где-то что-то
подсмотрел, показал.
Ну, как это...
Мойша,
значит, напел
Паваротти, да, так называется.
Так этот поворотти, как выяснилось,
картавит, шепелявит
и фальшивит в каждой ноте.
Вот. То есть, у нас вот такой
был момент, когда доступа
к нормальной информации не было.
И когда приехал первый внятный
китаец, мы на него как на Бога смотрели
просто. Когда вот китаец,
который именно по школе,
который аутентичный, который
с детства тренировался, а не какой-то там...

\I
Самозванец.

\A
Ну, не то, что... Понимаешь, есть же не самозванцы,
есть, знаешь, какие...
К нам однажды обратился, Лера, помнишь?
Этот китаец. Приехал
чувак в Новосибирск
учиться
ну, там, по обмену
или как там, в
педуниверситет приехал.
Значит, нашёл нас.
Говорит, ``Здравствуйте, я
предлагаю вам сотрудничество.
Меня зовут так-то, так-то.
Я преподаватель тайцзицюань.''
И мы такие, ``Иинтересно.
Ну, а поподробнее.''
Он говорит, ``А ещё я преподаватель шаолиньцюань, мэйхуа, там,
танланцюань.'' В общем, 15
стилей преподаёт.
Я такой, ``Так, стоп.''
Чего к чему. Короче,
он нахватался там, пока
в школе учился.
Всем подряд занимался, всякие
комплексы выучил.
Ему кажется, что вот он
может преподавать.
Он этим тайцзицюанем, может, два месяца
занимался. Но он же приехал в
дикую Россию, где
как бы варвары, которые
ничего не понимают в высоком искусстве.
Значит, им можно
всё это преподавать, деньги
за это брать. ``Ну, приходите, посмотрим
на ваше искусство.'' Он так и не
пришёл. ``Приходите ---
посмотрим, на что вы способны.''
Не пришёл. Не решился.
То есть, таких китайцев
полно просто.

\I
Но они, наверное, на это и
ловят, что люди думают,
что раз мы китайцы, значит, мы всё знаем.

\A
Да, да, конечно.
В древнем монастыре изучали.
И лично даосский
патриарх посвящение
провёл.
И так далее. И 15 предков
за спиной, которые там...\ мечи
на стене хранятся фамильные.

\L
Была ещё подобная история.
Звонит мне...\
по организации какого-то мероприятия...\
какой-то фестиваль в городе был.
Ну, на уровне города прям.
И говорит: ``К нам приезжает
китаец. Ну, то ли
мастер-класс будет показывать, то ли что.
И у него будет ещё время.
Если хотите, он вам проведёт
семинар.'' Я говорю, ``А как зовут китайца?
Ну, кто вообще?''
``Мастер, китаец.''
Я говорю, ``А зовут как?
Кого вы хотите
нам впихнуть в уроки?''
Они такие, ``Ну,
типа, чего непонятно?''

\A
``Чего выпендриваетесь? Китаец приехал.''

\L
Китаец приехал, блин, ну.

\A
В сам\'ом Китае
это признак
либо обмана, либо
неуважения.
То есть уважающий себя мастер
должен назвать минимум
три поколения учителей, у которых он
обучался.

\L
Какой стиль? Что вообще?

\A
То есть это просто вопрос в приличии.
Если ты не можешь назвать трёх
поколений своих мастеров,
ты сам не можешь
считаться мастером.
Второй момент,
это всё проверяется.
То есть, если ты назвал кого-то у кого ты
не учился, это можно
позвонить, это можно написать письмо.
Если ты соврал, ну...\
вплоть до того, что побить могут.

\I
Опять же, вот мы когда по лекциям-то
там было три аспекта.
Это
оздоровительный, потом...\
ну, как для
обороны, да, боевой.
И третий аспект -- это
именно, ну, как
джентльменский
кодекс для саморазвития.

\A
Духовный.

\I
Каждый, да, уважающий
себя, как это сказать,
китаец, или как это сказать, который
хотел называть себя, грубо говоря,
джентльменом, да, он занимался...\
вот это
было его. То есть...\
в том аспекте,
когда мы сравнивали бодибилдинг,
да, и вот тайцзи, то есть
тут
именно тайцзи вот это даёт...\
вот этот третий
аспект, который в лекциях упоминался.
А бодибилдинг уже тут
такого подтекста же не имеет, да?

\A
Скажем так, по умолчанию
не имеет. То есть, наверное,
отдельные мастера бодибилдинга
и давали какую-то
философию, но это уже
в самой
методике не было прописано.
Из чего, кстати, произошли
всякие забавные штуки,
типа там, славянских
каких-то практик,
когда брали, грубо говоря,
ну, буквально,
мастер спорта по тяжёлой атлетике
создавал славянскую
школу
«древней традиции»,
в которой поднятие
штанги сочеталось с пением
каких-то славянских гимнов.
Ну, потому что
людям нужна была вот эта
духовная философская составляющая.
И он им её
просто придумывал
там, как-то
наглючил и так далее. Это ж известная
история.

\N
Зато красиво.

\A
Да. Причём это всё отслеживалось по источникам,
что сначала человек просто штангист
ведёт занятия
или какой-ни\-будь там боксёр,
хороший боксёр, ну, просто боксёр.
Потом он
начинает заявлять о том, что он
обучался в Китае.
Вот. И что он
уже не просто боксёр, а, например,
какой-нибудь мастер шаолинь.
А то, что он штангист
ещё --- ну, как бы
уже где-то сбоку.
Потом он говорит, что вообще-то
дед у него был
волхвом. Это всё
в статьях, прям...\ это же сейчас...\
ну, источниковедение
развито. Интернет многое сохраняет.
В архивах многое есть.
Дед был волхвом.
И оказывается, дед, когда он был
маленький, дал ему
эти тайные знания, а потом
он, ну, просто дед был старенький,
он пошёл в штангисты там и прочее,
прочее. И в какой-то
момент он говорит, ``Ну, всё, я наследник древних
волхвов, и поэтому
вот будем делать так, как
наши предки пять
поколений, пять веков назад
в глубинах Сибири,
в тайге.''

\N
``Культивировали.''

\A
Да, да, да.
А потом эти все китайцы у нас всё это
переняли, понимаете. Это же русские
всё придумали, русские.
Так интересно смотреть эволюцию.
Человек
как бы...

\L
Растёт.

\A
Да, растёт над собой.

\I
Вот, и в этом плане как раз
тайцзи, оно более
структурировано и уже более проверено
временем, и более
достоверно и авторитетно,
чем вот эти всякие боковые, которые
сейчас развелись течения там всякие.
Илицюань
всякие там, или вот эти
контактники\footnote*{Я имел ввиду не ММА, а контактную импровизацию,
которая на верхней базе проводилась.}, они же всё об одном, только
они как, как эти,
шарлатаны. Ну, короче,
на одном и том же --- просто
создают из воздуха.
А основу если брать,
то её было бы достаточно.

\A
Ну,
контактные единоборства,
у них там
цель другая. У них цель
с одной стороны шоу,
ну, тот же самый ММА,
это же всё-таки шоу,
а с другой стороны они
постоянно в поиске,
то есть там...\ я просто
немножко читал, ну,
интересовался ММА,
у меня знакомые есть, которые им занимаются.
Там идея, например, что
создаются новые техники,
которые контрят старые,
то есть там, например,
пришли в ММА боксёры ---
их законтрили
кикбоксёры. Потом
кикбоксёров законтрили
борцы-вольники,
которые начали их просто вязать.
Вольников законтрили тайские
боксёры, которые начали локтями
рубить. Тайских законтрили
джиуджитсовцы,
которые начали
валить их на пол и там месить.
И так постепенно, постепенно, постепенно ---
как бы вот.
Современная ММА ---
это уже какая-нибудь
там двадцатая итерация...\
ну как бы тактики,
стратегии и так далее.
В сравнении с тем что было
когда он только начался.

\L
Сейчас они все по большей части
борются.

\A
Это просто проще.
Бороться проще.
Есть ударники в ММА.
Есть каратисты в ММА.
Есть...\ ну, разные.

\L
Они сначала, да, начинают там чего-нибудь...
Просто уход вниз у них у всех разный.
А по сути потом --- бах!
И всё --- они на полу.

\A
Есть те, которые не любят бороться.
По разному.
Больше --- да.

\I
Просто что
если бы была некая
авторитетная, общепризнанная
одна школа,
людям бы, ну вообще --- неподготовленным,
было бы проще,
потому что все сразу
как-бы настороженно:
``Секта, секта, секта, секта.''
То есть это...
Ну или как там это сказать,
что-то такое непонятное.
У всех сразу квадратные глаза.
А если бы это было
как общепринято,
как йога
типа того что,
ну уже не такое было бы у людей...
Потому что начинаешь говорить
«тайцзицюань»,
``А что это?
А как это?''

\A
Ну, дело в том что опять-таки,
йога --- она общепринятая
только в нашем представлении.
В самой Индии
даже хатха-йога,
хотя хатха-йога
не единственная йога,
даже хатха-йога --- это не единое учение.
Последователи школы Айенгара
вообще считаются еретиками,
потому что они отказались
от традиции
именно духовного самосовершенствования.
То есть Айенгар
первым начал говорить,
что вот растяжка важнее
чем духовное самосовершенствование.
Поэтому...
А у нас
Айенгара больше преподают,
чем других,
больше знают.
И есть те же самые сектанты
среди йогов.
Есть очень жёсткие сектанты.
То есть,
это нам так кажется,
что это всё одинаковое.
Но другой вопрос,
что есть некая
-- как сказать ---
культура, и есть некая
государственная поддержка.
Вот это вот у них есть.
Это очень важно, что
люди, которые занимаются
хатха-йогой ---
им проще получить
господдержку
в плане, там, открытия залов,
открытия федераций.
То есть это
национальное,
это работает на поддержку
национального какого-то духа
и прочее.

\I
И в интернете везде эта
реклама: хатха, хатха...
На слуху.

\A
Ну да, на слуху.
Но, кстати, индусы
планомерно проводят политику.
Это называется мягкая сила.
Китайцы тоже это научились делать.
То есть продвижение
своих культурных ценностей
через йогу,
боевые искусства,
медицину, и так далее, и так далее.
То есть вот американцы
продвигают свои культурные ценности
через Голливуд,
потому что все мы смотрим
американские фильмы.
Сначала мультики,
а потом взрослые фильмы.
А китайцы, например,
через китайскую медицину
и через тайцзицюань, цигун.
Индусы --- через йогу.
Это просто
такой вот...
Та же самая,
например,
Бразилия ---
капоэйра,
да.
Ну вообще-то
капоэйра --- это
увлечение бомжей.
То есть
уличной гопоты,
условно говоря.
Но из этого сделали
красивую систему
с ритуалами,
с музыкой,
с одеждами красивыми.
Ну почему ---
потому что
это
формирует позитивный образ страны
в глазах иностранцев.
То есть у нас не гопники
сидят у костров,
там, курят
анашу какую-нибудь
и бьют прохожих.
А у нас люди занимаются
древним боевым искусством.
И всех готовы к нему,
так сказать,
приобщить за небольшую плату.
То есть это вот
мягкая сила.
Это сейчас многие используют.

\I
Ну вот это здорово,
что вот
в Томске вот занимаются
Никита, Ирина, Ольга.
То есть вот
что у них такая прям
секция есть.

\L
Лера.

\N
Ты преподаёшь.

\A
Мы в Томске,
вот опять,
чтобы это всё было возможным,
мы в 2005 году
начали ездить
в Томск
ежегодно.
Иногда у нас даже
в год было
два-три семинара
в Томске.
То есть я
в какой-то момент,
меня уже люди
начали спрашивать
``Ты в Томск переехал?
Всё?''
Я говорю: ``Какой переехал?
У меня тут семья.
У меня...
Какой переехал?''
Но я дорогу
Новосибирск Томск
выучил
с точностью до
кирпича,
до выбоин
в асфальте.
Поэтому
просто
получается
сколько ---
20 лет,
вот,
как на работу
я просто ездил
в Томск.
Это очень большой
такой процесс,
долгий.
Сейчас, да.
Сейчас
уже в Томске
наверное
каждый второй
кто занимается
тайцзи ---
это
либо наша школа,
либо
мои ученики,
которые создали
какие-то свои
уже...

\I
Вообще
в 90-е же
пришёлся
расцвет.
Когда
фильмы
вот эти
появились.
Кассеты.
«Видики».

\A
Ну вот
как раз
то что ты говоришь:
вот эта
делегация
Госкомспорта
привезла.
То есть
до этого
это было
всё полулегально.
А
когда
это всё
приехало
официально,
то начали открываться клубы, секции.
Я начал заниматься в Москве в 90-ом
году.
К тому времени
где-то уже
года два
уже всё это
пышным цветом
расцветало.
Учитель
говорит,
что
в Питере
это всё
появилось
где-то
в 87--88 году.
То есть
чуть-чуть
пораньше,
чем в Москве.

\I
Да,
то есть 90-е
как бы
и у нас
в городе
вот было
две секции.
Да.
А потом
как-то это всё на «нет» сошло.

\A
Ну... Это сложно.
А сложностей люди не любят.
Даже
вот
элементарно
где-нибудь
там,
ну не знаю,
в поезде,
в самолёте,
разговор заходит ---
я говорю, там ``У меня тренировки.''
``А чем ты занимаешься?''
Я говорю ``Ну вот, ушу.''
``А, ты спортсмен. Понятно.''
То есть
людям
очень сложно
объяснить: ``Я не спортсмен,
это боевые искусства,
это вообще ничего общего
со спортом
не имеет.''
``Ты спортсмен. Всё с тобой понятно.''
Так проще. Люди любят упрощать всё.
А вот эти вот
духовные
саморазвития...
``Ты спортсмен.
Ты же тренируешься --- всё.''

\I
Да, и вот
по интернету
если учиться,
там
такой
есть
подвох,
что если
много источников ---
тяжело
сориентироваться,
особенно
новичку.
То есть
он «разбрасывается».

\A
Да.
Ну поэтому
выбирать, выбирать, выбирать...
Лера,
срочно
нужно
сделать...
Что?
Закреп. Для новичков.
Закреп нужно сделать срочно.
И всю
информацию
компактно
собрать.
Ну
ладно,
давайте
спать
наверное.

\I
Ну.

\A
У нас
конечно
не в пять утра
начинаются
занятия,
но
тоже
утром.

\kern2cm
\it
\lineskip=7pt
\hskip6cm Июнь 2025 г. \par
\hskip4cm Горный Алтай \par
\hskip5cm База отдыха «Центр Дао» \par
\kern3cm
\hskip2cm \tensl Тренер: Андрей Владимирович Ширай

\bye
