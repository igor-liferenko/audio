%&12pt
\pdfpagewidth=297mm
\pdfpageheight=210mm
\pdfhorigin=1in
\pdfvorigin=0pt
\input quire
\shhtotal=\pdfpagewidth
\htotal=.5\shhtotal
\vtotal=\pdfpageheight
\shoutline=0pt
\shstaplewidth=0.1pt \shstaplelength=.5\vtotal
\shcrop=0pt
\shfootline={}
\shthickness=0pt
\quire{4}

\horigin=9mm % outer margin
\hsize=\htotal \advance\hsize by-2\horigin \advance\hsize by-9mm % additional width for inner margin
\dimen0=\htotal \advance\dimen0 by-\hsize \advance\dimen0 by-\horigin \horigin=\dimen0
\hoffset=\htotal \advance\hoffset-\hsize \advance\hoffset-2\horigin
\output={\ifodd\pageno\hoffset=0pt\fi \plainoutput}
\vorigin=15mm
\vsize=\topskip \advance\vsize by35\baselineskip

\font\TENSL=OMSL10
\headline={\rlap{\vbox to0pt{\vss\hbox{\raise5pt\line{\TENSL \ifodd\pageno\hfil\fi
  Разговор у костра\kern.5pt\ifodd\pageno\else\hfil\fi}}}}\line{\hrulefill}}
\footline={\raise3pt\line{\hss\tenrm\folio\hss}}

\font\speakerF=omssbx10
\def\A{\item{\speakerF А.}}
\def\I{\item{\speakerF И.}}
\def\L{\item{\speakerF Л.}}
\setbox0=\hbox{\speakerF А.\enskip}
\parindent=\wd0

\A
... со знанием программ развития, реабилитации и так далее.
То есть нормальное советское образование.
И, соответственно, когда это всё появилось,
резко в Америке появились клубы, появились методики,
появилось понятие индивидуального тренера.
Потому что стало ясно, что можно обучать людей на условно-научной основе.
То есть в этом смысле в качалке тоже не всё так просто.

\I
Мне в качалке как-то меньше понравилось.
Я если днём устаю --- раз, кусочек комплекса сделал --- уже размялся.

\A
Это понятно, я с тобой согласен.
Мне тоже качалка не очень интересна.
Я дома иногда гири тягаю, по своим соображениям:
в какой-то момент не хватает физухи, и надо её добавлять.
У нас все, кто занимается больше 10--15 лет, в какой-то момент добавляют физуху.
Я для себя выбрал гири. И иногда работу с тяжёлым оружием.
Алебарды тяжёлые, ещё что-то.

\I
Тут основной момент в чём для меня лично заключается:
что мозг, он как бы...
То есть, занятия должны быть регулярными.
И я пришёл к тому, что регулярность, она появляется только...
Допустим, один человек может просто 5 минут просидеть,
а другой человек за 5 минут может какой-то комплекс сделать.
То есть, нужен какой-то фактор, который как-бы страгивает с места.
Чтобы это в привычку вошло.
А чтобы это вошло в привычку, нужно
чтобы оно с одной стороны не вызывало противодействия внутри мозга.
``Вот опять вставать, там, тягать.''
Чтобы это и чуть-чуть нагружало, и при этом давало бы некий приятный момент.
И вот как раз эти комплексы не вызывают такого внутреннего подсознательного протеста.
И моему мозгу легче их принять на регулярной основе.
Я поэтому к этому вот способу и пришёл, вот эти комплексы, вот это всё.
А качалка, мне её тяжело сделать регулярной.
Это надо конкретно, прям регулярно.
Если не будет регулярности, процесса не будет, толку не будет.
То есть надо прям,
чтобы мозг знал, что вот в этот день, у тебя в это время вот это,
чтобы уже прям на подкорке всё это было.
Тогда это для меня работает.
А иначе я буду какие-то искать оправдания: что-то, куда-то, чего-то.
И именно я пришёл к тому, что вот эти комплексы, они как раз для моего мозга
проще дают как бы разбег вот этот, чтобы в привычку всё это вошло.
И мы вот каждый день занимаемся и с одной стороны,
как бы утром неохота вставать, а потом
пришёл, цигунчик поделал,
и у тебя по телу всё приятно, и тебе приятно,
и тебе на следующий день уже охота
идти туда.
То есть, как бы
сильно не нагружать, и при этом
чуть-чуть прорабатывать. А потом уже
в конце и нагрузочки маленько можно
дать. И оно постепенно,
постепенно идёт.
Вот так.
Может просто, что
тут ещё аура добавляется некая.
То есть, вот эти все
фишечки восточные всякие,
такие мелочи. А спортзал
он как бы, ну, не
имеет некой
оболочки, да, упаковки
такой красивой.
Культуры, скажем так. Для меня,
по крайней мере. Может, если углубиться, да.

\A
Ну, или просто ты не нашёл
те залы, где это есть.
То есть, тут, как говорится, кому
с чем повезло
встретиться,
тот туда и пошёл.
Мне вот повезло встретиться
с боевыми искусствами.
А при этом, вот, например, мне
не повезло встретиться со скалолазами.
По крайней мере,
в раннем возрасте.
И когда я уже попробовал
скалолазы, я уже был достаточно взрослый.
Мне уже лениво было переучиваться.
Или, например, бокс.
Вот не повезло мне.
Не было интересной
секции бокса.
В наше время вообще бокса было мало.
Где-то там существовал,
как некое такое, что-то
спортивное-спортивное совсем.
Вот.
Зато рукопашный бой повезло.
Десантники у нас
в Туле, где я рос,
создали военно-патриотический
клуб, и я там занимался
рукопашным боем.
Ну, короче, вот
такие моменты.
Один из наших
преподавателей, Роман Романов,
говорит ``Судьба.''
С чем тебя судьба свела,
вот туда ты идёшь.

\I
Да. И когда ты уже
чем-то занимаешься, неважно чем,
чем больше занимаешься, тем больше
это раскрывается для тебя.
А для меня, например,
вот сейчас в тайцзи
понятие столба, я вообще
это не понимал, а сейчас для меня
это уже как бы вошло... Ну, то есть,
постепенно вот эти все...
Я когда сюда попал, для меня это,
ну, как бы с одной стороны странно,
что люди, которые
понимают, то есть, на
одном языке можешь говорить,
потому что в нашем городе это бесполезно.
Даже просто там вот эти слова
там какие-то, тебя не поймут просто.

\A
Ну, понятно.
Ну, и второй момент это то, что
в разном возрасте разные
как бы ограничения
существуют и разные задачи.
То есть, одна и та же
общая задача
развивать физическое
тело и вот этот храм души
поддерживать.
Когда тебе
20, это
один момент. Когда тебе 40 --- другой.
Когда тебе 60 -- третий.
Известный пример это
Бола Йенг, который
приятель
Вандама по фильму
«Кровавый спорт». Значит,
он до
45 лет
был президентом
Гонконговской федерации бодибилдинга.
Сам постоянно качался.
Значит,
ну, и снимался в кино.
В основном занимался
такими жёсткими видами
единоборств, типа Хунгара
какого-нибудь там, школа тигра.
Вот.
Ну, такой гигантский мужик был.
Вот. Потом
после 45-и,
ну, как бы здоровье-то уже не то,
он
остался почётным
президентом
Федерации бодибилдинга. Тренироваться
перестал. Сейчас
занимается тайцзи.
Фигура совершенно поменялась. Фигура,
ну, вот, как у Никиты.
Даже не как у Никиты,
как у нашего
Сергея «чёрненького».
То есть, он щупленький вообще-то.
Как его на этом бодибилдинге разнесло,
вообще непонятно.

\L
Химия.

\A
Ну, наверное,
химия. Ну, хотя в те годы
ещё химия не очень использовалась.
Вот. И всё. И занимается тайцзи.
Прекрасно себя чувствует. Ему сейчас уже
под 80. Вот. Он спортсмен.
Он... Даже в каких-то фильмах, по-моему, снимается ещё.
А его сын, опять-таки,
фигачит бодибилдинг.
Опять-таки, это как бы
по стопам папы идёт.
Ему сейчас где-то под
сорокет, сыну.
Ну, так же качается.
Раздутый весь.

\I
Тайцзи ещё плюс, что можно заниматься
в разных местах. То есть,
с теми же палками, там,
на природе. То есть...\spacefactor=1000
в парке, просто вот
минута свободная --- раз, ты позанимался.
Тебе не надо инвентаря.
Ну, и опять же, это как часть
культуры. То есть, если бы
вот у нас в городе, допустим,
организовалось бы что-нибудь или я не знаю.
Вообще, в принципе, общение...\spacefactor=1000
как некий интерес общий.

\A
Главное, чтобы тебе самому
интересно было. Будет интересно,
либо ты
дойдёшь до того
уровня, когда сможешь
группу собирать.
Ну, до уровня понимания,
до уровня, там, какой-то
трансляции. Я же вот, например,
никогда не собирался преподавать.
У меня... Ну,
преподавать именно боевые искусства.
У меня цели в жизни
совершенно другие были. Я психолог.
У меня есть профессия.
Я в Новосибирск приехал...
Вообще никак не связано
было с тайцзицюань.
Из Москвы приехал. Вот.
Ну, так получилось, что остался здесь
и начал искать учителя.
Учителя нету.
То есть, как бы ездить к учителю в Москву,
ну, такое себе
удовольствие из Новосибирска.
Значит, ну,
что делать? Пришлось... А!
Значит, начал обзванивать
секции айкидо, потому что я раньше
ещё айкидо занимался. Думаю, ну,
хотя бы в айкидо буду ходить.
Значит,
позвонил по объявлению,
а мне говорят,
``А у вас опыт есть
занятий айкидо?'' Я говорю, ``Есть. Я год
занимался.'' ``О! Нам инструктор
нужен.'' Я говорю, ``Всё понятно с вами.''
Если годовалый
человек вам как инструктор сойдёт.
Ну, нет, спасибо.
Вот. Ну, и
пришлось свою группу собирать.
То есть, просто чтобы хоть с кем-то
тренироваться. Потом
уже и учитель появился.
Был такой
период. Вот.
Либо второй вариант. Например,
ну, соберём какой-нибудь семинар,
приедем к вам, проведём занятия.
Таким способом
иногда группа
собирается. Ну, то есть,
вариантов прям много.

\I
Я ещё вообще
как бы имею...
Да, это тоже. И
дополнительно как бы
к этому: есть
некая
сфера интересов, где
на одном языке люди говорят
и на определённых
понятиях, то есть, уже
друг друга
понимают. То есть, именно тайцзи
оно вот это всё в себе собирает:
тут и туйшоу, и

все эти комплексы, и все эти движения,

и с палками, и так, и так.

Это уже как бы формирует

определенный

коллектив. То есть,

уже сама по себе культуру какую-то

создает. Да, не просто коллектив,

а поле. Да, поле, поле.

То есть, например, как вот среди негров

баскетбол, да, например. Ну, это мем, да.

И здесь вот нечто такое,

которое достаточно емкое, чтобы

в себе все это собрать.

Я понимаю.

Да.

Я сейчас с младшим сыном

в баскетбол играю. Я в молодости

баскетболистом был.

Потом боевые искусства победили.

То есть, у меня в какой-то момент серьезный

выбор был, где-то на уровне

четвертого курса.

Меня звали в сборную

играть в баскетбол.

И при этом я интенсивно

тренировался по ушу. В какой-то момент

я прям понимаю, что меня

разрывает. Я такой

все, ушу это серьезно, а баскетбол

ладно. По остаточному принципу.

Вот.

Но тут вот с младшим сыном

начали ходить на площадку.

Я тут же прям

начал вспоминать все эти

там термины, приемы,

упражнения.

А там другие ребята приходят.

То есть, вот тут

культура общения, она очень много

значит. Да, да, да. Это поле, да.

Да, поле общее. Там интересы,

ценности. Когда есть

о чем поговорить.

Да, именно принципы вот эти все

базовые, что там.

Тайцзи, которые характерны.

Вот. Учитывая, что ты человек

такой методичный,

постепенно разберешься.

Ну, а насчет видео надо

надо прям аккуратнее, потому что

опять, на начальных этапах это не страшно.

На начальных этапах

мы такое смотрели, что

по сравнению с этим любой

современный преподаватель

просто огонь.

То есть, мы учились по каким-то жутко корявым

записям.

По принципу, кто-то где-то что-то

подсмотрел, показал.

Ну, как это...

Мойша,

значит, напел

поворотти, да, так называется.

Так этот поворотти, как выяснилось,

картавит, шепелявит

и фальшивит в каждой ноте.

Вот. То есть, у нас вот такой

был момент, когда доступа

к нормальной информации не было.

И когда приехал первый внятный

китаец, мы на него как на Бога смотрели

просто. Когда вот китаец,

который именно по школе,

который аутентичный, который

с детства тренировался, а не какой-то там...

Самозванец.

Ну, не то, что... Понимаешь, есть же не самозванцы,

есть, знаешь, какие...

К нам однажды обратился, помнишь?

Этот китаец. Приехал

чувак в Новосибирск

учиться

ну, там, по обмену

или как там, в

педуниверситет приехал.

Значит, нашел нас.

Говорит, здравствуйте, я

предлагаю вам сотрудничество.

Меня зовут так-то, так-то.

Я преподаватель тайзи цюань.

И мы такие, интересно.

Ну, а поподробнее.

Он говорит, а еще я преподаватель шаолин

цюань, мэйхуа, там,

танлан цюань. В общем, 15

стилей преподает.

Я такой, так, стоп.

Чего к чему. Короче,

он нахватался там, пока

в школе учился.

Всем подряд занимался, всякие

комплексы выучил.

Ему кажется, что вот он

может преподавать.

Он этим тайзи цюаньем, может, два месяца

занимался. Но он же приехал в

дикую Россию, где

как бы варвары, которые

ничего не понимают в высоком искусстве.

Значит, им можно

все это преподавать, деньги

за это брать. Ну, приходите, посмотрим

на ваше искусство. Он так не

пришел. Приходите,

посмотрим, на что он способен.

Не пришел. Не решился.

То есть, таких китайцев

полно просто.

Но они, наверное, на это и

как ловят, что люди думают,

что раз мы китайцы, значит, мы все знаем.

Да, да, конечно.

В древнем монастыре изучали.

И лично даосский

патриарх посвящение

провел.

И так далее. И 15 предков

за спиной, которые там мечи

на стене хранятся фамильные.

У нас был фестиваль в городе.

Какой-то клуб. К нам приезжает

китаец. Ну, то ли

мастер-класс будет показывать, то ли что.

И у него будет еще время.

Если хотите, он вам проведет

семинар. Я говорю, а как зовут китайца?

Ну, кто вообще?

Мастер, китаец.

Я говорю, а зовут как?

Кого вы хотите

нам впихнуть в уроки? Чьи?

Они такие, ну,

типа, чего непонятно?

Чего вы пендриваете? Китаец приехал.

Китаец приехал? Блин, ну.

Просто в самом Китае

это признак

либо обмана, либо

неуважения.

То есть уважающий себя мастер

должен назвать минимум

три поколения учителей, у которых он

обучался. Какой стиль? Что, в общем?

То есть это просто вопрос в приличии.

Если ты не можешь назвать трех

поколений своих мастеров,

ты сам не можешь

считаться мастером.

Второй момент, это значит

что это все проверяется.

То есть, если ты назвал кого-то и у кого-то

не учился, это можно

позвонить, это можно написать письмо.

Если ты соврал, ну,

вплоть до того, что побить могут.

Опять же, вот мы, когда по лекциям-то

там было, три аспекта.

Это

здоровительный, потом

ну, как для

обороны, да, боевой.

И третий аспект, это

именно, ну, как

джентльменский

кодекс для саморазвития.

Каждый, да, уважающий

себя, как это сказать,

китаец, или как это сказать, который

хотел называть себя, грубо говоря,

джентльменом, да, он занимался

вот это

было его. То есть

в том аспекте,

когда мы сравнивали бодибилдинг,

да, и вот тайчи, то есть

именно тайчи, вот это дает

вот этот третий

аспект, который в лекциях упоминал.

А бодибилдинг уже тут

такого подтекста же не имеет, да?

Скажем так, по умолчанию

не имеет. То есть, наверное,

отдельные мастера бодибилдинга

и давали какую-то

философию, но это уже

в самой

методике не было прописано.

Из чего, кстати, произошли

всякие забавные штуки,

типа там, славянских

каких-то практик,

когда брали, грубо говоря,

ну, буквально,

мастер спорта по тяжелой атлетике

создавал славянскую

школу

древней традиции,

в которой поднятие

штанги сочеталось с пением

каких-то славянских гимнов.

Ну, потому что

людям нужна была вот эта

духовная философская составляющая.

И он им ее

просто придумывал

там, как-то

наглючил и так далее. Это ж известная

история.

Зато красиво.

Да. Причем это все отслеживалось по источникам,

что сначала человек просто штангист

ведет занятия

или какой-нибудь там боксер,

хороший боксер, ну, просто боксер.

Потом он

начинает заявлять о том, что он

обучался в Китае.

Вот. И что он

уже не просто боксер, а, например,

какой-нибудь мастер шаоли.

А то, что он штангист

еще, ну, как бы

уже где-то сбоку.

Потом он говорит, что вообще-то

дед у него был

волхвом. Это все

в статьях, прямо это вот сейчас,

ну, источниковедение

развито. Интернет многое сохраняет.

В архивах многое есть.

Дед был волхвом.

И оказывается, дед, когда он был

маленький, дал ему

эти тайные знания, а потом

он, ну, просто дед был старенький,

он пошел в штангисты там и прочее,

прочее. И в какой-то

момент он говорит, ну, все, я наследник древних

волхвов, и поэтому

вот будем делать так, как

наши предки пять

поколений, пять веков назад

в глубинах Сибири,

в тайге.

Культивировали.

Да, да, да.

А потом эти все китайцы у нас все это

переняли, понимаете. Это же русские

все придумали, русские.

Так интересно смотреть эволюцию.

Человек

как бы...

Растет.

Вот, и в этом плане как раз

тайцзи, оно более

структурировано и уже более проверено

временем, и более

достоверно и авторитетно,

чем вот эти всякие боковые, которые

сейчас развились в течение, там всякие.

Или цуань

всякие там, или вот эти

контактники, они же все об одном, только

они как, как эти,

шарлот. Ну, короче,

на одном и том же просто

создают из воздуха

основу, если брать,

то ее было бы достаточно.

Ну,

контактные единоборства

у них там

цель другая. У них цель

с одной стороны шоу,

ну, тот же самый ММА,

это же все-таки шоу,

а с другой стороны они

постоянно в поиске,

то есть там я просто

немножко читал, ну,

интересовался ММА,

у меня знакомые есть, которые им занимаются,

там идея, например, что

создаются новые техники,

которые контрят старые,

то есть там, например,

пришли в ММА боксеры,

их законтрили

кикбоксеры, потом

кикбоксеров законтрили

борцы-вольники,

которые начали их просто вязать,

вольников законтрили тайские

боксеры, которые начали локтями

рубить, тайских законтрили

джиу-джитсовсы,

которые начали

Валить их на пол и там месить

И так постепенно

Современные ММА

Это уже какая-нибудь

Двадцатая итерация

Тактики

Стратегии и так далее

В сравнении с тем что было

Когда он только начался

Сейчас они все по большей части

Борются

Это просто проще

Бороться проще

Есть ударники в ММА

Есть каратисты в ММА

Есть разные

Они сначала начинают

Уход вниз у них

А по сути потом бах

И все они на полу

Есть те которые не любят бороться

По разному

Больше да

Просто что

Если бы была некая

Авторитетная и вообще признанная

Одна школа

Людям бы

Было бы проще

Потому что все сразу

Наблюдают

Осторожно секта

Секта

Секта

Секта

То есть это

Ну или как там это сказать

Что-то такое

Непонятное

У всех сразу квадратные глаза

А если бы это было

Как вообще принято

Как йога

Типа того что

Ну уже не такое было бы у людей

Потому что начинаешь говорить

Тадзисуань

А что это?

А как это?

Ну дело в том что опять таки

Йога

Она общепринятая

Только в нашем представлении

В самой Индии

Даже хатха

Йога

Хотя хатха йога

Не единственная йога

Даже хатха йога

Это не единое учение

Последователи школы Айенгара

Вообще считаются еретиками

Потому что они отказались

От традиции

Именно духовного самосовершенствования

То есть Айенгар

Первым начал говорить

Что вот растяжка важнее

Чем духовное самосовершенство

Поэтому

А у нас

Айенгара больше преподают

Чем других

И больше знают

И есть те же самые сектанты

Среди йогов

Есть очень жесткие сектанты

То есть

Это нам так кажется

Что это все одинаковое

Но другой вопрос

Что есть некая

Как сказать

Культура

И есть некая

Государственная поддержка

Вот это вот у них есть

Это очень важно

Что

Люди которые занимаются

Хатха йогой

Им проще получать

Господдержку

В плане

Открытия залов

Открытия федераций

То есть это

Национальное

Это работает на поддержку

Национального какого-то духа

И прочее

И в интернете везде это

Реклама

На слуху

Ну да на слуху

Но кстати индусы

Планомерно проводят политику

Это называется мягкая сила

Китайцы тоже это научились делать

То есть продвижение

Своих культурных ценностей

Через йогу

Боевые искусства

Медицину

И так далее и так далее

То есть вот американцы

Продвигают свои культурные ценности

Через Голливуд

Потому что все мы смотрим

Американские фильмы

Сначала мультики

А потом взрослые фильмы

А китайцы например

Через китайскую медицину

И через тайцзюаньцы

Индусы через йогу

То есть вот это просто

Такой вот

Таблетка

Та же самая

Например

Бразилия

Капоэйра

Да

Ну вообще то

Капоэйра это

Увлечение бомжей

То есть

Уличной гопоты

Словно говоря

Но из этого сделали

Красивую систему

С ритуалами

С музыкой

С одеждами красивыми

Ну почему

Потому что

Это

Формирует позитивный образ страны

В глазах иностранцев

То есть у нас не бопники

Сидят у костров

Там курят

Анашу какую нибудь

И бьют прохожих

А у нас люди занимаются

Древним боевым искусством

Всех готовы к нему

Как сказать

Приобщить

За небольшую вкладку

То есть это вот

Мягкая сила

Это сейчас многие используют

Ну вот это здорово

Что вот

В Томске вот занимается

Никита там

И вот

Элина

Ходит

Ольга

То есть вот

Что у них такая прям

Секция есть

Димаш

Да

Да

Да

Да

Да

Да

Да

Да

То есть

Мы в Томске

Вот опять

Это

Чтобы это все было возможным

Мы в 2005 году

Начали ездить

В Томске

Ежегодно

Иногда у нас даже

В год было

Два-три семинара

В Томске

То есть я

В какой-то момент

Меня уже люди

Начали спрашивать

Ты в Томск переехал

Все

Какой переехал

У меня тут семья

У меня

Какой переехал

Но я дорогу

Новосибирск-Томск

Выучил

Терпича

До выбоев

В асфальте

Поэтому

Просто

Получается

Сколько

20 лет

Вот

Как на работу

Я просто ездил

В Томск

Это

Это очень большой

Такой процесс

Долгий

Сейчас

Да

Сейчас

Уже в Томске

Наверное

Каждый второй

Кто занимается

Паричи

Это

Либо наша школа

Либо

Мои ученики

Которые создали

Какие-то свои

Деньги

Книги

Вообще

В 90-е

Пришел

Расцвет

Когда

Фильмы

Вот эти

Появились

Кассеты

Митки

Ну

Вот

Как раз

То что ты говоришь

Делегация

Госкомспорта

Привезла

То есть

До этого

Это было

Все полулегально

А

Когда

Это все

Приехало

Официально

То

В 90-ом

Году

К тому времени

Где-то уже

Годы два

Уже все это

Хорошим цветом

Расцветало

Учитель

Говорит

Что

В Питере

Это все

Появилось

Где-то

В 87-88

Чуть-чуть

Пораньше

Чем в Москве

Да

То есть 90-е

Как бы

И у нас

В городе

Вот было

Две секции

Вот

Да

А потом

Даже

Вот

Элементарно

Где-нибудь

Там

Ну не знаю

В поезде

В самолете

Разговор заходит

Я говорю

Там

У меня тренировки

А чем ты занимаешься

Я говорю

Ну вот

Пушу

А

Ты спортсмен

Понятно

То есть

Людям

Очень сложно

Объяснить

Я не спортсмен

Это боевые искусства

Это вообще ничего общего

Со спортом

Не имеет

Ты спортсмен

А вот эти вот

Духовные

Саморазвития

Ты спортсмен

Ты же тренируешься

Да

И вот

По интернету

Если учиться

Там

Такое

Есть

Подвох

Что если

Много источников

Тяжело

Сориентироваться

Особенно

Новичку

То есть

Он разбрасывается

Да

Ну

Поэтому

Выбирать

Лера

Срочно

Нужно

Сделать

Что

Срочно

Срочно

И всю

Информацию

Компактно

Собрать

Ну

Ладно

Давайте

Спать

Ну

У нас

Конечно

Не в пять утра

Начинаются

Занятия

Но

Тоже

Утром

\bye
