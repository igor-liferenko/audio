

Эпиграф: ``Мерой твоего невежества служит глубина твоей веры в несправедливость и человеческую трагедию. То, что гусеница называет концом света, Учитель называет бабочкой.'' Ричард Бах. «Иллюзии».

0:00
История была интересная. Я во Вражке ночевал, то есть я в 13-го поехал в Вражку.

0:08
Ночевал утром, меня родители поздравили. Я забрался, на тренировку поехал.

0:16
У меня должна была тренировка быть в 7.30, я хотел на 6.44 ехать. Я проснулся в 4 часа.

0:22
Не хочу дома сидеть, поеду на 5.34 пораньше. А это у нас там на

0:31
Павлодарах, где вот хоккейный корт, там площадка спортивная. Мы там должны были с

0:36
Мариной Львовной тренировку делать. Она же ездила на Сибирск тоже, и она училась у

0:42
этого же. И мы с ней хотели по его методике, которую я нашлюсь, которую она ездила

0:47
передать, по ней позаниматься, по этой же системе. Договорились, значит. Я приехал на час

0:55
раньше, там поделал звуки, поделал. Ну там специальные звуки, короче, делаешь. И цыбун поделал.

1:01
Да.

1:02
18 форм. Потом она подошла, мы там с ней эти круги эти-то, ну всю эту программу провели.

1:09
И так хорошо, знаешь, все так срослось, так замечательно. Я потом пошел, ну так прям

1:14
погода, все так классно. Прям вообще идеал. Я иду, значит, мы все попрощались, она пошла,

1:22
я пошел. И пошел по дорожке туда, ну в сторону визит там, где моторинский магазин. И иду мимо

1:31
озона, там как раз озон. А я когда на йогу хожу, господь, я мимо него хожу. Я иду мимо озона,

1:37
я вспоминаю, что мне надо забрать кроссовки. Как нищий. Озон, кроссовки. Я такой, о, классно,

1:43
как раз я вспомнил. Зашел в озон, забрал кроссовки. Такой довольный. Иду домой, значит, думаю,

1:50
сейчас приедут там за мной, поедем там. Хотел на озеро Парк, план у меня был там.

1:56
Который новый открытый комплекс. Там побыть. Ну потом это,

2:02
ну это у меня был план. Потом мама звонит, говорит, папа повез тебе молоко. Ты выйди,

2:07
типа спусти, забери. Говорю, ладно. А я как раз к дому подхожу. Думаю, не буду подниматься. Подожду,

2:14
пока он привезет. Беру молоко, значит. А он должен уезжать там по делам. Ну, значит,

2:20
он привозит молоко. У меня это, кроссовки эти там, рюкзак с тренировки, сумочка эта. Я новую

2:26
купил перед этим, поездку. Я туда уже заранее паспорт положил, телефон. Она у меня болтается,

2:32
а там в машине что-то, какие-то сумки. Я, чтобы эта сумочка мне не мешала взять, я ее снял и положил

2:38
на крышу папе. Там паспорт и телефон. Собрал молоко, думаю, о, как классно все. Поднимаюсь,

2:45
тык-тык-тык, где телефон? Соседям стучу. Папе, дайте позвонить. Папе звоню, говорю,

2:53
срочно возвращайся, ищи, где телефон упал с крыши. С паспорта. Как ты? Ты что? Паспорт.

3:02
Мне уезжать через 5 дней. Короче, звоню Зое. А, пошел я, ну, прошел до набережной, нету телефона

3:13
нигде. Ну, сумки-то, да? Сумки. Поднимаюсь, опять соседям этой же стучу, говорю, дайте позвонить.

3:19
Звоню уже себе, сразу что-то не сообразил. Звоню на свой телефон, не отвечает. Я Зое звоню, говорю,

3:25
Зоя, звони периодически на мой телефон, вдруг кто-нибудь найдет. Ну, что делать? Ну, все, я...

3:32
Дома сел и сижу. Проходит где-то полчаса, стучит Рома. Говорит, я твой телефон нашел, говорит,

3:40
это. Я говорю, где? На газели, я говорю. Ну, какой газели? Он мне фотку показывает. Стоит, значит,

3:47
газель вдоль дороги, вот, которая идет мимо Багуна, я туда, в сторону набережной. Там рядом лужа

3:53
огромная. Вдоль дороги эта газель стоит. Значит, телефон висит на зеркале вот этой. Сумка. Сумка.

4:02
Висит на зеркале, где машина этой газели. Я так понимаю, папа ехал, телефон выпал. Сумка абсолютно

4:09
чистая. То есть, она рядом с лужей, видимо, упала, потому что газель стоит прямо рядом с лужей. Видимо,

4:14
кто-то увидел. Подумал, что это из этой газели выпало. Ну, там ничего не было. Паспорт и телефон.

4:20
Потому что люди, как правило, боятся такие вещи, с такими вещами играть. Наверное, посмотрели,

4:25
денег нет. Ну, может, просто даже порядочно попали. Просто, ну, думали, повесили, куда ее девать. А Рома

4:31
просто часы случайно нашел? Нет, он раз прошел. Я так же прошел, не увидел. Ну, все. Потом Зоя отправила

4:38
Рому. Он прошел туда-обратно, не увидел. Потом уже стал идти звонить. Идет и звонит. Слышит,

4:45
там звонит. Опа, на машине висит телефон. Он его снял, короче, сфотографировал это все. Принес,

4:53
говорит. Я такой, о, Рома, блин, ты меня спас. Спасибо. Ну, вообще. Потом, ну, такие, пошли, ладно, пошли к вам. Нафиг

5:01
уже это озеро парк. Я довольный такой. А у меня уже все мысли лезут. Я пока дома сижу, что делать? Не

5:07
позвонить, ничего. Я не знаю, где, кто, ищут, не ищут. Думаю, сейчас надо ехать, паспорт менять,

5:12
скоро уезжать, там, телефон фиг с ним, там, ладно, уже, там, контакты восстановлю. Паспорт где брать,

5:18
там, уже мысли лезут горой. Рома такой, я нашел. А, Рома, спасибо. Ничего не надо. Я счастлив, ну. Паспорт войти.

5:27
Идем уже, ну, к ним пошли, от Зои. Идем, этой машины, уже нет. То есть, она уже уехала. Буквально,

5:39
вот он, за мной сошел, я тут же собрался, мы пошли уже этой машине идти. Я вообще говорю, книжки не

5:45
напишешь. И мы пошли к ним, я там зашел. Хотя, какое-то у тебя обнуление было. Вообще. Ты знаешь,

5:51
шок-терапия. Вот, реально. Ну, это, прикинь, потерять паспорт. Ну, то есть, это, как бы,

5:55
идентификация. И вообще, мы пошли грешить, да? Ну да, конечно, со мной. Уже мы вышли, да? Ну, не, тогда,

5:56
Идентификация личности, да, как будто.

5:58
И телефон, да, так как бы тоже

5:59
мое пространство. Да. И как будто тебе в день рождения

6:02
как бы так. Да. Такая перезагрузка

6:04
произошла. Вообще. Так интересно.

6:07
Прям, да, не говори, классно.

6:08
Я бы так прям порадовалась. Ну, понятно,

6:10
что я порадовалась, что это нашлось в плане,

6:12
а если бы не нашлось, наверное, это не сильно радостно, но тем не менее.

6:14
Да, это реально шок терапии

6:16
вообще. Вообще.

6:19
Вот мы к ним пришли,

6:21
значит,

6:22
там я, ну, в магазин по пути зашел,

6:24
набрал мороженого, сок там у нас, мы по чаю подпили.

6:27
Потом я говорю, Рома, никуда

6:28
не идем, короче. Не, и так

6:30
хорошо. Давай, включай кино.

6:32
Поставили мы игру в кальмара,

6:34
второй сезон. Я сидел

6:36
до четырех часов, мы с ним смотрели

6:38
игру в кальмара, смотрели четыре серии.

6:41
Потом Зоя там приехала,

6:43
говорит, собирайтесь,

6:45
поедем во Вражку.

6:46
Ну, мы заехали там. А перед этим

6:48
еще Зоя, когда мы с Ромой

6:50
пришли, она перед тем, как уехать, это суши

6:52
заказала на вечер. Мы поехали,

6:55
забрали суши, поехали

6:56
во Вражку, потом там папа приехал,

6:58
короче, суши это посидели. И вечером

7:00
они меня в город уже

7:02
увезли.

7:06
Вообще капец.

7:09
А сегодня я с утра

7:10
пораньше пошел

7:11
на эту тренировку.

7:16
И главное, как раз тренировка закончилась,

7:18
я домой сбегал, переоделся и как раз

7:20
успел в ДК вообще. Прям вообще

7:22
все. Успел везде.

7:26
А я приходил к Татьяне,

7:31
отдала книгу,

7:31
в понедельник. Она вышла с отпуска

7:34
в понедельник. Татьяна Николаевна.

7:36
Я думаю, отдам ей книгу,

7:38
потому что я сейчас, во-первых, не буду

7:39
появляться до отъезда, потом еще

7:41
в ответе, пока буду она мне потерять.

7:43
Вот ходил каждый день.

7:45
Я теперь пропал, книгу взял и пропал.

7:49
И потому что...

7:50
Ну, а я как раз ее уже прочитал.

7:52
Ну, а в конце уже маленько торопился, чтобы

7:53
как раз ей отдать. Ну, вот она

7:55
первый день за отпуск вышла, я к ней в понедельник

7:58
девятого пришел.

7:59
Книжку эту отдал.

8:01
Сказал, что я в восход

8:03
теперь хожу. И

8:05
уже еще до отпуска я не решил,

8:08
как. Но потом

8:09
в понедельник буду определиться, куда

8:11
ходить. Определюсь.

8:14
Ну и, короче говоря, мы с ней как-то

8:16
поговорили нормально.

8:17
Так уже, знаешь, когда я на занятия хожу,

8:20
это одно. А тут я в ней кабинете

8:22
все. Мы там все, там

8:23
тости-боси. Уже

8:26
такое, другое маленькое формат.

8:28
Она там порассказывала с вами.

8:30
Ну, интересно.

8:32
Получше узнал.

8:34
Вот. Ну, вообще, вот у меня

8:41
как бы с Мариной Львовной

8:43
еще нормальный вот этот контакт.

8:46
Как бы, что мы съездили,

8:48
да, в одну школу.

8:49
И вот она мне тоже там кое-какие

8:51
штучки рассказывала.

8:53
Примерно.

8:53
Так.

8:55
У меня мама, например, тоже иногда что-то вышла.

8:57
Два, может, за шестьдесят. Я даже не знаю.

8:59
Потому что у меня, в качестве представления,

9:01
было, когда мама мне рассказывала,

9:03
какое-то время там не было, что она

9:05
вообще там глубокая чуть ли не бабушка.

9:07
Ну, в плане, ну, бодрая. Ну, как боже.

9:09
Лет восемьдесят. Ну, это как бы.

9:11
Потом это как бы она,

9:13
я прислушиваюсь, что там у нее

9:14
дочка, внучка. Я понимаю,

9:18
что внучка это явно, что, ну, не правнучка

9:19
же там, если восемьдесят лет, да.

9:22
Наверное, это где-то все-таки помоложе.

9:24
И вот мне всегда интересно было.

9:26
Я ее знать не знаю,

9:27
но по ней я постоянно слышу. То от мамы,

9:29
то от тебя.

9:31
Ну, что за человек такой интересный. Ну, да, наверное, мы шестьдесят.

9:33
Там где-то есть.

9:34
Ну, мы с ней уже сколько раз.

9:38
Собирались.

9:38
Вот два раза там,

9:41
на Павлодарах, потом

9:42
здесь, вот,

9:45
с ихними, с ихней бандой.

9:48
А вдвоем мы с ней только два раза

9:49
тет на тет собирались.

9:52
Ну, так чисто вот, что в Новосибирске,

9:54
в которой я ездил, по этой программе

9:55
позанимались. Я там ей

9:57
еще скинул

9:59
это то, что... А, она мне

10:01
видео скинула, в которое она ездила.

10:03
Там вот мастер, он

10:05
диск. То есть, то, что мы делали,

10:08
этот же самый мужик,

10:09
он записал видео, а я даже не знал.

10:11
А она перед этим ездила, а он

10:13
предлагал купить. Она купила.

10:15
А я там пытался в блокнот

10:17
записывать это все дело.

10:19
А она такая говорит, а у меня...

10:21
Ну, я потом в итоге все эти записи

10:23
потерял. Короче, я

10:25
не сильно, в принципе, расстроился.

10:27
Думаю, ну, как бы так.

10:29
Потому что этих...

10:31
Там постоянно везде в интернете.

10:33
Думаю, ладно, переживу.

10:35
А потом она

10:36
мне, значит,

10:39
говорит, я вот купила

10:40
у него эти... Вот, видео.

10:43
...твое время, сколько ты посвящаешь

10:44
вообще твоему, как бы, этому...

10:47
Ну, твоим способностям.

10:49
Ну, да.

10:49
...потенциал, да, по сути.

10:51
То есть, я так думаю, здесь как бы

10:53
никто не сможет остановить, кроме тебя самого.

10:55
Насколько ты там углубляешься и тебе это...

10:58
А мне, по большому счету,

10:59
я такой подход

11:01
на...

11:04
на твою жизнь.

11:05
То есть, я расслабленный, не для кого.

11:08
Так вот так и нужно. Как сделал?

11:09
Да, Игорь. Ну, конечно.

11:11
Сто процентов. Потому что, когда напрягаешься и думаешь,

11:13
у тебя там правильно, неправильно ли сделать.

11:15
Ну, то есть, ничего хорошего там нет.

11:17
У меня как на саксофоне тоже постоянно.

11:18
Я же там напрягаюсь, там всё...

11:21
Для меня и так

11:23
дышать, как бы, это не особо...

11:25
Ну, как бы, всегда с дыхалкой

11:27
эти проблемы, что всё детство там, да.

11:29
А тут, как бы, нужно специально дышать,

11:31
там, вдыхивать. Я напрягаюсь.

11:33
И я такой...

11:34
Расслабьтесь.

11:37
Не могу.

11:40
Не могу.

11:42
Я прям напряжение играю постоянно.

11:45
А сегодня, вот, мы делали,

11:46
когда эти занятия были,

11:49
я себя вспоминаю,

11:50
какое я первое занятие пришёл.

11:51
Вообще, дубовые, вообще,

11:54
элементарных вещей не мог сделать.

11:57
И сейчас мне кажется, что

11:58
или это я такой тормозной был,

12:00
или это просто что новый

12:01
предмет для мозга. А для меня это кажется...

12:04
Как ты раньше этого не мог сразу понять?

12:06
Это же так всё просто.

12:08
Нет, это сейчас... Нет, естественно, что мозг...

12:10
Ему нужно маленько перевернуться

12:13
на новые вот эти шаги.

12:14
Для него это всё же новое.

12:16
Ну да. Во-первых, ещё, смотри,

12:18
у нас же психика очень статичная, она не любит изменений.

12:21
И вот ты пришёл на йогу,

12:23
да, получается.

12:24
А для мозга это опасность, для психики.

12:26
То есть что-то новое, это всегда плохо.

12:29
Вот для психики и мозга новое плохо.

12:30
Почему человек в новое очень сложно иногда идёт?

12:33
Потому что это для мозга небезопасно,

12:34
для психики. И когда ты начинаешь делать,

12:37
начинается сопротивление вначале.

12:39
Даже, ну, получается,

12:40
на уровне головы сопротивление.

12:42
И, естественно, тело тоже не даёт

12:43
это до конца сделать.

12:46
А когда ты после даже первого занятия

12:49
выжил, ну, условно говоря,

12:50
что с тобой ничего не произошло плохого,

12:53
и тогда пошло уже всё,

12:54
это можно, расслабление,

12:56
и как бы дальше уже и принятие.

12:58
Вначале у нас всему новому мозг

13:00
сопротивляется всегда.

13:02
Почему люди бросают, не начинают?

13:05
Потому что психика защищает человека.

13:07
Почему там некоторые живут

13:08
в семье, где им плохо,

13:10
ну, реально плохо, там, в паре, да, что-то,

13:13
но они живут дальше,

13:15
потому что менять страшно.

13:17
Я здесь живу, мне вроде как я живой,

13:19
потому что для психики всегда, ну,

13:20
умереть там дальше идёт за всем этим.

13:23
Я живой, живу, да, мне плохо,

13:25
но я живой. А если я уйду,

13:27
то я могу умереть.

13:28
И всё. И психика не даёт, пока

13:30
вообще там жопа не станет, ну, я так уже условно говорю.

13:33
Тогда уже человек

13:34
какие-то решаются шаги. А может, они решаются,

13:37
может, так всю жизнь живут.

13:39
И для вот этого так работает.

13:40
Потому что если бы этого не было, сопротивления,

13:42
человек бы там всё бы там новое, фу, фу.

13:45
Ну, мало что так идёт.

13:48
Работают на работах,

13:49
домой приходят,

13:50
ничего там, ну, не меняют.

13:51
Ну, всё вот так.

13:55
Безопасно,

13:55
ну, вроде и плохо, ну, вроде и нормально.

13:58
Ну, то есть, вот эти вещи.

14:00
И поэтому тогда вот реально тело,

14:02
оно на уровне, ну, как бы

14:03
мозг не давал телу до конца

14:06
всё это

14:06
сделать, вот эти асаны, упражнения.

14:10
И даже вот сегодня

14:14
на тренировке там некоторые,

14:16
ну, новички совсем, видимо, были.

14:18
Они что-то поначалу

14:20
попытались, а потом просто сели

14:22
на лавочке тупо и сидели, смотрели.

14:24
Они вообще не могут понять, что это такое.

14:28
Ну, потому что, говорю, что

14:30
ну, это, это либо нужно сразу

14:32
с таким идти, ну, как бы доверие,

14:34
но тем не менее, что там остаётся сопротивление, да,

14:36
как у этого человека всегда.

14:38
Ну, как бы больше.

14:39
А когда этого нет доверия,

14:41
и человеку это навсегда не понадобится.

14:44
И для меня на 50%

14:46
именно личность преподавателя сыграла.

14:48
Ну, конечно.

14:49
Она, ну, притягивает к себе.

14:51
А на 50% просто, что вот именно

14:54
куда-то пойти.

14:55
Что ты вечером приходишь домой, и тебе охота куда-то пойти.

14:58
А сама йога это как бы

15:00
некий повод просто.

15:01
Ну, личность это очень много значит.

15:03
Потому что вот даже я сейчас слушаю

15:06
по обучению, там, бизнес,

15:08
вот же все, психология, да.

15:09
То есть 80% успеха,

15:12
успеха бизнеса составляет

15:14
сам

15:15
владелец бизнеса.

15:17
Его рост, его развитие,

15:19
его вообще восприятие этого мира.

15:22
А вот 20% там маркетинги,

15:24
всякие вот эти рекламы,

15:25
потому что, ну, как бы,

15:27
это не как бы насколько,

15:29
здесь тоже, насколько энергетика у преподавателя,

15:32
что на него идут.

15:35
Именно на него идут.

15:38
Время найдут всегда.

15:40
Найдут там

15:40
финансовые моменты, да.

15:43
Как бы, если чувствует человек,

15:44
что ему нужно, он будет ходить.

15:48
Кажется, здесь тоже, наверное,

15:49
процентов не 50, даже, наверное, больше.

15:52
Возможно.

15:53
Что именно идут на учителя

15:55
люди.

15:57
Ну, да, и вот себя, если

15:59
ставишь в шкуру ученика,

16:02
на самом деле,

16:04
это очень тяжело.

16:06
Я этот путь прошел, я считаю, год, да,

16:08
годил, и только вот пришло понимание.

16:11
А когда, допустим, я учитель,

16:13
да, и я вижу,

16:14
думаю, думаю, говорю ученику делать,

16:17
а в его шкуру залезть очень тяжело.

16:19
Это надо полностью

16:20
трансформировать сознание, чтобы

16:22
в его восприятие войти.

16:26
Но

16:26
преподаватель всех не может как бы так

16:28
и... Объять. Да, поэтому

16:30
просто говорит, делайте вот это.

16:32
У него же стандартизованный методик.

16:34
Делайте так. А что за этим

16:37
стоит, он не объясняет. И это правильно.

16:39
Потому что каждый ученик свое может

16:41
видеть. А если пытаться разжевать,

16:42
оно вообще загрузится и все.

16:44
Тут главное, что мозг отключить

16:47
и просто делать. Тебе говорят, и все.

16:49
Постепенно придет все.

16:55
Я, в принципе,

16:56
так и делал методику.

16:59
А у нас же, ну,

17:00
много людей сразу требуют

17:02
фу, да что это такое, им надо,

17:04
чтобы это все объяснили им,

17:06
что это для чего. Они думают, да это

17:08
странно, да это вообще что попало.

17:11
То есть, понятно,

17:12
что за этим все это стоит. Но если это

17:14
начать развертывать, это

17:16
можно уйти там вообще в дебри и

17:18
мета оторваться и вообще.

17:20
То есть, если это все разжевать.

17:22
Ну да. Это знаешь, это же опять же

17:23
человеческий контроль. Я делаю,

17:26
для меня это странно, непонятно.

17:28
И я тогда не смогу это проконтролировать.

17:30
То есть, я теряю контроль вообще на ситуации.

17:32
Чем я делаю, почему я

17:34
делаю, для чего я делаю. И поэтому

17:36
человеку всегда важно объяснение всего.

17:38
То есть, якобы, что он это проконтролирует

17:40
тогда, скорее всего.

17:42
А отпустить сложно контроль.

17:44
Да. А еще, знаешь, бывает,

17:46
что-нибудь делаешь, вот, например, цигунту,

17:48
тот же. Кто-нибудь подойдет, так стоит,

17:50
смотрит, типа, с таким этим.

17:52
Как, типа, ну и

17:54
что, помогает, типа?

17:58
А чего, интересно?

18:00
Ну, типа. Что за эти вопросы, да, стоят?

18:02
Вообще, да. То есть, или человек,

18:04
вообще непонятно, что он, что он.

18:06
То есть, или какой-то

18:08
ослиная, или вообще даже

18:10
непонятно, что. Вот такого человека

18:12
в чем-то убедить вообще невозможно.

18:14
Да.

18:18
То есть, у него изначально установка

18:20
все. Отрицание.

18:22
Ну, это, видимо, явно

18:28
такой самый

18:30
проявленный тип

18:32
непринятия чего-то нового.

18:34
Да. О споре, что человек

18:36
это делает неправильно. Ну, то есть, и что-то

18:38
это за этим стоит. А спорить

18:40
тут реально. Вот, почему люди, да, не

18:42
могут принять в других людях

18:44
что-то, что они делают для них, как бы,

18:46
необычное. И кажется, и хочется

18:48
сразу, что этот человек дурак. Ну, который

18:50
делает, да, это. Да он что,

18:52
дебил какой-то делает там это.

18:54
А я не делаю,

18:56
я, как бы, все нормально со мной.

19:00
То есть, себя, как бы, так

19:02
окрасить и выглядеть. Ну, да, если, например,

19:04
в метро, там, ты начнешь, там, улыбаться

19:06
во весь рот, там, или, там, песни петь,

19:08
тебя сразу попадешь

19:10
в категорию ненормального. Да, да, да.

19:12
А если газеткой закроешься,

19:14
будешь ехать, тогда будет нормально.

19:16
Да, вот эта

19:18
корма у нас получается очень сильно,

19:20
где-то, как бы, это.

19:22
Ну, это взрослые же, начиная от детей,

19:24
нормировать. И у детей это при,

19:26
у некоторых

19:29
выживается у них. То есть,

19:31
вот не надо, там, на улице, там,

19:33
ребенок лежит, там, кричит, песни

19:35
поет. А родители, там, что ты орешь,

19:37
там, еще. Ну, то есть, им самим не комфортно.

19:39
Даже улица, да, по сути, как бы, какая разница.

19:41
То есть, начинают

19:43
это пресекать. Дома тоже, там, ну,

19:45
еще что-то. Конечно, бывают моменты, когда

19:47
где-то неуместно так, да, как бы.

19:49
То есть, мне кажется, родители очень сильно

19:51
детей, иногда, зажимают

19:53
в эти моменты. Где-то зажимают,

19:55
а где-то в других ситуациях

19:57
не дожимают, знаешь, как-то

19:59
вот этот момент. И потом человек такой

20:01
становится нормированный.

20:03
Сильно. Где надо, где не надо.

20:05
Ну, то есть, да, это как

20:07
некое серое масло получается.

20:09
И вот как раз вырабатывается

20:11
вот этот вот

20:13
тип людей, которые, вот, на работу, с работы

20:15
и боятся что-то менять, там.

20:17
Такое какое-то, пассивность такая.

20:19
Вот.

20:23
Ну, да, я вот уже сейчас

20:26
стал по этим меркам

20:28
рассуждать, что действительно,

20:31
как если взять меня,

20:33
например, даже.

20:36
Ну, и вообще даже

20:38
не только меня.

20:40
Допустим, если так рассуждать

20:42
трезво, да.

20:44
Человек находится как пленник.

20:46
Вот у него вот такие настолько

20:48
узкие рамки, то есть вот работа, дом,

20:50
это вообще жесть, вообще

20:52
невообразимо даже.

20:54
И тем не менее, некоторые люди так живут.

20:56
Пытаются

20:58
что-то изменить.

21:00
Вот это, со стороны смотришь,

21:02
блин, это так дико вообще.

21:04
Одно и то же, одно и то же,

21:06
одно и то же, вообще.

21:08
А знаешь, почему так жить нельзя?

21:11
Ну, в плане того, что до поры,

21:13
до времени все это.

21:15
Вообще, как вот я слышала, что человек пришел

21:17
на Землю для того, чтобы развиваться.

21:19
Всегда. Ну, вот прям каждый

21:21
это не касается, там, школа и все,

21:23
я закончила свое развитие, да.

21:25
То есть всегда развиваться.

21:27
Когда человек живет в таком, ну, как бы болоте,

21:29
по сути, да, когда все там у него все стабилизировалось,

21:31
только, конечно, работа, дом, все.

21:33
И ничего нового не появляется.

21:35
И тогда мир, там, Бог, Вселенная

21:37
того, что доверит, да, обязательно

21:39
какую-то ситуацию провернет,

21:41
там, какой-то кризис произойдет в чем-то,

21:43
в семейных отношениях, на работе, там,

21:45
вот хоть где, там, с друзьями, там, здоровье.

21:47
Чтобы человек маленько

21:49
встряхнулся и начал

21:51
что-то как будто бы, ну, менять, да, в жизни

21:53
в своей, там, ну, в своей жизни,

21:55
чтобы он как-то это маленько

21:57
из болота выбрался.

21:59
И вот когда человек меняется,

22:01
сам, то есть, делает шаги, там, учится,

22:03
там, на тренировки ходит, там, какие-то

22:05
вот эти, тогда жизнь не будет ему таких

22:07
стрессовых ситуаций, ну, часто давать, по крайней

22:09
мере. Понимаешь, как это работает? Если я

22:11
делаю шаги, чтобы меняться постоянно,

22:13
меня не будет Вселенная, там, Бог, учить

22:15
меняться, ну, как бы, будет. Я сам

22:17
выбирать буду свои стрессовые ситуации,

22:19
по сути, которые я, ну, пройду.

22:21
Вот ты едешь из путешествия, тоже, по сути, стрессовая

22:23
ситуация. Там поезд.

22:25
Нужно успеть, там, до этого автобуса.

22:27
Неизвестные люди. Неизвестные люди, да.

22:29
Ну, как бы, придет на такой стресс, ну, ты на это идешь,

22:31
и ты себе создаешь ситуацию, что будет,

22:33
ну, развиваться в этом плане.

22:35
И это так надо жить, в плане, как бы,

22:37
иначе тебя, там, будут

22:39
проучивать, как говорится.

22:41
Да.

22:43
Это прямо, вот, я так хорошо запомнила, думаю, постоянно

22:45
нужно что-то с собой, ну, как бы, где-то

22:47
пройтись и менять.

22:51
Мне кажется, еще знаешь, что тренировки,

22:53
которые, там, для тела, такие,

22:55
ну, полезные, там, нужные,

22:57
там, йога, цивун,

22:59
они, мне кажется, развивают еще

23:01
гибкость, не только, там, тела,

23:03
а такую пластичность,

23:05
гибкость нашего ума,

23:07
наше, вообще, восприятие.

23:09
Насколько ты, вот,

23:12
как начал заниматься, стал относиться

23:14
к ситуациям более так,

23:16
не категорично, а более гибко,

23:18
ну, вот, к любому, то есть, я не знаю,

23:20
или ничего не поменялось именно

23:22
в восприятии, там, как люди себя ведут?

23:24
Как это муха укусила?

23:26
Ну,

23:28
очень сильно

23:30
поменялось. И даже я

23:32
часто анализирую,

23:34
даже, вот, то, что

23:36
я говорю, вот, например,

23:38
свои действия

23:40
стал анализировать.

23:42
Может, это и не связано с йогой, но,

23:44
по крайней мере, я постоянно пытаюсь

23:46
себя исправить. Вчера,

23:48
например, вот, Рома не захотел

23:50
мне помочь, там, скачать,

23:52
этот, музыку я нашел.

23:54
Ну, мы сидели, говорю, Рома, давай

23:56
поставим эту музыку на закачку.

23:58
Я, там, торрент хочу скачать. Он говорит,

24:00
я не хочу это устанавливать, программу,

24:02
там будет вирус у меня в компьютере.

24:04
Я говорю, ну, ладно,

24:06
скинь мне торрент на почту, я

24:08
во вражку поеду, там,

24:10
ноутбуки скачаю. Не стал

24:12
давить на него. Хотя мог.

24:14
Не, просто, что, ну, думаю,

24:16
ладно, не надо, зачем,

24:18
не хочет, не хочет. Ну, да.

24:20
Вот это тоже же про принятие позиции

24:22
другого. Да, а потом, значит, во вражке, когда мы

24:24
были, у меня это, я поставил на закачку,

24:26
ну, мы, там, пока сидели, я думал,

24:28
скачается, оно не успело, там, много.

24:30
И скорость маленькая. Я что-то так

24:32
маленько, ну, раздраженный

24:34
был. Говорю, Рома,

24:36
почему ты не поставил мне? Мы бы

24:38
скачали, там, у кого у тебя были,

24:40
фильм смотрели. Он говорит,

24:42
да, я, говорит, это, качал.

24:44
И я ему начал,

24:46
да ты, блин, не понимаешь, что ли,

24:48
это, ты качаешь, там, что попало,

24:50
свои торренты с вирусами, а тут

24:52
это все, ну,

24:54
нормальный торрент сайт, я, так,

24:56
все это официально. Это,

24:58
ну, это, там, музыка, что

25:00
ну, что,

25:02
ну, а он мне говорит, я же не знаю,

25:04
как это все работает. Я ему говорю, да я знаю,

25:06
как это работает, слушай меня.

25:08
Потом, ну, ладно, а потом

25:10
думаю про себя. Надо было ему

25:12
объяснить, Рома, торрент, оно работает

25:14
так. Там, вот, есть такая система,

25:16
что ты, вот, это, это идет туда,

25:18
это так, это так. Ну, объяснить ему

25:20
подоплеку, ну, как

25:22
это все организовано. А я ему сказал,

25:24
да ты делай так, потому что я знаю, что это

25:26
так. Так импульсивно

25:28
я ему сказал.

25:30
Потом думаю, ну, как бы это ни к чему

25:32
же не приведет все равно. А так я ему

25:34
расскажу, как этот торрент и все это

25:36
работает. Он, наоборот, мне скажет

25:38
спасибо, что ты мне все объяснил,

25:40
теперь мне все понятно, я тебе это могу,

25:42
скачаю, все.

25:44
Ну, видишь, может, когда вот такая эмоция

25:46
поднимается, вот это раздражение, да,

25:48
мы себя уже не можем отконтролировать

25:50
и правильно и начать эти, делать,

25:52
ну, шаги, как бы

25:54
говорят, да, действия.

25:56
Начинаем срываться, других виноватыми там искать.

25:58
Ну, он же, получается, у тебя виноватый,

26:00
казалось. Не ты там, да?

26:02
А он виноват.

26:04
Вот, это вот самое простое, да, тебе виноватого,

26:06
и психики такие, фух, не я.

26:08
Все со мной нормально.

26:10
Это другие.

26:12
Вот, а когда ты это анализируешь,

26:14
то уже на будущее, возможно, уже по-другому

26:16
будешь, как бы, и все.

26:18
Хорошо, когда анализ идет за

26:20
ситуацией. Да, потому что это тренировка.

26:22
Конечно. Ты не можешь сразу

26:24
быть таким это уравновешенным.

26:26
Это постепенно, да, ты себя

26:28
как бы получаешь, получаешь. Мозг

26:30
уже, как бы, переводишь

26:32
в режим такой более

26:36
осторожный, да,

26:38
и в следующий раз, когда такая ситуация

26:40
возникнет, ты уже, а, я уже про это

26:42
думал, так, тихо.

26:44
Сейчас будем делать вот так. И постепенно,

26:46
постепенно, постепенно это уже

26:48
перестраивается именно рефлексорно.

26:50
Угу, да. Идет уже

26:52
эта наработка, как бы, да.

26:54
Потому что сразу нельзя стать таким

26:56
спокойным. Просто самое, знаешь, самое

26:58
хорошо, очень, мне кажется, редко.

27:00
Прям редко, по моим наблюдениям,

27:02
люди анализируют

27:04
ситуацию. Вот прям

27:06
я, почему

27:08
в таком, как бы, они не могут

27:10
выбраться. Они вот, типа, по этому кругу

27:12
кружатся, по своим

27:14
стратегиям, да, действиям. Вот они

27:16
раздраженные, вот они там сказали,

27:18
нашли виноватого, и все. И они потом

27:20
просто ситуацию не анализируют. Они оставляют

27:22
виноватого этого человека,

27:24
и следующая ситуация будет такой же.

27:26
Будет такой же, будет такой же.

27:28
То есть они себя там не увидят,

27:30
ну, как бы, условно, плохим.

27:32
Ну, условно, да, говоришь. Ну, каким-то таким, что

27:34
я это неправильно там поступила.

27:36
И это

27:38
очень такой классный навык, когда

27:40
свои поступки потом анализируют.

27:42
Ну, свое, такие, проживание,

27:44
события. Вот. Это,

27:46
ну, очень редкие люди. Очень.

27:48
Мне кажется, это

27:50
процентов 15. Всех людей,

27:52
которые живут. Ну, потому что это работа.

27:54
Это так же, как растяжка. Ты не можешь

27:56
сразу стать гибким там.

27:58
Тебе надо постепенно тянуться, тянуться,

28:00
тянуться, тянуться. Так же, как

28:02
и мускулы, за один день ты не можешь выросить.

28:04
Да, да, да. Ты должен их накачивать, накачивать.

28:06
Ну, по самом деле, вот эти практики, которые там

28:08
физика, да, какие-то

28:10
моменты понимаешь, что это постепенно,

28:12
это потихоньку. И так же, вот,

28:14
своими

28:16
стратегиями, с каким-то поведением

28:18
начинаешь тоже анализировать.

28:20
То есть, это навык какой-то. Тренировка, тренировка,

28:22
да. Даже псих. Тренируешь

28:24
то же самое, как мышцы. Все одинаково.

28:26
как так же как ты навык любой тренируешь тот же третий сайт или там когда там на

28:31
велосипеде ездить что угодно на машине сначала там это чуть-чуть потом больше

28:36
больше потом уже все это никогда на тренируется но уже сам муфтик

28:42
люди живут стратегиях я от свидетель насколько люди не желают вообще свое

28:56
мнение категоричность а мир понимаешь он не терпит они же считают что они правы

29:05
там нет они что-то несправедливо да а вообще есть такое понятие что нет

29:10
справедливости праведливости не существует у каждого на будет от правда

29:14
у каждого своя правда у каждого своя а истина деда там далеко от которой мы даже

29:18
не знаю а мы-то хотим свою правду навязать я когда вот стал вот сейчас как

29:25
раз ты спросила когда

29:26
стал заниматься угоном что поменялось поменялось вот что я именно раньше

29:31
воспринимал что вот я считаю что вот должно быть вот как каким-то образом вот

29:37
у меня какое-то мнение и если это под моим не не укладывается значит я считаю

29:43
что со мной поступили несправедливо и и все с кем не игра а сейчас уже когда я

29:54
стал заниматься я

29:56
пришел к такому видению что все не надо там кому-то завидовать или как это сказать

30:06
что наказывать то есть например если кому-то какие-то блага до делают а тебе

30:19
не а тебе не делать то ты уже не относишься к этому человеку которым

30:23
делать блага негативно вот с негативного отношения у меня уже не

30:27
то есть хорошо ему и хорошо а то что у меня не сказаться значит я не смог найти

30:33
тот вариант который устроил да и вот это мне понимание пришло то есть есть некая

30:42
как бы море вероятности возможности и ты должен исходя из того что есть вот у

30:51
тебя выстроить все так оптимально чтобы сложить карту свою пользу вот то есть и

30:58
ты у тебя видишь как что он снова Breakups куратор Landes der Grund Кавказской школе

31:06
если у тебя не складся значит ты что-то не дадибл là думаю как как сделаем вот как

31:15
таком плане перестроился тоже неправильно� vibrio нати виноватого то том что он виноват

31:24
того что у него получилось ну как бы так и все и тебе ничего делать мира human league

31:24
dahli Muse отлич thousands stand to me прям я потому уверена что когда человек начинает чем-то заниматься эта физика

31:27
физика, да, как-то свою физику менять.

31:29
Ну, не то, что там...

31:30
Может, и любое занятие физикой.

31:34
То есть, оно как-то меняется

31:35
внутри человека чуть-чуть.

31:37
Не чуть-чуть много, ну, не важно,

31:39
как оно меняется, но как-то меняется.

31:42
Физику всегда когда-то...

31:43
То есть, знаешь, есть же такое, что

31:45
изменения идут изнутри.

31:47
То есть, человек меняется, и внешнее

31:50
меняется, ну, мир, условно говоря.

31:52
Но у меня я прям убеждена, что

31:53
если мы с физикой начинаем работать,

31:55
то внутри тоже что-то поменяется у нас.

31:58
Сто процентов. Ну, какая-то наша

31:59
составляющая, не только тело.

32:02
И вообще, даже я тебе больше скажу,

32:04
физика — это

32:05
основа. Ум, но он, наоборот,

32:08
паразит. Он не нужен

32:10
вообще, по идее.

32:12
Ну, я даже не про ум, я про вот то, что

32:14
наше внутреннее состояние, вот как ты сейчас

32:16
говоришь, да, что перестала там завидовать,

32:18
да. Вот это меняется.

32:20
Какое-то восприятие вообще мира

32:21
тоже от физики может поменяться.

32:24
Вот.

32:26
То есть, как ты его принимаешь, более гибким становишься.

32:28
Ну, у меня прям, мне прям гибкость

32:30
идёт про то, что я принимаю

32:32
то, что в мире происходит.

32:34
Без каких-то там своих

32:36
навязываний. То есть,

32:38
вот ты тоже, да, человека получилась.

32:40
То есть, не он в этом там виноват, да, ну, условно.

32:43
А я ищу

32:44
решение, чтобы мне тоже получилось.

32:46
То есть, в себе разворачиваю.

32:49
В себе всегда. Что?

32:50
Почему? Да, и ты

32:52
оставляешь за собой

32:53
возможность

32:57
улучшения и изменения. А когда

32:59
ты, как бы, всех обвиняешь,

33:01
всё, ты сел, всё, ты

33:03
с собой не справедливый,

33:06
тебя обидели, тебе ничего делать не надо.

33:08
Всё. Ну, это же тоже, блин, мне кажется,

33:09
большая часть людей так делает.

33:12
Я, типа, всё, типа,

33:13
свои дела сделал, типа,

33:16
я ни в чём не виноват. Мне ничего не надо делать.

33:19
Да.

33:20
И ответственность

33:22
снимается, но...

33:23
И, как бы, получается, коридор вот этих вот

33:26
действий, возможностей, он

33:30
закрывается. А тут он наоборот.

33:32
Ты должен, ты понимаешь что-то,

33:34
и ты что-то делаешь.

33:36
Интересно. Делаешь, делаешь, делаешь.

33:38
Когда ты понимаешь, когда я понимаю,

33:40
что от меня зависит моя жизнь,

33:42
как складывается мир, да,

33:44
то я же понимаю, что

33:46
если я сижу

33:47
просто сидя, ничего не произойдёт.

33:50
Я должна какие-то делать шаги

33:52
постоянно. Шаги, шаги, шаги.

33:54
То, что я хочу. Да.

33:56
И так же вот, как замкнутый круг,

33:58
вот, когда человек, ну вот,

34:00
яркий пример, далеко ходить надо

34:02
мои родители. Они

34:04
постоянно ищут друг друга виноватыми,

34:06
и оно идёт бесконечный круг.

34:08
Да, конечно. Бесконечный.

34:10
И я поражаюсь, ну, до чего однотипное

34:12
у них мышление,

34:14
это просто. Ну, хотя бы один кто-то

34:16
проявил какое-то, не знаю,

34:18
творчество, нетворчество, просто

34:20
некое, да даже, какой-то

34:22
разум или там, не знаю.

34:24
Ну, знаешь, потому что... Одно и то же,

34:26
одно и то же. Им нужно

34:28
этот путь прожить. Им информация новая

34:30
не идёт, потому что им пока

34:32
это не надо, и они к этому не готовы.

34:34
И, то есть, им надо вот почему-то

34:36
такую судьбу прожить, без вот каких-то вот этих

34:38
новшеств, ну, как бы, рябких изменений.

34:41
Ну, почему-то вот так.

34:42
Ну, да, вот как робот, реально.

34:44
Ты маленько хоть сверни, что ты

34:46
едешь, вот как упёртый, вот по этому.

34:48
Ну, потому что так уже привык.

34:50
Ну, как бы уже, понимаешь, насколько там,

34:53
знаешь, для примера,

34:54
нам приводили. Представьте

34:56
вот поле, да, поле,

34:58
и вот эта вот дорога идёт,

35:00
асфальтированная, да, это вот ваши навыки.

35:03
А там просто трава вот такая вот, да,

35:06
там какие-то ещё. И чтобы

35:08
новый навык, надо просто там какой-то, ну,

35:10
даже, может, какой-то бурелом,

35:12
там ещё что-то, там нужно приложить усилия.

35:14
То понятно, мне здесь комфортно идти.

35:16
Ну, да. То есть, по этой дороге

35:18
зачем я туда полезу.

35:20
И нет веры, что что-то поменяется.

35:21
Да я смотрю на них, я это понимаю.

35:24
Обвильнуть один сантиметр чуть-чуть

35:26
не могут. Опасно.

35:28
Не то, что в бурелом, просто на самой

35:30
этой дороге, но чуть-чуть в сторону.

35:33
Хотя бы, да.

35:33
Чуть-чуть обвильнуть вот так.

35:36
Даже этого не могут.

35:37
Я вот поражаюсь, вот реально.

35:40
Вот одна, стереотип

35:41
вот один и тот же. Каждый

35:43
друг друга обвиняет

35:44
и бесполезно. Всё. Это вообще

35:48
жесть. Я не знаю, как так жить можно.

35:51
Ну, всё-таки, знаешь, я,

35:53
ну, ко мне приходят, там, на консультации,

35:54
всё-таки, я уверена, что

35:56
изменения начинаются

35:58
с женщины.

36:00
Я это и себе тоже, ну, как бы, скажу.

36:03
В плане того, что

36:04
когда женщина начинает меняться,

36:06
мужчина тоже поменяется. Ну, да.

36:08
Это вот прям я уверена.

36:10
Конечно, бывает такое, что женщина, да, вроде

36:12
как бы, а он, как бы, вот, остаётся

36:14
таким, тогда, как бы, ну, нет.

36:16
Это вместе уже невозможно, да, допустим. Ну, бывает

36:18
такое. Это же, как бы, мы все

36:20
меняемся. Нет такого, что

36:22
я вышла, вышла, там, замуж, я, там,

36:24
да, да. Ну. Может человек меняться

36:26
настолько, что мне с ним уже как-то, ну, некомфортно.

36:29
И это тоже нормально, в принципе.

36:31
Просто вот говорю, что

36:32
начинать должна, ну, как должна.

36:35
В изменении, конечно, идут

36:36
женщины. Как только начнёт меняться,

36:39
то мужчина тоже может, как бы,

36:40
поменяться. Ну, какие тут эти моменты.

36:42
Ну, да. То есть, получается, когда

36:44
как бы ты толкнул,

36:46
он тебя толкнул. А когда ты чуть-чуть

36:49
его сбоку толкнул, он уже сам

36:50
волей-неволей сместится.

36:52
У него уже другая реакция пойдёт.

36:54
То есть, хотя бы один, кто-то

36:56
по-другому чуть что-то сделает,

36:58
уже картина пойдёт другая.

37:01
Может даже мужчина что-то

37:02
по-другому сделает. Конечно. Какой-то

37:04
триггер у женщины сработает. А, слушай,

37:06
он так сделал. Может мне тоже как-то

37:08
так сделать? Угу. То есть,

37:10
и уже пойдёт другая картина.

37:12
И для этого достаточно человеку,

37:15
хотя бы как

37:16
одному, просто

37:18
сделать что-то выходящее

37:20
из этого цикла бесконечного.

37:22
Да, да, да. Просто, ну, включить.

37:24
Хотя бы. Нет, посмотришь, получается, люди,

37:26
они не видят, что они в этом цикле.

37:28
Они же, ну, внутри-то не видят. По идее, да.

37:30
Хорошо. Слушай, правда.

37:32
Вот ты сейчас сказала, я вспомнил

37:34
историю. Там один

37:36
квебек, ну, француз один

37:38
с Димой работал. Он из Франции.

37:41
Мы с Димой, когда

37:42
я звонил ему, он говорит, вот, на вахте

37:44
был как раз. Вот, он говорит, подойди

37:46
с Игорем поговорить.

37:48
Ну, мы с ним чуть-чуть поговорили по-французски.

37:50
Я говорю, о, у тебя такой классный французский.

37:52
Прям вообще. Он говорит, а с

37:54
здесь квебеки, говорит,

37:56
меня считают, что у меня вообще стрёмный французский.

37:59
Я говорю, да я знаю,

38:00
как квебекцы говорят, у них такой ужасный

38:02
акцент, они вообще

38:03
слушать невозможно.

38:06
Он говорит, это ты так кажется. А для них

38:08
их язык нормальный, и они тебя считают.

38:11
Что ты француз,

38:12
у тебя кривой язык.

38:15
Я такой думаю, блин, а в натуре

38:16
они же не понимают, что

38:17
у них не нормально. Для них

38:19
их язык нормальный.

38:23
А настоящий

38:24
француз...

38:24
Французский коверкан.

38:26
Для многих, для них жизнь

38:28
нормальная. Вот для них же их жизнь

38:30
таки нормальная. Представляешь?

38:33
Да. Вообще.

38:35
Да, там, где-то им плохо,

38:37
где-то ещё что-то. Но

38:38
они считают, что это нормально так жить,

38:40
раз они ничего не меняют. Именно как бы вот так.

38:43
Но это должен какой-то внутренний

38:44
импульс идти, какая-то

38:45
природа, что, как это лягушки, да,

38:48
вот, допустим,

38:49
в пруду, да, они, какая-то лягушка,

38:52
вон там за горой ещё круче

38:54
пруд есть. А эти сидят,

38:56
да, ну, там, здесь, в болоте

38:58
в нашем.

38:58
Ну, тоже, видишь, человека выбрали, надо его принимать.

39:01
И, ну, не лезть там со своими

39:02
такими уставами, да,

39:04
как говорится. Понятно, что мы даже родители,

39:06
ну, после того, что мы родители, то не сможем

39:08
сказать, что они неправильно живут.

39:10
Да, и наши полномочия. И тут уже думаешь,

39:13
думаешь, ну, неужели

39:14
вот как человек, он же как

39:16
мыслящий, сознательный. А тут

39:18
получается уже, думаешь, ну, вообще

39:20
человек может, человек вообще мыслящий

39:22
существует или нет?

39:24
Ни концов. Или он просто

39:26
как на рефлексах. Нет, просто для них это нормальная

39:28
жизнь. Они, ну, как бы всё.

39:30
Ну, это априори. Ну, это как

39:32
с французским. Ну, значит, надо это просто

39:34
принять. То есть, да, для тебя

39:36
это ненормально, а для них это норма.

39:38
Какие-нибудь.

39:40
Это как бы, как бы это ни казалось. А ещё,

39:42
а ещё для себя я,

39:44
вот, кто мне говорят все,

39:46
вот, ну, там женщины особенно любят

39:48
козырять, вот там,

39:50
типа,

39:51
почему ты не находишь там подружку себе,

39:55
ну, для них это кажется, что вот

39:56
они, типа, такие всезнающие, вот.

39:59
Типа, мужчина должен...

40:00
Нет, нет, мужчина должен

40:02
их там найти, то есть

40:04
мужчина обязательно должен

40:06
там кого-то встретить, вот они

40:08
такие, значит, женщины,

40:10
что без них никуда. Вот.

40:12
А я для

40:14
себя

40:15
такую картину вижу, что

40:17
если человек хочет, он

40:20
его ничто не остановит. Он может

40:22
сделать всё, что он хочет.

40:25
Но тут добавляется ещё, ну, опять же,

40:27
если он не хочет, то ты ему любые

40:29
доводы приводи, он не

40:31
хочет. Вот. Он любые

40:34
там найдёт отговорки и прочее.

40:36
Даже есть поговорка, что

40:37
желание – это тысяча причин,

40:39
нежелание – это тысяча...

40:41
Нежелание – это тысяча причин,

40:44
а желание – это тысяча возможностей.

40:46
Но для меня тут идёт ещё

40:47
уровень, как бы, ниже.

40:49
Я заставляю себя,

40:53
чтобы убрать

40:54
желание,

40:55
потому что

40:56
я не вижу

41:00
на данный момент возможности

41:02
их осуществления.

41:03
Я как бы, когда убираю желание,

41:06
то есть на уровень ниже спускаюсь,

41:08
для меня обоснование простое.

41:10
Мне не надо там выстраивать какие-то

41:11
доводы. Я просто говорю «не хочу».

41:14
А потом, допустим,

41:16
если я вижу, что подоплёку

41:17
я могу выстроить, которая

41:19
всё обосновает, я тогда захочу.

41:22
Я буду говорить «я хочу».

41:24
Получается, довод очень простой.

41:25
«Я хочу». «Почему ты это делаешь?»

41:28
«Потому что я хочу». «Потому что я не хочу».

41:30
И всё просто становится.

41:33
Не надо

41:33
никому ничего доказывать.

41:35
Я даже это слышала, что, типа, когда вас спрашивают,

41:37
почему вы там... Ну, у вас, допустим, просят что-то сделать, да?

41:39
Ну, просят что-то сделать.

41:41
А вы там можете сказать «нет»,

41:43
потому что надо обосновать в поле себя.

41:46
Не надо обосновывать. Просто «не хочу».

41:48
Просто.

41:52
Я такая... Я про это запомнила,

41:53
когда ты об этом говоришь.

41:55
Как всё просто, реально.

41:57
«Не хочу». А я тебе говорю, что это будет странным,

42:00
как бы, да? В смысле,

42:01
ну, как бы, нужна такая веская

42:04
причина, что ты мне отказал.

42:06
Типа, в моей же картине

42:07
мира ты обязан хотеть?

42:09
Как? Ты что делаешь?

42:12
Это прям большой вот

42:13
повестка, когда человек скажет «чему?»

42:15
«Тебе хочется? Почему?» «Ну, не хочу».

42:18
Сразу ступор такой.

42:20
А мы же, понимаешь, почему боимся

42:21
сказать-то «нет»? Ну, многие

42:23
не хотят. Я в том числе.

42:25
Мы не можем сказать «нет».

42:28
Потому что я скажу человеку

42:30
«не хочу», да, на его просьбу,

42:32
а он от меня, как бы, отвернётся.

42:33
А для человека быть изнанным из стаи,

42:36
ну, условно, это уже смерть.

42:37
Как вот древние люди же не могли от ней выжить.

42:40
Они же жили из стаи, да?

42:41
А, ну да. И вот. И это в древнем нашем

42:43
ящерном мозге сохранилась информация,

42:45
что для меня страшно, если меня, как бы,

42:47
социум от меня отвернётся.

42:50
И прикинь, насколько это в нас прошито.

42:53
Просто.

42:54
Слушай, а может, это и есть корень,

42:55
того, что вот человек смиряется

42:57
вот с этой вот рутинной жизнью на работу,

42:59
с работой?

43:01
Может, это и держит? То есть, я работаю,

43:03
ну, так принято в социуме, да?

43:05
Вот я, ну, тут ещё добавляется того,

43:07
что страх, что я умру от голода,

43:09
что денег не будет, я не смогу заработать.

43:12
Ну, это же тоже то.

43:13
Я такой, ящерная наша прошивка

43:15
древняя. Вот. И как раз вот эта

43:18
ящерная прошивка, она

43:19
для меня сработала триггером. То есть,

43:22
чтобы этого не бояться,

43:23
нужно это допустить.

43:25
А раз ты это допускаешь, у тебя

43:28
автоматически отходят те

43:30
желания, которые могут

43:31
вернуть этот страх. Это как из психологии.

43:34
Когда у человека, допустим,

43:36
есть сильный страх там чего-то,

43:38
да? А у нас есть такая поговорка,

43:40
что все страхи у человека стоят в очередь

43:42
на воплощение. Пока я чего-то

43:44
очень сильно боюсь, мир такой,

43:46
блин, да хватит уже бояться. Вот это произойдёт,

43:48
и она больше не будет бояться.

43:50
И тогда эти страхи нужно реально прожить.

43:52
Даже себе в голове, прям до конца

43:54
пойти. Вот как бы такая,

43:55
тоже такой инструмент есть, чтобы этот страх

43:57
воплотить даже себе внутри. Что

43:59
вот я уволилась с работы, вот я

44:01
не могу зарабатывать,

44:03
и вот я умерла. Ну и что дальше?

44:05
Ну всё, умерла. Ну а что страшного-то, по сути?

44:08
Ну как бы,

44:10
ну, иногда

44:11
может облегчение, но это у кого-то

44:14
будет. Ну да, что

44:15
страшного, да, умру, а как мои дети?

44:18
Ну как дети, там бабушки,

44:19
отец есть, там детские дома, да? Ну вроде как

44:21
с тобой же они будут жить. Как-то, но

44:23
будут. Ну в принципе, да. И вот как раз,

44:25
вот я вот к этому сейчас, как раз

44:27
восприятию и подхожу, вот именно то, что

44:29
ты говоришь. И я вот начал к этому

44:31
восприятию подходить, как раз вот, когда я стал

44:33
заниматься йогой, потому что

44:35
это расширяет вот это восприятие.

44:38
Раньше ты видел только это болото, и всё,

44:39
для тебя это было болото. А сейчас ты увидел

44:41
его с одного ракурса, с другого,

44:44
ты уже понимаешь, что есть другой язык,

44:46
есть испанский, есть там

44:47
португальский, и ты уже, о,

44:49
оказывается, вот. Это значит, это вот как раз,

44:51
как раньше говорили, и сейчас говорят,

44:53
то есть, меняйте своё мышление,

44:55
меняется жизнь, да. А мне всегда было непонятно,

44:57
в смысле, ну а как менять мышление, ну как, ну как его менять?

45:00
А менять мышление, это допускать

45:01
варианты другого,

45:03
других знаний, других возможностей.

45:06
Вот есть такой ракурс сейчас про французский,

45:07
да, есть любой французский.

45:09
И у тебя хоп такой, мы же все как бы живём

45:11
в туннельном своём мышлении, да,

45:13
наши знания, вот они здесь выстраиваются, мы других

45:15
не допускаем. А когда мы начинаем допускать

45:17
другие знания, другие жизни, другие

45:19
вообще восприятия жизни, у нас

45:21
такой расширяется,

45:23
меняется мышление.

45:24
Да, и я вот именно, что

45:26
себе выработал,

45:28
что не отторгать.

45:30
То есть некоторые смотрят, вот, да это

45:32
сказки, да так не бывает, да что-то

45:34
вот сочиняешь ерунду там,

45:37
ну вот допустим, что какие-то

45:38
возможности у людей, да,

45:40
сверхъестественные есть. Я допускаю,

45:43
и я исправниваю как с кругом.

45:45
То есть ты вот принимаешь,

45:46
обрабатываешь и возвращаешь.

45:49
То есть ты не сразу

45:50
отталкиваешь, а ты, ага, можешь

45:52
что-то в этом есть, так, ага.

45:54
Как бы, что-то взял, и потом раз,

45:57
и как бы уже

45:58
отдачу другую совсем даешь.

46:01
Сразу не отторгаешь.

46:03
Я вот такое восприятие стал делать.

46:04
Говорит что-то, а я его слушаю.

46:07
Слушаю, ага, может что-то

46:08
реальное в этом есть.

46:10
Все. И так же, допустим,

46:13
в споре, да, ты, допустим,

46:14
можешь сразу отреагировать, а можешь

46:16
допустим промолчать, просто молчишь

46:18
и все. И ты даешь своему собеседнику

46:21
возможность тоже

46:22
как бы сообразить, ага, может

46:25
что-то воспалил. То есть его,

46:27
это тоже его охладит, да.

46:30
Резкость вот эту

46:31
убираешь. И когда ты допускаешь,

46:33
я вот стал допускать, что вот

46:35
тебе эти цыганевцы, эти мантры,

46:37
да, мантры петь.

46:39
Мантры, ты что, с ума сошел?

46:42
А я пою дома.

46:44
У меня эти листочки я нашел,

46:45
я пою. А мне пофиг, мне эти

46:47
соседи. Я оттуда съеду.

46:55
Главное, чтобы они не позвонили.

46:57
Такой пример.

46:58
С проверкой, да.

47:01
Я говорю, я ко всему, все воспринимаю.

47:04
Все, что есть в мире, я

47:06
а может быть и что-то

47:08
в этом есть. Конечно, конечно.

47:09
А как еще мир узнать, да?

47:11
Если я буду все отрицать, принимать только то,

47:14
что я там, типа, считаю правильно.

47:16
Будет вот такой маленький

47:18
мирок. Да, я кому-то вот говорю,

47:20
я говорю, вот дома прихожу, мантры пою.

47:22
Все такие, а соседи что скажут?

47:24
А-а-а.

47:26
Я говорю, да мне пофигу на них.

47:28
Да. Мне пофиг

47:30
на соседей.

47:32
Так же, как ребенок маленький по улице, ты пример

47:34
приводила, да бежит, орет. Так и я, как

47:36
ребенок маленький. Кто мне что скажет?

47:45
А?

47:53
Как раз вот этого ребенка включать надо

47:55
чаще. Тогда жизнь

47:58
может измениться. Именно, что

48:04
мне нравится в путешествии, то, что ты можешь

48:06
представить в другом качестве.

48:08
То есть, когда ты, допустим, в какой-то

48:10
определенной атмосфере, от тебя

48:12
уже как бы ждут чего-то,

48:14
и ты от других людей ждешь чего-то.

48:16
А когда ты едешь везде,

48:18
ты можешь играть роль другую,

48:21
которую ты хотел.

48:22
Потому что люди тебя видят первый раз, и ты можешь уже

48:24
в этой роли выставить, попробовать, как это будет

48:26
выглядеть. Потому что пока ты не попробуешь,

48:28
ты не поймешь. Я вот, например,

48:30
анализирую некоторые свои действия,

48:32
например, вот у меня что-то

48:35
вот, какой-то интерес, да.

48:36
Я пришел на эту группу английского.

48:39
Нас посадили, напротив

48:41
вот там сидела девушка.

48:43
Ну, мы с ней там, надо было

48:44
слова.

48:45
Я там это выписывал.

48:46
Я такой, а ты чем занимаешься?

48:49
А где училась?

48:50
Она такая, я на математическом училась.

48:53
Я ей сразу давай писать.

48:55
А вот эту задачу сможешь решить?

48:57
Она такая, блин, да отстань, давай эти слова.

48:59
А я потом уже думаю, пришел, так

49:00
сижу, ну, про себя думаю.

49:03
Думаю, блин, нафиг ты

49:04
в натуре с этой задачей вообще

49:06
тебе она в голову пришла. Потому что ты

49:08
ее знаешь, и типа ты стал уже

49:10
всех этой задачей досаждать.

49:12
Типа у нее красивое решение, ты тут же начал.

49:15
Первому встречному, который сказал, что

49:16
знает математику, ее втюхивает.

49:19
Типа, ну, это вообще поведение

49:21
какое-то.

49:22
Ну, вот.

49:25
Поведение какое-то реально.

49:27
Я вот, когда прошел это поведение,

49:29
я его уже могу осмыслять.

49:31
А пока я его не прошел,

49:33
мне не с чем сравнить. И поэтому

49:34
когда вот я езжу, да,

49:36
я могу себя как бы

49:38
разные примерять, как бы костюмы, да,

49:40
и потом смотреть, как оно сидит,

49:42
грубо говоря.

49:48
И знаешь, очень интересно,

49:50
я как бы сейчас начинаю додумывать, то есть

49:52
в плане того, что, то есть, ну,

49:54
есть выражение, что будьте с собой, не одевайте

49:56
там маски, да, как бы вот эти моменты.

49:58
Это да, вот ты в этой задаче тоже

50:00
был собой. Но ты потом

50:02
это проанализировал и понял, что как бы

50:04
это, ну, не всегда уместно.

50:06
Да, да, да. Я настоящий,

50:08
да, то есть как бы это мне уже больше не надо

50:10
так. Да. Ну, как будто да.

50:12
И это, ну, на самом деле анализ

50:14
очень важен. То есть он не просто там с собой

50:16
позволяет быть, а еще,

50:18
где-то, ну, каким-то,

50:20
не знаю. Да, и более того, ты же, ты же

50:22
меняешься постоянно. Да, да.

50:24
Ты, ты, ты, как говорится, сам,

50:26
ты разный. Вот здесь. Здесь?

50:28
Вот здесь каждый день. Да. Ты буквально

50:30
меняешься каждый день. Вот.

50:32
И ты не статичен, и твои мировоззрения

50:34
меняются. Поэтому, что такое

50:36
будь собой? Собой это понятие

50:38
изменчивое. Конечно, каждый день ты

50:40
меняешься. Да. Главное, что про это как бы

50:42
думать, а не так вот думать.

50:45
Да.

50:47
Вам сейчас вот

50:48
Ладно, рад был.

50:52
Фотку покажу.

50:54
Ну, я

50:55
2-го уже освобожусь.

50:57
И у меня до 6-го будет еще отпуск.

50:59
Можно будет в этот период.

51:01
Ну, да. Напишешь мне, ладно?

51:03
Ну, все, пока.
