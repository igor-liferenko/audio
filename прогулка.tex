Эпиграф: ``Мерой твоего невежества служит глубина твоей веры в несправедливость и человеческую трагедию. То, что гусеница называет концом света, Учитель называет бабочкой.'' Ричард Бах. «Иллюзии».
История была интересная. Я во Вражке ночевал, то есть я в 13-го поехал в Вражку.
Ночевал утром, меня родители поздравили. Я забрался, на тренировку поехал.
У меня должна была тренировка быть в 7.30, я хотел на 6.44 ехать. Я проснулся в 4 часа.
Не хочу дома сидеть, поеду на 5.34 пораньше. А это у нас там на
Павлодарах, где вот хоккейный корт, там площадка спортивная. Мы там должны были с
Мариной Львовной тренировку делать. Она же ездила на Сибирск тоже, и она училась у
этого же. И мы с ней хотели по его методике, которую я нашлюсь, которую она ездила
передать, по ней позаниматься, по этой же системе. Договорились, значит. Я приехал на час
раньше, там поделал звуки, поделал. Ну там специальные звуки, короче, делаешь. И цыбун поделал.
Да.
18 форм. Потом она подошла, мы там с ней эти круги эти-то, ну всю эту программу провели.
И так хорошо, знаешь, все так срослось, так замечательно. Я потом пошел, ну так прям
погода, все так классно. Прям вообще идеал. Я иду, значит, мы все попрощались, она пошла,
я пошел. И пошел по дорожке туда, ну в сторону визит там, где моторинский магазин. И иду мимо
озона, там как раз озон. А я когда на йогу хожу, господь, я мимо него хожу. Я иду мимо озона,
я вспоминаю, что мне надо забрать кроссовки. Как нищий. Озон, кроссовки. Я такой, о, классно,
как раз я вспомнил. Зашел в озон, забрал кроссовки. Такой довольный. Иду домой, значит, думаю,
сейчас приедут там за мной, поедем там. Хотел на озеро Парк, план у меня был там.
Который новый открытый комплекс. Там побыть. Ну потом это,
ну это у меня был план. Потом мама звонит, говорит, папа повез тебе молоко. Ты выйди,
типа спусти, забери. Говорю, ладно. А я как раз к дому подхожу. Думаю, не буду подниматься. Подожду,
пока он привезет. Беру молоко, значит. А он должен уезжать там по делам. Ну, значит,
он привозит молоко. У меня это, кроссовки эти там, рюкзак с тренировки, сумочка эта. Я новую
купил перед этим, поездку. Я туда уже заранее паспорт положил, телефон. Она у меня болтается,
а там в машине что-то, какие-то сумки. Я, чтобы эта сумочка мне не мешала взять, я ее снял и положил
на крышу папе. Там паспорт и телефон. Собрал молоко, думаю, о, как классно все. Поднимаюсь,
тык-тык-тык, где телефон? Соседям стучу. Папе, дайте позвонить. Папе звоню, говорю,
срочно возвращайся, ищи, где телефон упал с крыши. С паспорта. Как ты? Ты что? Паспорт.
Мне уезжать через 5 дней. Короче, звоню Зое. А, пошел я, ну, прошел до набережной, нету телефона
нигде. Ну, сумки-то, да? Сумки. Поднимаюсь, опять соседям этой же стучу, говорю, дайте позвонить.
Звоню уже себе, сразу что-то не сообразил. Звоню на свой телефон, не отвечает. Я Зое звоню, говорю,
Зоя, звони периодически на мой телефон, вдруг кто-нибудь найдет. Ну, что делать? Ну, все, я...
Дома сел и сижу. Проходит где-то полчаса, стучит Рома. Говорит, я твой телефон нашел, говорит,
это. Я говорю, где? На газели, я говорю. Ну, какой газели? Он мне фотку показывает. Стоит, значит,
газель вдоль дороги, вот, которая идет мимо Багуна, я туда, в сторону набережной. Там рядом лужа
огромная. Вдоль дороги эта газель стоит. Значит, телефон висит на зеркале вот этой. Сумка. Сумка.
Висит на зеркале, где машина этой газели. Я так понимаю, папа ехал, телефон выпал. Сумка абсолютно
чистая. То есть, она рядом с лужей, видимо, упала, потому что газель стоит прямо рядом с лужей. Видимо,
кто-то увидел. Подумал, что это из этой газели выпало. Ну, там ничего не было. Паспорт и телефон.
Потому что люди, как правило, боятся такие вещи, с такими вещами играть. Наверное, посмотрели,
денег нет. Ну, может, просто даже порядочно попали. Просто, ну, думали, повесили, куда ее девать. А Рома
просто часы случайно нашел? Нет, он раз прошел. Я так же прошел, не увидел. Ну, все. Потом Зоя отправила
Рому. Он прошел туда-обратно, не увидел. Потом уже стал идти звонить. Идет и звонит. Слышит,
там звонит. Опа, на машине висит телефон. Он его снял, короче, сфотографировал это все. Принес,
говорит. Я такой, о, Рома, блин, ты меня спас. Спасибо. Ну, вообще. Потом, ну, такие, пошли, ладно, пошли к вам. Нафиг
уже это озеро парк. Я довольный такой. А у меня уже все мысли лезут. Я пока дома сижу, что делать? Не
позвонить, ничего. Я не знаю, где, кто, ищут, не ищут. Думаю, сейчас надо ехать, паспорт менять,
скоро уезжать, там, телефон фиг с ним, там, ладно, уже, там, контакты восстановлю. Паспорт где брать,
там, уже мысли лезут горой. Рома такой, я нашел. А, Рома, спасибо. Ничего не надо. Я счастлив, ну. Паспорт войти.
Идем уже, ну, к ним пошли, от Зои. Идем, этой машины, уже нет. То есть, она уже уехала. Буквально,
вот он, за мной сошел, я тут же собрался, мы пошли уже этой машине идти. Я вообще говорю, книжки не
напишешь. И мы пошли к ним, я там зашел. Хотя, какое-то у тебя обнуление было. Вообще. Ты знаешь,
шок-терапия. Вот, реально. Ну, это, прикинь, потерять паспорт. Ну, то есть, это, как бы,
идентификация. И вообще, мы пошли грешить, да? Ну да, конечно, со мной. Уже мы вышли, да? Ну, не, тогда,
Идентификация личности, да, как будто.
И телефон, да, так как бы тоже
мое пространство. Да. И как будто тебе в день рождения
как бы так. Да. Такая перезагрузка
произошла. Вообще. Так интересно.
Прям, да, не говори, классно.
Я бы так прям порадовалась. Ну, понятно,
что я порадовалась, что это нашлось в плане,
а если бы не нашлось, наверное, это не сильно радостно, но тем не менее.
Да, это реально шок терапии
вообще. Вообще.
Вот мы к ним пришли,
значит,
там я, ну, в магазин по пути зашел,
набрал мороженого, сок там у нас, мы по чаю подпили.
Потом я говорю, Рома, никуда
не идем, короче. Не, и так
хорошо. Давай, включай кино.
Поставили мы игру в кальмара,
второй сезон. Я сидел
до четырех часов, мы с ним смотрели
игру в кальмара, смотрели четыре серии.
Потом Зоя там приехала,
говорит, собирайтесь,
поедем во Вражку.
Ну, мы заехали там. А перед этим
еще Зоя, когда мы с Ромой
пришли, она перед тем, как уехать, это суши
заказала на вечер. Мы поехали,
забрали суши, поехали
во Вражку, потом там папа приехал,
короче, суши это посидели. И вечером
они меня в город уже
увезли.
Вообще капец.
А сегодня я с утра
пораньше пошел
на эту тренировку.
И главное, как раз тренировка закончилась,
я домой сбегал, переоделся и как раз
успел в ДК вообще. Прям вообще
все. Успел везде.
А я приходил к Татьяне,
отдала книгу,
в понедельник. Она вышла с отпуска
в понедельник. Татьяна Николаевна.
Я думаю, отдам ей книгу,
потому что я сейчас, во-первых, не буду
появляться до отъезда, потом еще
в ответе, пока буду она мне потерять.
Вот ходил каждый день.
Я теперь пропал, книгу взял и пропал.
И потому что...
Ну, а я как раз ее уже прочитал.
Ну, а в конце уже маленько торопился, чтобы
как раз ей отдать. Ну, вот она
первый день за отпуск вышла, я к ней в понедельник
девятого пришел.
Книжку эту отдал.
Сказал, что я в восход
теперь хожу. И
уже еще до отпуска я не решил,
как. Но потом
в понедельник буду определиться, куда
ходить. Определюсь.
Ну и, короче говоря, мы с ней как-то
поговорили нормально.
Так уже, знаешь, когда я на занятия хожу,
это одно. А тут я в ней кабинете
все. Мы там все, там
тости-боси. Уже
такое, другое маленькое формат.
Она там порассказывала с вами.
Ну, интересно.
Получше узнал.
Вот. Ну, вообще, вот у меня
как бы с Мариной Львовной
еще нормальный вот этот контакт.
Как бы, что мы съездили,
да, в одну школу.
И вот она мне тоже там кое-какие
штучки рассказывала.
Примерно.
Так.
У меня мама, например, тоже иногда что-то вышла.
Два, может, за шестьдесят. Я даже не знаю.
Потому что у меня, в качестве представления,
было, когда мама мне рассказывала,
какое-то время там не было, что она
вообще там глубокая чуть ли не бабушка.
Ну, в плане, ну, бодрая. Ну, как боже.
Лет восемьдесят. Ну, это как бы.
Потом это как бы она,
я прислушиваюсь, что там у нее
дочка, внучка. Я понимаю,
что внучка это явно, что, ну, не правнучка
же там, если восемьдесят лет, да.
Наверное, это где-то все-таки помоложе.
И вот мне всегда интересно было.
Я ее знать не знаю,
но по ней я постоянно слышу. То от мамы,
то от тебя.
Ну, что за человек такой интересный. Ну, да, наверное, мы шестьдесят.
Там где-то есть.
Ну, мы с ней уже сколько раз.
Собирались.
Вот два раза там,
на Павлодарах, потом
здесь, вот,
с ихними, с ихней бандой.
А вдвоем мы с ней только два раза
тет на тет собирались.
Ну, так чисто вот, что в Новосибирске,
в которой я ездил, по этой программе
позанимались. Я там ей
еще скинул
это то, что... А, она мне
видео скинула, в которое она ездила.
Там вот мастер, он
диск. То есть, то, что мы делали,
этот же самый мужик,
он записал видео, а я даже не знал.
А она перед этим ездила, а он
предлагал купить. Она купила.
А я там пытался в блокнот
записывать это все дело.
А она такая говорит, а у меня...
Ну, я потом в итоге все эти записи
потерял. Короче, я
не сильно, в принципе, расстроился.
Думаю, ну, как бы так.
Потому что этих...
Там постоянно везде в интернете.
Думаю, ладно, переживу.
А потом она
мне, значит,
говорит, я вот купила
у него эти... Вот, видео.
...твое время, сколько ты посвящаешь
вообще твоему, как бы, этому...
Ну, твоим способностям.
Ну, да.
...потенциал, да, по сути.
То есть, я так думаю, здесь как бы
никто не сможет остановить, кроме тебя самого.
Насколько ты там углубляешься и тебе это...
А мне, по большому счету,
я такой подход
на...
на твою жизнь.
То есть, я расслабленный, не для кого.
Так вот так и нужно. Как сделал?
Да, Игорь. Ну, конечно.
Сто процентов. Потому что, когда напрягаешься и думаешь,
у тебя там правильно, неправильно ли сделать.
Ну, то есть, ничего хорошего там нет.
У меня как на саксофоне тоже постоянно.
Я же там напрягаюсь, там всё...
Для меня и так
дышать, как бы, это не особо...
Ну, как бы, всегда с дыхалкой
эти проблемы, что всё детство там, да.
А тут, как бы, нужно специально дышать,
там, вдыхивать. Я напрягаюсь.
И я такой...
Расслабьтесь.
Не могу.
Не могу.
Я прям напряжение играю постоянно.
А сегодня, вот, мы делали,
когда эти занятия были,
я себя вспоминаю,
какое я первое занятие пришёл.
Вообще, дубовые, вообще,
элементарных вещей не мог сделать.
И сейчас мне кажется, что
или это я такой тормозной был,
или это просто что новый
предмет для мозга. А для меня это кажется...
Как ты раньше этого не мог сразу понять?
Это же так всё просто.
Нет, это сейчас... Нет, естественно, что мозг...
Ему нужно маленько перевернуться
на новые вот эти шаги.
Для него это всё же новое.
Ну да. Во-первых, ещё, смотри,
у нас же психика очень статичная, она не любит изменений.
И вот ты пришёл на йогу,
да, получается.
А для мозга это опасность, для психики.
То есть что-то новое, это всегда плохо.
Вот для психики и мозга новое плохо.
Почему человек в новое очень сложно иногда идёт?
Потому что это для мозга небезопасно,
для психики. И когда ты начинаешь делать,
начинается сопротивление вначале.
Даже, ну, получается,
на уровне головы сопротивление.
И, естественно, тело тоже не даёт
это до конца сделать.
А когда ты после даже первого занятия
выжил, ну, условно говоря,
что с тобой ничего не произошло плохого,
и тогда пошло уже всё,
это можно, расслабление,
и как бы дальше уже и принятие.
Вначале у нас всему новому мозг
сопротивляется всегда.
Почему люди бросают, не начинают?
Потому что психика защищает человека.
Почему там некоторые живут
в семье, где им плохо,
ну, реально плохо, там, в паре, да, что-то,
но они живут дальше,
потому что менять страшно.
Я здесь живу, мне вроде как я живой,
потому что для психики всегда, ну,
умереть там дальше идёт за всем этим.
Я живой, живу, да, мне плохо,
но я живой. А если я уйду,
то я могу умереть.
И всё. И психика не даёт, пока
вообще там жопа не станет, ну, я так уже условно говорю.
Тогда уже человек
какие-то решаются шаги. А может, они решаются,
может, так всю жизнь живут.
И для вот этого так работает.
Потому что если бы этого не было, сопротивления,
человек бы там всё бы там новое, фу, фу.
Ну, мало что так идёт.
Работают на работах,
домой приходят,
ничего там, ну, не меняют.
Ну, всё вот так.
Безопасно,
ну, вроде и плохо, ну, вроде и нормально.
Ну, то есть, вот эти вещи.
И поэтому тогда вот реально тело,
оно на уровне, ну, как бы
мозг не давал телу до конца
всё это
сделать, вот эти асаны, упражнения.
И даже вот сегодня
на тренировке там некоторые,
ну, новички совсем, видимо, были.
Они что-то поначалу
попытались, а потом просто сели
на лавочке тупо и сидели, смотрели.
Они вообще не могут понять, что это такое.
Ну, потому что, говорю, что
ну, это, это либо нужно сразу
с таким идти, ну, как бы доверие,
но тем не менее, что там остаётся сопротивление, да,
как у этого человека всегда.
Ну, как бы больше.
А когда этого нет доверия,
и человеку это навсегда не понадобится.
И для меня на 50%
именно личность преподавателя сыграла.
Ну, конечно.
Она, ну, притягивает к себе.
А на 50% просто, что вот именно
куда-то пойти.
Что ты вечером приходишь домой, и тебе охота куда-то пойти.
А сама йога это как бы
некий повод просто.
Ну, личность это очень много значит.
Потому что вот даже я сейчас слушаю
по обучению, там, бизнес,
вот же все, психология, да.
То есть 80% успеха,
успеха бизнеса составляет
сам
владелец бизнеса.
Его рост, его развитие,
его вообще восприятие этого мира.
А вот 20% там маркетинги,
всякие вот эти рекламы,
потому что, ну, как бы,
это не как бы насколько,
здесь тоже, насколько энергетика у преподавателя,
что на него идут.
Именно на него идут.
Время найдут всегда.
Найдут там
финансовые моменты, да.
Как бы, если чувствует человек,
что ему нужно, он будет ходить.
Кажется, здесь тоже, наверное,
процентов не 50, даже, наверное, больше.
Возможно.
Что именно идут на учителя
люди.
Ну, да, и вот себя, если
ставишь в шкуру ученика,
на самом деле,
это очень тяжело.
Я этот путь прошел, я считаю, год, да,
годил, и только вот пришло понимание.
А когда, допустим, я учитель,
да, и я вижу,
думаю, думаю, говорю ученику делать,
а в его шкуру залезть очень тяжело.
Это надо полностью
трансформировать сознание, чтобы
в его восприятие войти.
Но
преподаватель всех не может как бы так
и... Объять. Да, поэтому
просто говорит, делайте вот это.
У него же стандартизованный методик.
Делайте так. А что за этим
стоит, он не объясняет. И это правильно.
Потому что каждый ученик свое может
видеть. А если пытаться разжевать,
оно вообще загрузится и все.
Тут главное, что мозг отключить
и просто делать. Тебе говорят, и все.
Постепенно придет все.
Я, в принципе,
так и делал методику.
А у нас же, ну,
много людей сразу требуют
фу, да что это такое, им надо,
чтобы это все объяснили им,
что это для чего. Они думают, да это
странно, да это вообще что попало.
То есть, понятно,
что за этим все это стоит. Но если это
начать развертывать, это
можно уйти там вообще в дебри и
мета оторваться и вообще.
То есть, если это все разжевать.
Ну да. Это знаешь, это же опять же
человеческий контроль. Я делаю,
для меня это странно, непонятно.
И я тогда не смогу это проконтролировать.
То есть, я теряю контроль вообще на ситуации.
Чем я делаю, почему я
делаю, для чего я делаю. И поэтому
человеку всегда важно объяснение всего.
То есть, якобы, что он это проконтролирует
тогда, скорее всего.
А отпустить сложно контроль.
Да. А еще, знаешь, бывает,
что-нибудь делаешь, вот, например, цигунту,
тот же. Кто-нибудь подойдет, так стоит,
смотрит, типа, с таким этим.
Как, типа, ну и
что, помогает, типа?
А чего, интересно?
Ну, типа. Что за эти вопросы, да, стоят?
Вообще, да. То есть, или человек,
вообще непонятно, что он, что он.
То есть, или какой-то
ослиная, или вообще даже
непонятно, что. Вот такого человека
в чем-то убедить вообще невозможно.
Да.
То есть, у него изначально установка
все. Отрицание.
Ну, это, видимо, явно
такой самый
проявленный тип
непринятия чего-то нового.
Да. О споре, что человек
это делает неправильно. Ну, то есть, и что-то
это за этим стоит. А спорить
тут реально. Вот, почему люди, да, не
могут принять в других людях
что-то, что они делают для них, как бы,
необычное. И кажется, и хочется
сразу, что этот человек дурак. Ну, который
делает, да, это. Да он что,
дебил какой-то делает там это.
А я не делаю,
я, как бы, все нормально со мной.
То есть, себя, как бы, так
окрасить и выглядеть. Ну, да, если, например,
в метро, там, ты начнешь, там, улыбаться
во весь рот, там, или, там, песни петь,
тебя сразу попадешь
в категорию ненормального. Да, да, да.
А если газеткой закроешься,
будешь ехать, тогда будет нормально.
Да, вот эта
корма у нас получается очень сильно,
где-то, как бы, это.
Ну, это взрослые же, начиная от детей,
нормировать. И у детей это при,
у некоторых
выживается у них. То есть,
вот не надо, там, на улице, там,
ребенок лежит, там, кричит, песни
поет. А родители, там, что ты орешь,
там, еще. Ну, то есть, им самим не комфортно.
Даже улица, да, по сути, как бы, какая разница.
То есть, начинают
это пресекать. Дома тоже, там, ну,
еще что-то. Конечно, бывают моменты, когда
где-то неуместно так, да, как бы.
То есть, мне кажется, родители очень сильно
детей, иногда, зажимают
в эти моменты. Где-то зажимают,
а где-то в других ситуациях
не дожимают, знаешь, как-то
вот этот момент. И потом человек такой
становится нормированный.
Сильно. Где надо, где не надо.
Ну, то есть, да, это как
некое серое масло получается.
И вот как раз вырабатывается
вот этот вот
тип людей, которые, вот, на работу, с работы
и боятся что-то менять, там.
Такое какое-то, пассивность такая.
Вот.
Ну, да, я вот уже сейчас
стал по этим меркам
рассуждать, что действительно,
как если взять меня,
например, даже.
Ну, и вообще даже
не только меня.
Допустим, если так рассуждать
трезво, да.
Человек находится как пленник.
Вот у него вот такие настолько
узкие рамки, то есть вот работа, дом,
это вообще жесть, вообще
невообразимо даже.
И тем не менее, некоторые люди так живут.
Пытаются
что-то изменить.
Вот это, со стороны смотришь,
блин, это так дико вообще.
Одно и то же, одно и то же,
одно и то же, вообще.
А знаешь, почему так жить нельзя?
Ну, в плане того, что до поры,
до времени все это.
Вообще, как вот я слышала, что человек пришел
на Землю для того, чтобы развиваться.
Всегда. Ну, вот прям каждый
это не касается, там, школа и все,
я закончила свое развитие, да.
То есть всегда развиваться.
Когда человек живет в таком, ну, как бы болоте,
по сути, да, когда все там у него все стабилизировалось,
только, конечно, работа, дом, все.
И ничего нового не появляется.
И тогда мир, там, Бог, Вселенная
того, что доверит, да, обязательно
какую-то ситуацию провернет,
там, какой-то кризис произойдет в чем-то,
в семейных отношениях, на работе, там,
вот хоть где, там, с друзьями, там, здоровье.
Чтобы человек маленько
встряхнулся и начал
что-то как будто бы, ну, менять, да, в жизни
в своей, там, ну, в своей жизни,
чтобы он как-то это маленько
из болота выбрался.
И вот когда человек меняется,
сам, то есть, делает шаги, там, учится,
там, на тренировки ходит, там, какие-то
вот эти, тогда жизнь не будет ему таких
стрессовых ситуаций, ну, часто давать, по крайней
мере. Понимаешь, как это работает? Если я
делаю шаги, чтобы меняться постоянно,
меня не будет Вселенная, там, Бог, учить
меняться, ну, как бы, будет. Я сам
выбирать буду свои стрессовые ситуации,
по сути, которые я, ну, пройду.
Вот ты едешь из путешествия, тоже, по сути, стрессовая
ситуация. Там поезд.
Нужно успеть, там, до этого автобуса.
Неизвестные люди. Неизвестные люди, да.
Ну, как бы, придет на такой стресс, ну, ты на это идешь,
и ты себе создаешь ситуацию, что будет,
ну, развиваться в этом плане.
И это так надо жить, в плане, как бы,
иначе тебя, там, будут
проучивать, как говорится.
Да.
Это прямо, вот, я так хорошо запомнила, думаю, постоянно
нужно что-то с собой, ну, как бы, где-то
пройтись и менять.
Мне кажется, еще знаешь, что тренировки,
которые, там, для тела, такие,
ну, полезные, там, нужные,
там, йога, цивун,
они, мне кажется, развивают еще
гибкость, не только, там, тела,
а такую пластичность,
гибкость нашего ума,
наше, вообще, восприятие.
Насколько ты, вот,
как начал заниматься, стал относиться
к ситуациям более так,
не категорично, а более гибко,
ну, вот, к любому, то есть, я не знаю,
или ничего не поменялось именно
в восприятии, там, как люди себя ведут?
Как это муха укусила?
Ну,
очень сильно
поменялось. И даже я
часто анализирую,
даже, вот, то, что
я говорю, вот, например,
свои действия
стал анализировать.
Может, это и не связано с йогой, но,
по крайней мере, я постоянно пытаюсь
себя исправить. Вчера,
например, вот, Рома не захотел
мне помочь, там, скачать,
этот, музыку я нашел.
Ну, мы сидели, говорю, Рома, давай
поставим эту музыку на закачку.
Я, там, торрент хочу скачать. Он говорит,
я не хочу это устанавливать, программу,
там будет вирус у меня в компьютере.
Я говорю, ну, ладно,
скинь мне торрент на почту, я
во вражку поеду, там,
ноутбуки скачаю. Не стал
давить на него. Хотя мог.
Не, просто, что, ну, думаю,
ладно, не надо, зачем,
не хочет, не хочет. Ну, да.
Вот это тоже же про принятие позиции
другого. Да, а потом, значит, во вражке, когда мы
были, у меня это, я поставил на закачку,
ну, мы, там, пока сидели, я думал,
скачается, оно не успело, там, много.
И скорость маленькая. Я что-то так
маленько, ну, раздраженный
был. Говорю, Рома,
почему ты не поставил мне? Мы бы
скачали, там, у кого у тебя были,
фильм смотрели. Он говорит,
да, я, говорит, это, качал.
И я ему начал,
да ты, блин, не понимаешь, что ли,
это, ты качаешь, там, что попало,
свои торренты с вирусами, а тут
это все, ну,
нормальный торрент сайт, я, так,
все это официально. Это,
ну, это, там, музыка, что
ну, что,
ну, а он мне говорит, я же не знаю,
как это все работает. Я ему говорю, да я знаю,
как это работает, слушай меня.
Потом, ну, ладно, а потом
думаю про себя. Надо было ему
объяснить, Рома, торрент, оно работает
так. Там, вот, есть такая система,
что ты, вот, это, это идет туда,
это так, это так. Ну, объяснить ему
подоплеку, ну, как
это все организовано. А я ему сказал,
да ты делай так, потому что я знаю, что это
так. Так импульсивно
я ему сказал.
Потом думаю, ну, как бы это ни к чему
же не приведет все равно. А так я ему
расскажу, как этот торрент и все это
работает. Он, наоборот, мне скажет
спасибо, что ты мне все объяснил,
теперь мне все понятно, я тебе это могу,
скачаю, все.
Ну, видишь, может, когда вот такая эмоция
поднимается, вот это раздражение, да,
мы себя уже не можем отконтролировать
и правильно и начать эти, делать,
ну, шаги, как бы
говорят, да, действия.
Начинаем срываться, других виноватыми там искать.
Ну, он же, получается, у тебя виноватый,
казалось. Не ты там, да?
А он виноват.
Вот, это вот самое простое, да, тебе виноватого,
и психики такие, фух, не я.
Все со мной нормально.
Это другие.
Вот, а когда ты это анализируешь,
то уже на будущее, возможно, уже по-другому
будешь, как бы, и все.
Хорошо, когда анализ идет за
ситуацией. Да, потому что это тренировка.
Конечно. Ты не можешь сразу
быть таким это уравновешенным.
Это постепенно, да, ты себя
как бы получаешь, получаешь. Мозг
уже, как бы, переводишь
в режим такой более
осторожный, да,
и в следующий раз, когда такая ситуация
возникнет, ты уже, а, я уже про это
думал, так, тихо.
Сейчас будем делать вот так. И постепенно,
постепенно, постепенно это уже
перестраивается именно рефлексорно.
Угу, да. Идет уже
эта наработка, как бы, да.
Потому что сразу нельзя стать таким
спокойным. Просто самое, знаешь, самое
хорошо, очень, мне кажется, редко.
Прям редко, по моим наблюдениям,
люди анализируют
ситуацию. Вот прям
я, почему
в таком, как бы, они не могут
выбраться. Они вот, типа, по этому кругу
кружатся, по своим
стратегиям, да, действиям. Вот они
раздраженные, вот они там сказали,
нашли виноватого, и все. И они потом
просто ситуацию не анализируют. Они оставляют
виноватого этого человека,
и следующая ситуация будет такой же.
Будет такой же, будет такой же.
То есть они себя там не увидят,
ну, как бы, условно, плохим.
Ну, условно, да, говоришь. Ну, каким-то таким, что
я это неправильно там поступила.
И это
очень такой классный навык, когда
свои поступки потом анализируют.
Ну, свое, такие, проживание,
события. Вот. Это,
ну, очень редкие люди. Очень.
Мне кажется, это
процентов 15. Всех людей,
которые живут. Ну, потому что это работа.
Это так же, как растяжка. Ты не можешь
сразу стать гибким там.
Тебе надо постепенно тянуться, тянуться,
тянуться, тянуться. Так же, как
и мускулы, за один день ты не можешь выросить.
Да, да, да. Ты должен их накачивать, накачивать.
Ну, по самом деле, вот эти практики, которые там
физика, да, какие-то
моменты понимаешь, что это постепенно,
это потихоньку. И так же, вот,
своими
стратегиями, с каким-то поведением
начинаешь тоже анализировать.
То есть, это навык какой-то. Тренировка, тренировка,
да. Даже псих. Тренируешь
то же самое, как мышцы. Все одинаково.
как так же как ты навык любой тренируешь тот же третий сайт или там когда там на
велосипеде ездить что угодно на машине сначала там это чуть-чуть потом больше
больше потом уже все это никогда на тренируется но уже сам муфтик
люди живут стратегиях я от свидетель насколько люди не желают вообще свое
мнение категоричность а мир понимаешь он не терпит они же считают что они правы
там нет они что-то несправедливо да а вообще есть такое понятие что нет
справедливости праведливости не существует у каждого на будет от правда
у каждого своя правда у каждого своя а истина деда там далеко от которой мы даже
не знаю а мы-то хотим свою правду навязать я когда вот стал вот сейчас как
раз ты спросила когда
стал заниматься угоном что поменялось поменялось вот что я именно раньше
воспринимал что вот я считаю что вот должно быть вот как каким-то образом вот
у меня какое-то мнение и если это под моим не не укладывается значит я считаю
что со мной поступили несправедливо и и все с кем не игра а сейчас уже когда я
стал заниматься я
пришел к такому видению что все не надо там кому-то завидовать или как это сказать
что наказывать то есть например если кому-то какие-то блага до делают а тебе
не а тебе не делать то ты уже не относишься к этому человеку которым
делать блага негативно вот с негативного отношения у меня уже не
то есть хорошо ему и хорошо а то что у меня не сказаться значит я не смог найти
тот вариант который устроил да и вот это мне понимание пришло то есть есть некая
как бы море вероятности возможности и ты должен исходя из того что есть вот у
тебя выстроить все так оптимально чтобы сложить карту свою пользу вот то есть и
ты у тебя видишь как что он снова Breakups куратор Landes der Grund Кавказской школе
если у тебя не складся значит ты что-то не дадибл là думаю как как сделаем вот как
таком плане перестроился тоже неправильно� vibrio нати виноватого то том что он виноват
того что у него получилось ну как бы так и все и тебе ничего делать мира human league
dahli Muse отлич thousands stand to me прям я потому уверена что когда человек начинает чем-то заниматься эта физика
физика, да, как-то свою физику менять.
Ну, не то, что там...
Может, и любое занятие физикой.
То есть, оно как-то меняется
внутри человека чуть-чуть.
Не чуть-чуть много, ну, не важно,
как оно меняется, но как-то меняется.
Физику всегда когда-то...
То есть, знаешь, есть же такое, что
изменения идут изнутри.
То есть, человек меняется, и внешнее
меняется, ну, мир, условно говоря.
Но у меня я прям убеждена, что
если мы с физикой начинаем работать,
то внутри тоже что-то поменяется у нас.
Сто процентов. Ну, какая-то наша
составляющая, не только тело.
И вообще, даже я тебе больше скажу,
физика — это
основа. Ум, но он, наоборот,
паразит. Он не нужен
вообще, по идее.
Ну, я даже не про ум, я про вот то, что
наше внутреннее состояние, вот как ты сейчас
говоришь, да, что перестала там завидовать,
да. Вот это меняется.
Какое-то восприятие вообще мира
тоже от физики может поменяться.
Вот.
То есть, как ты его принимаешь, более гибким становишься.
Ну, у меня прям, мне прям гибкость
идёт про то, что я принимаю
то, что в мире происходит.
Без каких-то там своих
навязываний. То есть,
вот ты тоже, да, человека получилась.
То есть, не он в этом там виноват, да, ну, условно.
А я ищу
решение, чтобы мне тоже получилось.
То есть, в себе разворачиваю.
В себе всегда. Что?
Почему? Да, и ты
оставляешь за собой
возможность
улучшения и изменения. А когда
ты, как бы, всех обвиняешь,
всё, ты сел, всё, ты
с собой не справедливый,
тебя обидели, тебе ничего делать не надо.
Всё. Ну, это же тоже, блин, мне кажется,
большая часть людей так делает.
Я, типа, всё, типа,
свои дела сделал, типа,
я ни в чём не виноват. Мне ничего не надо делать.
Да.
И ответственность
снимается, но...
И, как бы, получается, коридор вот этих вот
действий, возможностей, он
закрывается. А тут он наоборот.
Ты должен, ты понимаешь что-то,
и ты что-то делаешь.
Интересно. Делаешь, делаешь, делаешь.
Когда ты понимаешь, когда я понимаю,
что от меня зависит моя жизнь,
как складывается мир, да,
то я же понимаю, что
если я сижу
просто сидя, ничего не произойдёт.
Я должна какие-то делать шаги
постоянно. Шаги, шаги, шаги.
То, что я хочу. Да.
И так же вот, как замкнутый круг,
вот, когда человек, ну вот,
яркий пример, далеко ходить надо
мои родители. Они
постоянно ищут друг друга виноватыми,
и оно идёт бесконечный круг.
Да, конечно. Бесконечный.
И я поражаюсь, ну, до чего однотипное
у них мышление,
это просто. Ну, хотя бы один кто-то
проявил какое-то, не знаю,
творчество, нетворчество, просто
некое, да даже, какой-то
разум или там, не знаю.
Ну, знаешь, потому что... Одно и то же,
одно и то же. Им нужно
этот путь прожить. Им информация новая
не идёт, потому что им пока
это не надо, и они к этому не готовы.
И, то есть, им надо вот почему-то
такую судьбу прожить, без вот каких-то вот этих
новшеств, ну, как бы, рябких изменений.
Ну, почему-то вот так.
Ну, да, вот как робот, реально.
Ты маленько хоть сверни, что ты
едешь, вот как упёртый, вот по этому.
Ну, потому что так уже привык.
Ну, как бы уже, понимаешь, насколько там,
знаешь, для примера,
нам приводили. Представьте
вот поле, да, поле,
и вот эта вот дорога идёт,
асфальтированная, да, это вот ваши навыки.
А там просто трава вот такая вот, да,
там какие-то ещё. И чтобы
новый навык, надо просто там какой-то, ну,
даже, может, какой-то бурелом,
там ещё что-то, там нужно приложить усилия.
То понятно, мне здесь комфортно идти.
Ну, да. То есть, по этой дороге
зачем я туда полезу.
И нет веры, что что-то поменяется.
Да я смотрю на них, я это понимаю.
Обвильнуть один сантиметр чуть-чуть
не могут. Опасно.
Не то, что в бурелом, просто на самой
этой дороге, но чуть-чуть в сторону.
Хотя бы, да.
Чуть-чуть обвильнуть вот так.
Даже этого не могут.
Я вот поражаюсь, вот реально.
Вот одна, стереотип
вот один и тот же. Каждый
друг друга обвиняет
и бесполезно. Всё. Это вообще
жесть. Я не знаю, как так жить можно.
Ну, всё-таки, знаешь, я,
ну, ко мне приходят, там, на консультации,
всё-таки, я уверена, что
изменения начинаются
с женщины.
Я это и себе тоже, ну, как бы, скажу.
В плане того, что
когда женщина начинает меняться,
мужчина тоже поменяется. Ну, да.
Это вот прям я уверена.
Конечно, бывает такое, что женщина, да, вроде
как бы, а он, как бы, вот, остаётся
таким, тогда, как бы, ну, нет.
Это вместе уже невозможно, да, допустим. Ну, бывает
такое. Это же, как бы, мы все
меняемся. Нет такого, что
я вышла, вышла, там, замуж, я, там,
да, да. Ну. Может человек меняться
настолько, что мне с ним уже как-то, ну, некомфортно.
И это тоже нормально, в принципе.
Просто вот говорю, что
начинать должна, ну, как должна.
В изменении, конечно, идут
женщины. Как только начнёт меняться,
то мужчина тоже может, как бы,
поменяться. Ну, какие тут эти моменты.
Ну, да. То есть, получается, когда
как бы ты толкнул,
он тебя толкнул. А когда ты чуть-чуть
его сбоку толкнул, он уже сам
волей-неволей сместится.
У него уже другая реакция пойдёт.
То есть, хотя бы один, кто-то
по-другому чуть что-то сделает,
уже картина пойдёт другая.
Может даже мужчина что-то
по-другому сделает. Конечно. Какой-то
триггер у женщины сработает. А, слушай,
он так сделал. Может мне тоже как-то
так сделать? Угу. То есть,
и уже пойдёт другая картина.
И для этого достаточно человеку,
хотя бы как
одному, просто
сделать что-то выходящее
из этого цикла бесконечного.
Да, да, да. Просто, ну, включить.
Хотя бы. Нет, посмотришь, получается, люди,
они не видят, что они в этом цикле.
Они же, ну, внутри-то не видят. По идее, да.
Хорошо. Слушай, правда.
Вот ты сейчас сказала, я вспомнил
историю. Там один
квебек, ну, француз один
с Димой работал. Он из Франции.
Мы с Димой, когда
я звонил ему, он говорит, вот, на вахте
был как раз. Вот, он говорит, подойди
с Игорем поговорить.
Ну, мы с ним чуть-чуть поговорили по-французски.
Я говорю, о, у тебя такой классный французский.
Прям вообще. Он говорит, а с
здесь квебеки, говорит,
меня считают, что у меня вообще стрёмный французский.
Я говорю, да я знаю,
как квебекцы говорят, у них такой ужасный
акцент, они вообще
слушать невозможно.
Он говорит, это ты так кажется. А для них
их язык нормальный, и они тебя считают.
Что ты француз,
у тебя кривой язык.
Я такой думаю, блин, а в натуре
они же не понимают, что
у них не нормально. Для них
их язык нормальный.
А настоящий
француз...
Французский коверкан.
Для многих, для них жизнь
нормальная. Вот для них же их жизнь
таки нормальная. Представляешь?
Да. Вообще.
Да, там, где-то им плохо,
где-то ещё что-то. Но
они считают, что это нормально так жить,
раз они ничего не меняют. Именно как бы вот так.
Но это должен какой-то внутренний
импульс идти, какая-то
природа, что, как это лягушки, да,
вот, допустим,
в пруду, да, они, какая-то лягушка,
вон там за горой ещё круче
пруд есть. А эти сидят,
да, ну, там, здесь, в болоте
в нашем.
Ну, тоже, видишь, человека выбрали, надо его принимать.
И, ну, не лезть там со своими
такими уставами, да,
как говорится. Понятно, что мы даже родители,
ну, после того, что мы родители, то не сможем
сказать, что они неправильно живут.
Да, и наши полномочия. И тут уже думаешь,
думаешь, ну, неужели
вот как человек, он же как
мыслящий, сознательный. А тут
получается уже, думаешь, ну, вообще
человек может, человек вообще мыслящий
существует или нет?
Ни концов. Или он просто
как на рефлексах. Нет, просто для них это нормальная
жизнь. Они, ну, как бы всё.
Ну, это априори. Ну, это как
с французским. Ну, значит, надо это просто
принять. То есть, да, для тебя
это ненормально, а для них это норма.
Какие-нибудь.
Это как бы, как бы это ни казалось. А ещё,
а ещё для себя я,
вот, кто мне говорят все,
вот, ну, там женщины особенно любят
козырять, вот там,
типа,
почему ты не находишь там подружку себе,
ну, для них это кажется, что вот
они, типа, такие всезнающие, вот.
Типа, мужчина должен...
Нет, нет, мужчина должен
их там найти, то есть
мужчина обязательно должен
там кого-то встретить, вот они
такие, значит, женщины,
что без них никуда. Вот.
А я для
себя
такую картину вижу, что
если человек хочет, он
его ничто не остановит. Он может
сделать всё, что он хочет.
Но тут добавляется ещё, ну, опять же,
если он не хочет, то ты ему любые
доводы приводи, он не
хочет. Вот. Он любые
там найдёт отговорки и прочее.
Даже есть поговорка, что
желание – это тысяча причин,
нежелание – это тысяча...
Нежелание – это тысяча причин,
а желание – это тысяча возможностей.
Но для меня тут идёт ещё
уровень, как бы, ниже.
Я заставляю себя,
чтобы убрать
желание,
потому что
я не вижу
на данный момент возможности
их осуществления.
Я как бы, когда убираю желание,
то есть на уровень ниже спускаюсь,
для меня обоснование простое.
Мне не надо там выстраивать какие-то
доводы. Я просто говорю «не хочу».
А потом, допустим,
если я вижу, что подоплёку
я могу выстроить, которая
всё обосновает, я тогда захочу.
Я буду говорить «я хочу».
Получается, довод очень простой.
«Я хочу». «Почему ты это делаешь?»
«Потому что я хочу». «Потому что я не хочу».
И всё просто становится.
Не надо
никому ничего доказывать.
Я даже это слышала, что, типа, когда вас спрашивают,
почему вы там... Ну, у вас, допустим, просят что-то сделать, да?
Ну, просят что-то сделать.
А вы там можете сказать «нет»,
потому что надо обосновать в поле себя.
Не надо обосновывать. Просто «не хочу».
Просто.
Я такая... Я про это запомнила,
когда ты об этом говоришь.
Как всё просто, реально.
«Не хочу». А я тебе говорю, что это будет странным,
как бы, да? В смысле,
ну, как бы, нужна такая веская
причина, что ты мне отказал.
Типа, в моей же картине
мира ты обязан хотеть?
Как? Ты что делаешь?
Это прям большой вот
повестка, когда человек скажет «чему?»
«Тебе хочется? Почему?» «Ну, не хочу».
Сразу ступор такой.
А мы же, понимаешь, почему боимся
сказать-то «нет»? Ну, многие
не хотят. Я в том числе.
Мы не можем сказать «нет».
Потому что я скажу человеку
«не хочу», да, на его просьбу,
а он от меня, как бы, отвернётся.
А для человека быть изнанным из стаи,
ну, условно, это уже смерть.
Как вот древние люди же не могли от ней выжить.
Они же жили из стаи, да?
А, ну да. И вот. И это в древнем нашем
ящерном мозге сохранилась информация,
что для меня страшно, если меня, как бы,
социум от меня отвернётся.
И прикинь, насколько это в нас прошито.
Просто.
Слушай, а может, это и есть корень,
того, что вот человек смиряется
вот с этой вот рутинной жизнью на работу,
с работой?
Может, это и держит? То есть, я работаю,
ну, так принято в социуме, да?
Вот я, ну, тут ещё добавляется того,
что страх, что я умру от голода,
что денег не будет, я не смогу заработать.
Ну, это же тоже то.
Я такой, ящерная наша прошивка
древняя. Вот. И как раз вот эта
ящерная прошивка, она
для меня сработала триггером. То есть,
чтобы этого не бояться,
нужно это допустить.
А раз ты это допускаешь, у тебя
автоматически отходят те
желания, которые могут
вернуть этот страх. Это как из психологии.
Когда у человека, допустим,
есть сильный страх там чего-то,
да? А у нас есть такая поговорка,
что все страхи у человека стоят в очередь
на воплощение. Пока я чего-то
очень сильно боюсь, мир такой,
блин, да хватит уже бояться. Вот это произойдёт,
и она больше не будет бояться.
И тогда эти страхи нужно реально прожить.
Даже себе в голове, прям до конца
пойти. Вот как бы такая,
тоже такой инструмент есть, чтобы этот страх
воплотить даже себе внутри. Что
вот я уволилась с работы, вот я
не могу зарабатывать,
и вот я умерла. Ну и что дальше?
Ну всё, умерла. Ну а что страшного-то, по сути?
Ну как бы,
ну, иногда
может облегчение, но это у кого-то
будет. Ну да, что
страшного, да, умру, а как мои дети?
Ну как дети, там бабушки,
отец есть, там детские дома, да? Ну вроде как
с тобой же они будут жить. Как-то, но
будут. Ну в принципе, да. И вот как раз,
вот я вот к этому сейчас, как раз
восприятию и подхожу, вот именно то, что
ты говоришь. И я вот начал к этому
восприятию подходить, как раз вот, когда я стал
заниматься йогой, потому что
это расширяет вот это восприятие.
Раньше ты видел только это болото, и всё,
для тебя это было болото. А сейчас ты увидел
его с одного ракурса, с другого,
ты уже понимаешь, что есть другой язык,
есть испанский, есть там
португальский, и ты уже, о,
оказывается, вот. Это значит, это вот как раз,
как раньше говорили, и сейчас говорят,
то есть, меняйте своё мышление,
меняется жизнь, да. А мне всегда было непонятно,
в смысле, ну а как менять мышление, ну как, ну как его менять?
А менять мышление, это допускать
варианты другого,
других знаний, других возможностей.
Вот есть такой ракурс сейчас про французский,
да, есть любой французский.
И у тебя хоп такой, мы же все как бы живём
в туннельном своём мышлении, да,
наши знания, вот они здесь выстраиваются, мы других
не допускаем. А когда мы начинаем допускать
другие знания, другие жизни, другие
вообще восприятия жизни, у нас
такой расширяется,
меняется мышление.
Да, и я вот именно, что
себе выработал,
что не отторгать.
То есть некоторые смотрят, вот, да это
сказки, да так не бывает, да что-то
вот сочиняешь ерунду там,
ну вот допустим, что какие-то
возможности у людей, да,
сверхъестественные есть. Я допускаю,
и я исправниваю как с кругом.
То есть ты вот принимаешь,
обрабатываешь и возвращаешь.
То есть ты не сразу
отталкиваешь, а ты, ага, можешь
что-то в этом есть, так, ага.
Как бы, что-то взял, и потом раз,
и как бы уже
отдачу другую совсем даешь.
Сразу не отторгаешь.
Я вот такое восприятие стал делать.
Говорит что-то, а я его слушаю.
Слушаю, ага, может что-то
реальное в этом есть.
Все. И так же, допустим,
в споре, да, ты, допустим,
можешь сразу отреагировать, а можешь
допустим промолчать, просто молчишь
и все. И ты даешь своему собеседнику
возможность тоже
как бы сообразить, ага, может
что-то воспалил. То есть его,
это тоже его охладит, да.
Резкость вот эту
убираешь. И когда ты допускаешь,
я вот стал допускать, что вот
тебе эти цыганевцы, эти мантры,
да, мантры петь.
Мантры, ты что, с ума сошел?
А я пою дома.
У меня эти листочки я нашел,
я пою. А мне пофиг, мне эти
соседи. Я оттуда съеду.
Главное, чтобы они не позвонили.
Такой пример.
С проверкой, да.
Я говорю, я ко всему, все воспринимаю.
Все, что есть в мире, я
а может быть и что-то
в этом есть. Конечно, конечно.
А как еще мир узнать, да?
Если я буду все отрицать, принимать только то,
что я там, типа, считаю правильно.
Будет вот такой маленький
мирок. Да, я кому-то вот говорю,
я говорю, вот дома прихожу, мантры пою.
Все такие, а соседи что скажут?
А-а-а.
Я говорю, да мне пофигу на них.
Да. Мне пофиг
на соседей.
Так же, как ребенок маленький по улице, ты пример
приводила, да бежит, орет. Так и я, как
ребенок маленький. Кто мне что скажет?
А?
Как раз вот этого ребенка включать надо
чаще. Тогда жизнь
может измениться. Именно, что
мне нравится в путешествии, то, что ты можешь
представить в другом качестве.
То есть, когда ты, допустим, в какой-то
определенной атмосфере, от тебя
уже как бы ждут чего-то,
и ты от других людей ждешь чего-то.
А когда ты едешь везде,
ты можешь играть роль другую,
которую ты хотел.
Потому что люди тебя видят первый раз, и ты можешь уже
в этой роли выставить, попробовать, как это будет
выглядеть. Потому что пока ты не попробуешь,
ты не поймешь. Я вот, например,
анализирую некоторые свои действия,
например, вот у меня что-то
вот, какой-то интерес, да.
Я пришел на эту группу английского.
Нас посадили, напротив
вот там сидела девушка.
Ну, мы с ней там, надо было
слова.
Я там это выписывал.
Я такой, а ты чем занимаешься?
А где училась?
Она такая, я на математическом училась.
Я ей сразу давай писать.
А вот эту задачу сможешь решить?
Она такая, блин, да отстань, давай эти слова.
А я потом уже думаю, пришел, так
сижу, ну, про себя думаю.
Думаю, блин, нафиг ты
в натуре с этой задачей вообще
тебе она в голову пришла. Потому что ты
ее знаешь, и типа ты стал уже
всех этой задачей досаждать.
Типа у нее красивое решение, ты тут же начал.
Первому встречному, который сказал, что
знает математику, ее втюхивает.
Типа, ну, это вообще поведение
какое-то.
Ну, вот.
Поведение какое-то реально.
Я вот, когда прошел это поведение,
я его уже могу осмыслять.
А пока я его не прошел,
мне не с чем сравнить. И поэтому
когда вот я езжу, да,
я могу себя как бы
разные примерять, как бы костюмы, да,
и потом смотреть, как оно сидит,
грубо говоря.
И знаешь, очень интересно,
я как бы сейчас начинаю додумывать, то есть
в плане того, что, то есть, ну,
есть выражение, что будьте с собой, не одевайте
там маски, да, как бы вот эти моменты.
Это да, вот ты в этой задаче тоже
был собой. Но ты потом
это проанализировал и понял, что как бы
это, ну, не всегда уместно.
Да, да, да. Я настоящий,
да, то есть как бы это мне уже больше не надо
так. Да. Ну, как будто да.
И это, ну, на самом деле анализ
очень важен. То есть он не просто там с собой
позволяет быть, а еще,
где-то, ну, каким-то,
не знаю. Да, и более того, ты же, ты же
меняешься постоянно. Да, да.
Ты, ты, ты, как говорится, сам,
ты разный. Вот здесь. Здесь?
Вот здесь каждый день. Да. Ты буквально
меняешься каждый день. Вот.
И ты не статичен, и твои мировоззрения
меняются. Поэтому, что такое
будь собой? Собой это понятие
изменчивое. Конечно, каждый день ты
меняешься. Да. Главное, что про это как бы
думать, а не так вот думать.
Да.
Вам сейчас вот
Ладно, рад был.
Фотку покажу.
Ну, я
2-го уже освобожусь.
И у меня до 6-го будет еще отпуск.
Можно будет в этот период.
Ну, да. Напишешь мне, ладно?
Ну, все, пока.
