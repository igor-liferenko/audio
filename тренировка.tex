%&12pt
\pdfpagewidth=297mm
\pdfpageheight=210mm
\pdfhorigin=1in
\pdfvorigin=0pt
\input QUIRE
\shhtotal=\pdfpagewidth
\htotal=.5\shhtotal
\vtotal=\pdfpageheight
\shoutline=0pt
\shstaplewidth=0pt
\shcrop=0pt
\shfootline={}
\shthickness=.27mm
\quire{36}

\horigin=9mm
\hoffset=9mm
\hsize=\htotal \advance\hsize by-2\horigin \advance\hsize by-\hoffset
\advance\hsize by-\QUIRE
\output={\ifodd\pageno\else\hoffset=0pt\fi \plainoutput}

\vorigin=15mm
\vsize=\topskip \advance\vsize by35\baselineskip

\nopagenumbers
\topglue 0pt plus7fil
\centerline{\font\F=omdunh10 at30pt \F ТРЕНИРОВКА}
\vskip 0pt plus10fil
\eject
\shipout\vbox{}
\pageno=1

\font\TENSL=OMSL10
\headline={\rlap{\vbox to0pt{\vss\hbox{\raise5pt\line{\TENSL
  \ifodd\pageno\hfil Разговор на тренировке\kern.5pt\else
  День рождения\hfil\fi}}}}\line{\hrulefill}}
\footline={\raise3pt\line{\hss\tenrm\folio\hss}}

\font\csc=omcsc10 at12pt

\font\speakerF=omssbx10
\def\I{\item{\speakerF И.}}
\def\M{\item{\speakerF М.}}
\setbox0=\hbox{\speakerF А.\enskip}
\parindent=\wd0

\hyphenation{тай-цзи-цю-ань}

\I
Так, я записал эти...

\M
Я видела то, что ты написал, но я не прочитала.

\I
А! Я...\ про другое. Я вот...\ выписал на листок как он\footnote*{Евгений Валерьевич Новиков} там делает.
Я потом красиво запишу, по порядку.
Я смотрел и записывал, чтобы нам ориентироваться.
Буду шпаргалку использовать.

\M
Слушай, а ты на работе-то где-нибудь в закутке ничего не делаешь?
Нет у вас таких мест?

\I
Ну, там есть в раздевалке у нас закуток.

\M
Ну в раздевалке...

\I
Я там делаю этот...\
18 форм делаю, «столбом» чуть-чуть стою.

\M
Не...\ ну, сейчас вот на улицу не ходишь на...

\I
На улицу, ну хожу на Кан вот...

\M
Вот на Кану там, да.

\I
Спускаюсь на обеде, делаю там...
Звуки делаю, 18 форм...

\M
Там чуть-чуть, вот здесь, вот там где эта река течёт, ну не река, а забор воды.

\I
Ну.

\M
А там ещё чуть-чуть пройти, и там тоже есть площадка.
Достаточно уединённая.

\I
О\null! Я не знал.
А тут ходят толпами.

\M
Дааа, да.
А там вот, если зайти, вот так вот
дорога какая-то идёт...

\I
Слева, да?

\M
Влево, да.
Ну, и она прям на берег Кана выходит.

\I
Ааа, ладно.

\M
И там какая-то такая площадка, но там...\ не все ходят туда.
Я как-то ходила, так, давно.
Всё. Я сейчас побрызгаюсь, извините, а то меня вчера наели.
Не хочу больше. Тебе не надо?

\I
Надо. Шею брызнуть. А то капюшон не удобно.
Мешает шею...

\M
Подальше! Зачем ты так? Подальше!

\I
Я уже с полседьмого здесь. Я уже
звуки сделал, 18 форм сделал.

\M
Ну, я бы знала, я б пришла пораньше.
Я думала ты опять со Славой занимаешься.

\I
Нет, сегодня отменил я. Сегодня у меня день рождения.

\M
Ах...

\I
Я говорю: ``Слава...'' Я говорю...

\M
У тебя день рождения?! Поздравляю!

\I
Спасибо.
Я говорю: ``Слава, отменим на завтра --- я говорю --- пропущу.''

\M
Так а ты чего-нибудь собираешься?...\ отмечать.

\I
А мы пойдем с моими...\ ну, близкими родственниками...\ туда, «Озеро-парк».
Там, чисто, посидим. Вот...

\M
Так ты бы сказал, отменили бы и здесь.

\I
Нет, а я сам, для себя...
Я наоборот начал новый год с цигуна. Это хорошо.
Я говорю, я с полседьмого здесь.

\M
Кстати, пардон, сколько лет тебе?

\I
41

\M
41...

\I
Я звуки сделал, сейчас 18 форм сделал.

\M
Не, самое время что-то начать...\ заняться чем-то...
Ну, то есть единственное, знаешь, чтобы... Не кидайся на всё сразу.
Вот, постепенно вникай. Не... Иногда
бывает плохо, когда всё вместе и там «каша» такая. Хорошо, что выбрал учителя.

\I
Так, первое у нас,
значит, мы начинаем... Промывка, потом
шоу-гун, потом собирание, потом раскрытие-закрытие. Перерывы --- шоу-гун.

\M
Ну тогда пораньше разойдёмся. Раз ты уже с полседьмого здесь.

\I
Нет, мне вообще хорошо. Ну, смотрите как вам, я...

\M
Нет, я маме сказала, что в 9 буду. Ну, давай...

\I
На видео он чуть-чуть наклоняется, а на тренировке он прям до середины голени, вот, наклонялся.

\M
Ну вот мы такое
не делали.

\I
Потом он как-то вот движением из дантяня, вот, руки у него {\csc раз} так поднимааает.

\M
А вот так как-то, как это...

\I
Как бы вот из центра {\csc раз} так поднимает.

\M
Я говорю, вот так представляешь, что тебя как верёвочку вниз опускают.

\I
А! Ну.

\M
И они сами получается полетели.
То есть, не руками делаешь...

\I
Девять раз. А я считаю, чтобы
не сбиться: {\it счастье}, вот, {\it здоровье\/} идёт. Второе: {\it здоровье}.
Потом: {\it богатство}. Четвертое:
{\it успех}.
Сейчас {\it здоровье\/} сделали. Потом: {\it богатство}...\
{\it успех}. Второй раз: {\it счастье}...

\M
Ааа...

\I
Уже не собьёшься.
Потом шоу-гун.
Теперь собирание. \hfil\break Шпаргалку подсмотрю сейчас быстренько.

\M
Вначале вот сюда. Как это, забыла я слово...

\I
«Третий глаз» --- инь-тан. Так, сейчас секунду прочитаю, быстренько пробегу.
Так, верхний центр, исходное положение, значит...\ кисти рук у низа
живота на уровне гуань-юань. Это ниже пупка.
Ладони повёрнуты вверх, кончики пальцев смотрят друг на друга.
Руки через стороны вверх до уровня инь-тан, как будто собираем энергию земли и неба,
окружающей среды, в виде энергетического шара и приближаем к инь-тан на расстояние 5 см.
Представьте отверстие в
этой точке, которое сообщается со средним каналом.
Энергетический шар загоняем в это отверстие и проводим вниз в средний центр.
При поднятии вверх --- вдох.
Как бы забираем, да, к себе, вдыхаем. И опускаем --- выдох.
Количество повторений, значит, понятно.
Средний центр, исходное положение...\ так, руки по дуге поднимаются на уровень среднего центра,
образуют кольцо. Вдох. Вдохнули --- к себе, так же.
Поясничный отдел выпрямляется, то есть, как бы поясница выпрямляется, открывается поясница.
Обратное брюшное дыхание используем.
Вот, то есть на вдохе пупок движется к пояснице.
На выдохе живот расслабляется
и брюшная стенка
расслабляется, поясница прогибается.

\M
А, прогибается.

\I
На выдохе прогибается, а на вдохе распрямляется. То есть, живот собирается на вдохе,
диафрагма давит на внутренние органы, поясница распрямляется.
А на выдохе живот расслабляется, поясница выгибается. Значит, третья...

\M
Она не выгибается, она просто перестаёт быть напряженной.

\I
Да. 

\M
То есть её не надо
специально выгибать.

\I
Да. Наполнение нижнего центра. Исходное положение как в предыдущих упражнениях.
Значит, руки на уровне лобковой кости, направлены на... Руки вначале отдаляются от тела,
а затем приближаются к животу и поднимаются до уровня пупка, разводятся
в стороны и опускаются... А, то есть поднимаются, разводятся в стороны и опускаются.
На вдохе энергии из нижнего центра поднимаем в средний.
На выдохе брюшная стенка расслабляется. Дыхание такое
же как когда третий, то есть на вдохе поясница
распрямляется, на выдохе возвращается. Вот.

\M
Тоже девять раз?

\I
Да. Вчера мы делали этот...\ когда «петуха», я вспомнил что видеоролик...
Так, это у нас {\it успех\/} уже, да? Видел ролик, что там
есть три способа выходить из «петуха».

\M
Да их там миллион. У каждой школы.

\I
Да. А вот у вас кстати... Так, это {\it здоровье}, да?
Потом, я видел у
вас на группе аватарчик...
У меня есть книга. Я скачал в интернете. Автор Ван Лин, по-моему, как-то так.
«24 формы тайцзицюань». {\it Успех}, да? Последний раз. И там вот на обложке такая же фотография
как у вас аватар на группе. Вы случайно не из этой книги его?...

\M
Не знаю, там другой администратор делал.

\I
Вот, похоже он из этой книги взял обложку.
Ещё я какую-то
книгу скачал «Осознанное тайцзицюань».

\M
Могу тебе дать ещё одну книжку почитать: «Путь мастера» или что-то такое.
А...\ мы закончили?

\I
Да, да, да.
Я всю эту книжку «Внутренняя структура Тайцзи»...\ у меня есть такое подозрение что переводчик,
который переводил, это просто...\ Ну как. В предмете не ориентируется.
То есть, просто вот ему дали задание и он переводил, потому что там ляпы такие есть.
Я специально полез в интернет поискать в оригинале. Стал сравнивать, а там вдох с выдохом
перепутано кое-где. Ну то есть, ляпы в переводе есть. А вообще это хорошая книга.
Ещё бы вот её раз перечитать, но она такая нудная, там столько всего фильтровать надо.
Упражнения разминочные мне нравятся.
По ним бы как-нибудь ещё позаниматься.
«Подготовительные упражнения».
Я даже страницу запомнил: сотая.
На сотой странице начинается. Интересные там: и этот маховик, который мы делали; потом кручение,
это всё. Так, вроде это сделали.
Потом как-нибудь. Вот. Так значит: мы поднимаем, да? Поясницу распрямляем, вдыхаем,
потом расслабляем, руки опускаем, да?
Пока низко не будем делать. Сначала так привыкнуть.
А я уже между прочим
привык каждый день всё
это делать.

\M
Правильно.

\I
Ну. И вот сегодня я с цигунчика день начал --- так хорошо!

\M
Сейчас у тебя там погружение будет серьёзное.

\I
Ну, я его лекции послушал, ролики посмотрел.
Мне у него подход маленечко...\ я не со всем согласен.
С его подходом в плане...\ в каком же плане?...
Так, вроде всё. Шоу-гун делаем.
Что-то мне у него не понравилось, в его подходе.
Сейчас вспомню что меня зацепило, так прям покоробило.
В принципе он нормально так подаёт, но какую-то он мысль\footnote*{Вспомнил:
пост в телеграме про рекламу фитнес-клуба, по-моему в канале «Создай себя сам».}
высказал, я прям с ним не согласен.
Сейчас не могу вспомнить.

\M
Поедешь --- увидишь.

\I
Так, три раза?

\M
Три.

\I
А, уже сделали? Всё.

\M
Сейчас ты просто сказал и я вспомнила: я когда поехала первый раз на семинар,
я приехала туда одна. А обычно привозят, там, их привозят...\
преподаватель. Кто-то же послал туда. И он должен приехать вместе с ним. Наш, этот,
Юрий Николаевич не смог, я одна поехала. Они все сильно удивлялись, что как это
так я одна припёрлась сюда.
И мне по сути ни один человек там не был знаком. Ну, то есть, приезжаешь в общем...
И занятия утром, полтора часа днём, полтора часа вечером. Они все там что-то как-то едят.
Какие-то разговоры, которые я даже не могу поддержать, потому что я не все даже термины понимаю.
И как-то мне было скучновато там. И я думаю: чё я припёрлась сюда вообще. И...
Ну это во-первых, денег много, там, и далеко: в Рузу аж поехать --- не шуточки.
И я, значит...\ и какой-то день --- пять дней подряд --- и какой-то день я просыпаюсь...
А условия были прям, знаешь, как вот...\ ну не знаю...\ это было когда в лагерях,
в этих вот...\ железные кровати, вот эти вот...\ сетки, которые прям до пола практически.
И вообще жуть какая-то.
Ну, вот...\ столовая тут чё-то, туалеты где-то там...
А, нет, туалет был в комнате.
А душ был где-то --- туда идти...
Холодно всё время.
Ну, в общем...
В зале я надевала просто на себя всё, что у меня было, чтобы не замёрзнуть. Было холодно.
И в какой-то момент я утром просыпаюсь и у меня такое ощущение, что
меня вот так хоп, вот так вот над кроватью подняли.
И я вот так вот в этом пространстве нахожусь над кроватью.
Причём я всё понимаю.
Я понимаю, что время, что скоро надо вставать.
Такое хорошее ощущение!
И вот я понимаю, что если сейчас глаза открою, может быть упаду или что-то.
То есть вот, не хочется его терять.
Вот такое взвешенное
какое-то состояние.
И у меня вот мысль пришла,
вот если бы сейчас сказали: вот сейчас ты умрёшь.
И нормально. Вот в таком состоянии!
Потому что в жизни такого ни разу не было!
И думаю: блин, хоть бы подольше.
Но у меня действительно долго получилось.
И я потом думаю: не зря приехала!
То есть вот эти вот ощущения неприятные:
холодно, неизвестно,
что-то есть, какая-то еда,
они там все что-то запаслись чем-то,
я естественно ничего не привезла с собой.
Ну и ела то, что в столовой только.
Но нормально всё, не голодная.
Но дело в том, что вот...
А ощущение которое ты не ожидал, оно прилетело.
И как бы вообще, я до сих пор помню: было просто хорошо.
И знаешь, ни за кого...
Обычно там страшно: ой, сын, там что-то с ним случится, что-то с родителями --- вообще всё пофиг,
вот.
Всё пофиг, вот, ты просто вот тут вот находишься --- зд\'орово!
Не знаю, было у тебя такое когда-нибудь или нет.
Но вот я там считай, что единственный раз такое вот получила.
И за что спасибо.
Условия абсолютно некомфортные.
Абсолютно некомфортные.
И они там все ему знаешь... Вот этот сидит Балашов, они ему: ой, Николай Николаевич,
Николай Николаевич...
Почему они так перед ним все пресмыкаются?
Вот только посмотри на меня, пожалуйста, потрогай меня, покажи как...
Можно я, можно я с вами... вот так встану?
Почему они так суетятся вокруг него?
А он видимо чем-то обладает, потому что я думаю, что
это от него всё происходило.

\I
Ну я знаю, что вот люди рассказывают, когда вот они в
медитативное состояние входят,
у них такое состояние мира, безопасности...

\M
Хорошо всё, просто вот, не страшно ничего.

\I
Ну это медитативное погружение.

\M
Ну я ничего не делала специально. Я просто начала просыпаться, я полностью всё осуществляла...

\I
Но вообще вы же там погружались в себя, медитировали, да?

\M
Нет.

\I
Ну, там, какие-то особые...

\M
Тренировка была следующая. Ты заходишь
в зал в
этот холодный.
В 7 часов утра.
Вот только с койки встала и оделась.
То есть, там вот в этом же здании был зал.
Мне такое ощущение было градусов 15 в этом зале было.
То есть вот мёрзло всё.
То есть, ты должен был всё время шевелиться хотя бы.
И все вот в этих кофтах каких-то и в этих жилетках там,
потому что... И вот ты встал, в начале там наверное минут 20 или может быть 25 была разминка.
Вот как...\ вот это всё, это всё --- так разгоняешься.
Разогнался. А потом говорит: ``Ну всё, теперь 24 формы.''
Они делают 24 формы. Я вообще не догоняю что они там делают,
потому что я не могу даже это повторить. Мы делали как-то по-другому.
То есть я вообще по-другому делаю.
Потом он мне говорит: ``Так, покажите вот, что-то вы делали. Покажите.''
Ой! Вот представь себе: к тебе подходит твой мастер, тренер, да, и говорит:
``Покажите.''
Все смотрят.
У меня там без...\ деревянный, что там показать.
И он говорит...\ вот так вот посмотрел, говорит: ``А что вы делали-то сейчас?''
Я говорю: ``Ну, вы сказали там...'' Я не помню уже.
Он говорит: ``Ну нет, это совсем нет, нет! Это совсем...\ у кого вы занимались?''
Я говорю: ``Ой, я подвела этого замечательного человека, который мне всё показал.
Я вот такая бездарная, я здесь не могу даже воспроизвести. Почему...''
Понимаешь всё, вот полностью, вот так вот размазал тебя, да, и тебе уже там и холодно...
Господи, а уйти некуда, потому что для этого надо выехать куда-то там.
Во-первых, это населенный пункт, это где-то там на отщеплении.
Пешком не дойдёшь, надо вызывать такси. Вообще, вот это всё собираться. Все увидят.
Собираешься среди курса. И то есть, вот эта вот борьба какая-то с собой бесконечная.
Вот это было, да. Это не медитация.

\I
Это видимо что-то...\ какие-то переключения произвело.

\M
Конечно, конечно.
Думаешь: ну ладно, ты же заплатила за всё.
Билеты у тебя в конце концов не сейчас.
Что-то где-то будешь там в Москве болтаться.
Короче...\ и осталась.
Ладно, всё --- поехали.
Я просто к тому, что готовься, может быть такое будет.
Если будет --- это не плохо, это не плохо, это нормально.

\I
Погуляю по горам.

\M
Ну, во всяком случае...\ нет, не уходи. И даже те...\ вот прям натурально человек
сказал: ``Чё ты вообще сюда припёрлась? Где твой во-первых преподаватель,
почему мы его не видим. А во-вторых, чё ты сюда приехала.''
Я одна приехала.
Обычно там кого-то за руку приводят. И все претензии высказывают преподавателю,
чтобы он потом мягко сказал этому ученику.

\I
Вот, я слышал такую технику: на вдохе как бы надуваем живот, и живот как шарик,
всё тело за ним надувается. Выдох --- сжимается.

\M
Да, и причём это знаешь как...\ я долго не могла понять, что это такое. А
потом, когда костёр задуваешь, не задуваешь, а разжигаешь, он там вот плохо начинает гореть, да.
И ты начинаешь: так дуешь на него.
То есть, ты дуешь и он как бы затихает.
И при этом внутри разгорается. Вот как. Вот мне показалось, что вот эта ассоциация мне помогла.
Он так краснеет, краснеет.
А потом, когда перестаёшь дуть, а он расширяется, начинается пламя.

\I
И глазами следим за руками, то есть тренируем глаза заодно.

\M
Ну ты их не переводишь, а ты как бы одновременно видишь две руки. Не вот так вот сюда,
а сюда...

\I
Да, это как поле зрения увеличивается.
Проверяют когда в больнице.

\M
То есть
смотришь в центр, а видишь руки обязательно.

\I
И представляем что соединяем с усилием и разъединяем с усилием.
По-моему последний остаётся.

\M
И не смыкаешь ни разу. Как магнит...\ не даётся.

\I
Теперь на уровне...\ среднего центра.

\M
Вот это кай-хэ называлось по-моему.

\I
Кай-хэ, да.
Раскрытие, закрытие. Сейчас я быстро шпаргалку прочитаю. Как
его делать правильно. Кай-хэ. Так, три этапа.
Мысленно... Так, инь-тан --- средний центр. Значит...
А! Ну, ладно, это упражнение... Так значит, исходное положение: у-цзи.
Поднимите руки до уровня... В у-цзи могут ноги разведены быть. Поднимите
руки до уровня среднего дантяня в зоне на расстоянии 5--10
сантиметров. Предплечья немного согнуты... Взгляд устремлён...
Удерживая сознание на среднем дантяне... А! Ну это только средний рассказан. Вот,
то есть мы делаем к центру, значит... Мысленно проводится из инь-тан в средний центр,
талия через поясницу поднимается по позвоночнику и выходит в лао-гун на ладонях.
Прочувствуйте связь между ладонями. Когда ладони энергетизированы, в сознании появляется
стремление к раскрытию, а также стремление к раскрытию в дантяне и пояснице.
Тело наполняется ци и каждая пора стремится раскрыться.

\M
Это ты где берёшь?

\I
Это я сайт\footnote*{\tentt chelushu.narod.ru/trad.html} нашел: Челябинская Федерация Ушу.
Там у них список подготовительных упражнений и там
раздел есть «динамические упражнения». И
там вот их по-моему пять упражнений и первые три как раз то, что вот кай-хэ, цхай-ци и шоу-гун.
При выполнении...\ так, раскрытие вдох, закрытие выдох.
Также используется движение складывания в пояснице.
То есть поясница тоже как бы распрямляется и возвращается.

\M
Там, небольшой...

\I
Да. Так, ну средний сделали, ладно. Теперь нижний, да. Как
бы такое движение, я так понимаю...

\M
А вот это мы не делаем, вот это вот?

\I
Это мы делали, только это было как цхай-ци, мы его
чуть-чуть делали. То есть, мы его не до
низа тренировали.
На видео он не сильно наклоняется, а там он прям сильно.

\M
А эти, вот они все делают, вот так вот.

\I
Да, да, да.
Мы на этот раз просто чуть-чуть делали.
Это было цхай-ци, второе.
Теперь мы делаем кай-хэ.
Я так понимаю здесь движение раскрытия похоже на то, как вы вчера говорили,
когда мы напрягаем, не живот вперёд выдаём, а именно низ копчика.
То есть здесь такое же похожее движение, да?

\M
Да.

\I
Ну ещё по-моему последний раз.
Но вообще
конечно, когда занимаешься для себя и когда ты учишь других, то есть публично, публичную роль
играешь, это совсем разные вещи.

\M
Да конечно.
Я говорю, мне конечно надо съездить на какой-нибудь семинар когда ты как ученик там стоишь.
Потому что во-первых ты должен наблюдать, это уже внимание.
И поэтому, когда они говорят: ``А вот она,
смотрите, делает...'' Господи, как у тебя хватает... Была бы возможность ни на кого не смотреть,
я бы и не смотрела.

\I
Так, круги вперед теперь у нас идут.

\M
Ага, давай.

\I
Перед собой. Не, вот так.

\M
А! Вертикально.

\I
Ну, я вчера домой пришёл, включил этот ролик, всё сделал, прям повторял.
Потом ещё раз включил, взял ручку и уже сидел записывал. Я потом это хорошенько напечатаю красиво.

\M
Ты всё уже занимайся как ты поедешь.
Уже сейчас не отвлекайся. Там...\ возможно тебе там что-то другое скажут, но ты не возражай.

\I
Я не, возражаю просто, что...

\M
И не спорь внутри. Вообще, и внутри не спорь.

\I
Да.

\M
`А вот это вот, а я вот знаю по-другому.'

\I
Не, ну мне недолго, я же черновик выписал.
Набить на компьютере недолго. Пригодится всё равно.

\M
10 принципов, вот их надо как бы освоить, 10 принципов тайцзицюань. Они вот точно помогают.
Но они не всегда понятны. То есть определение которое
читаю, я там пыталась выложить. А половину непонятно
вообще, что там написано.

\I
Я вот распечатал, читал, да.

\M
Вот как-то со временем потом {\csc раз} второе. Через какое-то время почитаешь и думаешь:
блин, чего я не понимала?

\I
Ну оно придёт.

\M
Да, здесь не надо торопиться.
Вот это вот точно не надо торопиться.
Торопиться не надо. Не надо подгонять.
Оно само по себе придёт.

\I
А я вот это в принципе усвоил, что не надо торопиться. Я вот на йогу ходил полтора года,
и я ходил, вот просто как... Ну как зомби. То есть, вот я
просто ходил, потому что я...

\M
Ну зачем?

\I
Потому что это...

\M
Полезно.

\I
Нет, даже не знаю зачем.
Потому что я считал, что это некоторая для меня...\ что-то такое...\ отвлечение от будней.
Вот. Я фактически там был, ну как попка-дурак, ну совсем,
абсолютно ничего не понимал, что мы делаем, как мы делаем.
В каких-то шортах припёрся первоначально, в футболке. Все там ходят, а я как этот...\ ёжик
в тумане.
Тем более без очков.
Вообще, там, весь кривой.
Ничего не получалось.
Ничего не понимал, как дышать, как... И я год ходил.
Она там говорит что-то, вот эти термины, там, «три шесть» дыхание, там...
Опять же, там...
Я ничего не понимаю что она говорит.

\M
А чего ты не спрашивал?

\I
А я не люблю спрашивать.

\M
Вот это не правильно.

\I
А вот...
Я просто ходил туда, просто что нечто такое, чтобы куда-то сходить.
Вот вечером домой приходишь, дома скучно, охота куда-то сходить.

\M
Ааа, вот так вот.

\I
Я просто нашел, эту...

\M
Шоу-гун не будем делать
после каждой, или...\
или после вот этих вертикальных?

\I
Сейчас я
посмотрю. Значит это мы сделали, круги на себя, потом круги перед собой вправо и влево.

\M
Так, подожди, мы вот это не будем сейчас?

\I
Это потом.

\M
Ааа, да?

\I
Но сначала вот.

\M
Ааа, а я думала это...

\I
Да. Чтоб куда-то сходить просто и...
Значит...

\M
А как ты не зашёл к этому Юрию Николаевичу, интересно?
А, ты не знал...

\I
А я ходил на массаж, у меня спина болела, и там у них висел плакат, в грязёвке,
заходишь...\ эта...\ йога. Я помню пришёл к ним, заглянул туда. А, узнал в какой день,
пришёл.
Говорю: ``А у вас это...\ можно посмотреть?''
Она говорит: ``Ну приходите.''
Ну я на следующее занятие\footnote*{25 декабря 2023 г.} пришёл. Они... Ну...
Там главное... Ну, они делают, и со стороны может показаться странным что они делают.
Я думаю: ладно, хрен с ним `странно' --- буду делать. Ходил
просто, пофиг.
Ходил, делал что говорят.
Мозг просто главное отключить.
И я вот год ходил. Потом как-то где-то что-то наткнулся в интернете, какой-то термин {\csc раз}.
О\null! А мы это... Она нам говорила это слово.
Я вот за год так, по крупицам.
Оно так интереснее, когда сам ищешь.

\M
Конечно.

\I
Я специально ничего у неё не спрашиваю.
И оно как-то всё само потихоньку, потихоньку пришло. Я потом уже когда
на новую\footnote\dag{В «Восход» к Юлии Фоминой} йогу пришёл --- я уже профессионал.
Мне уже все эти трюки знакомы. Это вот там у них своя, как бы...
У них свои там принципы. Вот. Я их не сразу понял. Я сначала тоже косячил, там чё-то...
И...\ как говорится, диссонанс вносил. То есть, как это сказать...

\M
Нарушал гармонию.

\I
Нарушал...\ сумятицу вносил в процесс обучения. Потом понял, что надо это...\ притихнуть маленько.
Язык за зубы взять.

\M
Так, подожди. А сейчас что мы делаем?

\I
Так, это мы...\ в эту сторону сделали. Так,
сейчас посмотрю...

\M
Шоу-гун.

\I
Нет. Круги теперь «по столу» вправо и влево.
Потом только шоу-гун.

\M
А, то есть вот такие рядом две руки. Надо пройти все три плоскости, а потом сделать шоу-гун.

\I
Вот. Я это... А потом как-то у меня получилось...
А! Она\footnote\ddag{Татьяна Николаевна Шефер} пошла...\ она заболела по-моему.
А у меня сестра
двоюродная, мы как-то разговорились. Я говорю: ``Вот на йогу стал ходить, а она заболела.
Я...'' Мне, говорю, уже не хватает, мне надо куда-то идти. У меня уже,
как это сказать, распорядок, что понедельник, среда я должен куда-то идти.

\M
Нет, ну это нормально.

\I
Я говорю мне надо чем-то, замену какую-то. Она говорит: ``Ну, вот у меня знакомая есть,
она йогу ведёт.'' Я звоню. Она\footnote*{Анна Стивалле} говорит: ``Ну, приходи.''
Ну, я к той походил. Там у них своя маленько программа --- чуть-чуть другая.
Там вообще дыхания никакого у них нету.
Вот...

\M
Вот именно. Во-первых этих йог полно всяких. А во-вторых они, которые получились там где-то
в интернете, они своего напихали туда и что сами давно знали. И каша получается.

\I
И она мне говорит... Ну, я говорю, вот мол я...
Ну а я вообще профан. Потом только я костюм заказал на алиэкспрессе --- думаю надо что-то
поприличнее надевать.
Поискал костюм на озоне, на вайлдберрисе --- нет.
На алиэкспрессе смотрю костюм для йоги. Я выписал. Он мне пришёл год назад.
Специальный костюм для йоги такой классный. Вот. Я уже в этом костюме стал
ходить. Уже как бы маленько это,
себя стал чувствовать уже
в этой тусовке увереннее. Уже там всякие названия
уже потихоньку выучил, как что называется, уже
там с ним на равных стал
уже разговаривать, к которой новой я пришёл. Я говорю: вот мол я хожу, там, в грязелечебницу.
Она: ``Ааа... Там «Шивананда».''
Я такой --- опа!
«Шивананда». Что такое «Шивананда»? Полез потом сам в интернет. Залез ---
оказывается...\ ну, там...

\M
Ну это разновидность.

\I
Хирург это был, он какого-то йога лечил. Тот его научил, значит.
И он потом стал йогой заниматься. Он профессиональный хирург был. И он там достиг больших высот
в йоге. И вот разработал программу...
Шивананда, Свами Шивананда его звали.
И он разработал программу обучения по йоге и назвали
Шивананда-йога.

\M
В другую сторону сделали?

\I
Да. Сейчас шоу-гун.
И вот как раз в грязелечебнице оказывается мы занимались по Шивананда-йоге.
Это я узнал только чуть ли не год спустя. И потом залез в интернет.
Думаю: что же такое Шивананда-йога?
Оказывается она делится на пять частей.
Я уже стал когда сопоставлять, что мы делаем.
И мы действительно эти пять этапов проходим.
То есть, я ходил просто...

\M
Ну понятно.

\I
Потом уже стал я понимать: а это оказывается, вот мы все этапы проходим, то есть, мы это, это делаем.
То есть это оказывается всё не просто так.
Потом уже она вышла, когда из отпуска, я опять к ней стал ходить.
А потом она в этом году опять ушла в отпуск.
Я к той не захотел ходить, мне там не понравилось.
Ну, она сама как-то...\ отношение как-бы,
как-то не пошло в контакт.
Я решил другую поискать. И вот посоветовали...\ вот Анна Вадимовна говорит: ``Там йога
есть.'' Я уже сюда пришёл, ну какой как этот...\ ``Посмотреть.'' Она ``Ну, давайте...''
Всё, мы сделали? ``...давайте посмотрите.'' --- типа того, что пришёл тут лошарик.
Ну я такой на первом же занятии это делаю, это делаю, это делаю --- я это всё знаю.
Эти, названия все: мы вот эту позу делаем, эту позу делаем. Она говорит:
``А вы где занимались?''

\M
``Где только я не занимался.''

\I
Я говорю: ``Ну, вот, так... --- я говорю --- так чисто по верхам
нахватался.'' Ну, потом на второе занятие пришёл.
Так, значит мы делаем круги
правой ногой по полу от себя. Да, вот так.
Вот. То есть я уже как бы знаю, что у них там вот эти все...\ что они там все вот эти асаны делают.
А\null! Потом когда мне сказала вот эта Анна про «Шивананда», она сказала,
что у неё книги...
Вот говорит, у меня даже книга есть «Шивананда-йога». Показала мне.

\M
Да надо было взять у неё эту книгу!

\I
А потом, когда мы делаем «Приветствие солнцу» в грязёвке, я уже, ну, перестал там умничать, да,
чтоб преподавателя лишний раз не злить.
Она говорит: ``Делайте так.''
Мне как бы охота спросить: а почему так?
А так вот, я читал где-то что-то, что...

\M
Другая нога?

\I
Да. ...что так нельзя. Ну, я уже так, не стал как-бы нарываться. Думаю: ладно,
делаю как она говорит. Потом сам полез в интернет. Думаю: блин, мы когда делаем приветствие
солнцу, мы же там все распарены, у нас там кровь гуляет, там...\
пульс такой бешеный. И она говорит: ложитесь на пол теперь все и задирайте ноги кверху.
Я думаю: йё-кэ-лэ-мэ-нэ --- это
же может давление в башку ударить. Ну как же так --- так же нельзя!
Полезть искать как это правильно всё делать. Значит, потом вспомнил:
А! Мне же эта Анна книгу показывала.
Ну-ка попробую я у неё
книгу взять --- может там это расписано.
Вот.
Пошёл, а там её не было. Ну, я думаю, это...

\M
К себе да?

\I
Да, к себе.
...думаю ладно, думаю, фиг с ним.
А потом как-то получилось: у меня в этом костюме короткие штаны.
Ну, я получил...\ как япончик такой там, мы ходим, до середины эти штаны, ну рост-то большой.
Так-то костюм нормальный, а штаны короткие. И я как этот...\ кукла японская, или как сказать.
Широкие эти штаны такие до...\ чуть ниже колен, такой весь в этой...\ в рубашке.
Ну, вот, роба эта, или как там она называется.
Вот.
Как-то это неказисто выглядело.
Мне пыталась намекнуть, что вот, мол, кольчужка-то коротковата.
Ну, я как-то внимание не обратил, потом чё-то меня как-то заело.
Думаю: ладно, надо что-то с этим делать.
Пошёл в ателье.
Говорю это, а... А! И вот как получилось.
Я говорю: а у вас это, ну что-то мы разговорились,
я думаю, спрошу у неё где она-то штаны...\ это...\ костюм?
``У вас штаны где, это...?''
``Я --- говорит --- мы на обучении когда в Польше были, нам типа это, давали, типа.''
Ну, она на курсы ездила международные.
Я говорю: ``А у вас выкройки нет?''
``Нет.''
А потом что-то я вспомнил, что я хотел у Анны эту книгу взять.
И что-то мы так это с ней.
Ну, особо мы не общаемся, я так просто пришёл, ушёл.
А тут что-то про эти штаны разговорились --- я так это, расслабился --- думаю, спрошу у неё,
думаю. ``А книжку какую-нибудь у вас можно взять почитать?''
Она такая: ``Есть у меня книжка. Ладно, я --- говорит --- дам вам.''

\M
Сначала на себя или от себя?

\I
А! Сейчас... Значит, от себя, так, к себе.
Теперь попеременно руками перед собой.
То есть, мы чередуем руки, да.
``Ну есть --- говорит --- книжка.''
Принесла потом на следующее занятие.
А я как раз на больничный пошёл весной. И пока лежал-болел, я эту
книжку\footnote*{Свами Вишнудэвананда, «Йога»} всю прочитал. А она такая приличная книжка, и шрифт мелкий.
Там дофига. Я её всю прочитал.
Она вот как-раз 9-го вышла из отпуска. Я к ней пришёл.
Говорю, я эту всю книгу прочитал. То-то то-то мне...

\M
`Разговаривайте теперь со мной уважительно.'

\I
Да. Я там уже на высших материях с ней... Там философского много.
Мне, говорю, то-то понравилось, то-то понравилось, список литературы вот выписал.
Она говорит: ``Ну вот у меня эта, эта, эта книжка есть, а если ты найдешь, там, вот эти-эти,
то ты мне скажи. Я могу, говорит, тебе вот эти дать почитать.''
Я говорю: ``Я сейчас в отпуске, сейчас пока не надо. Потом.''
Короче, говорю, я от вас ухожу, говорю. Я нашёл другую йогу.

\M
У нас этих йогинь здесь...\ как...

\I
Я говорю: ну вот, пока вы болели, ой, в отпуске были, я говорю, переметнулся, нашёл...
Она говорит: ``К кому, к Юле?''
Я говорю: ``Да, к Юле.''
Не стал спрашивать откуда...
Ну, явно они все друг друга знают.

\M
Да конечно.

\I
``Я говорю у неё, там, говорю, дыхания такого нет как у вас, говорю, мне конечно этого не
хватает, дыхательных этих.
А, говорю, зато у неё там много асан и больше
силовую нагрузку идёт. Я говорю: до отпуска уже не буду дёргаться.
К ней раз начал ходить, похожу до отпуска. А там, говорю, наверное скорее всего
буду чередовать.''
Просто у них в один день и в одно время идёт.
``Скорее всего будут день к вам, день к ней ходить.''

\M
А народу много сейчас?

\I
Ууу... Ну много. Обычно около 10 человек.
Я когда был один раз было 16.
Ну, бывает...\ нас было четверо один раз.
Вот. И я говорю это...\ я говорю: вас не хватает, мне очень нравилось тем-то тем-то.
А в «Восходе» мне вот нравится тем-то, тем-то. Поэтому я говорю когда
с отпуска вернусь,
скорее всего буду и к вам, и к ней ходить. Вот.

\M
Ну, в общем...\ ну, это хорошо, что что-то ищешь, но мне кажется что всё-таки на одном
на чём-то надо остановиться, потому что...

\I
Да, и...\ я знаю, но просто, что мне это действительно зашло.
Зашло, как говорится.
Есть такое как-бы сленговое...

\M
Да, да, да.
Вот так по-моему надо делать.

\I
Так, руки...\ так. Сейчас. Попеременно вперёд, так...
Руки наперекрест вправо, наперекрест влево. Это нет, вот так.
То есть вот, внешне... А!

\M
Вот так? А! Вот так?

\I
Сейчас.
То есть...
А! Вот так. Во! Вот
так, да. То есть вот у нас как-бы...\ которая идёт вверх --- она наружу, да.
И ладони они как бы вот так, в разные стороны расширяются.
Ну, а я... Она объяснять мне начала, я её перебил. Сейчас вот жалею, дурак.
Думаю: что я полез? Она там своё начала высказывать, а я... Себя корю.
Думаю блин, что ты всё время лезешь, слушай человека. Она начала мне там объяснять одну
интересную вещь. Когда вот я к ней заходил-то --- книжку принёс в понедельник.
Как-бы мы чуть-чуть посидели у неё в кабинете. Она-ж там работает. После работы
ведёт йогу. Она там врач-реабилитолог. Ну, я говорю, рано пришёл, ещё до занятий
минут пятьдесят было. Зашёл. Ну, мы посидели, поговорили. Она мне там начала интересную штуку
рассказывать, а я дурак перебил её --- со своим полез. Потом думаю: блин, зачем ты полез?
Надо было сидеть слушать.

\M
Ну еще раз попробуй, второй заход сделай:
``Не отпускает меня, вот помните что вы говорили...''

\I
Ну, я хочу потом переспросить. Ну, я как со своим,
как говорится, со своими дурацкими
репликами полез.
Надо в следующий раз...

\M
С другой стороны, да?

\I
...сидеть, да, сидеть молчать лучше. Слушать что тебе говорят. Так,
получается...\ а, вот! Да, вот.
Ну, я что-то ей стал объяснять, что вот я типа просто так хожу, чисто для развлекухи.

\M
Зачем человека обижать? Она там распинается.
``А я тут смотрю, как вы это...\ выступаете.''
С другой стороны, ты же деньги платишь.

\I
Да.

\M
А, ну и всё. Тогда ладно.

\I
Ну, меня это цапануло тем, что она... У неё
энергетика такая. Я когда первый раз пришёл на занятия, меня как-то она притянула к себе сразу.
Я вот можно сказать на 50\% как-бы из-за этого стал ходить.

\M
Ну то есть ощущения какие-то запомнились.

\I
Да. То есть у неё...\ ну, человек такой интересный. У неё такая сила внутри чувствуется.

\M
Что там ещё?

\I
Так. Теперь мы сделали вправо-влево, теперь в противоход «по столу».
То есть, ладони перекрещиваются вверху,
а потом --- ладонями вверх когда перекрещиваются --- потом они вниз уходят.

\M
Вот мне что в этой системе нравится --- что здесь работа идёт на скручивание суставов.
Всех.

\I
Ну, у нас сегодня пока внутренней работы нет. У меня, по крайней мере. Внешняя только идёт.

\M
Ну, ты уже устал просто.

\I
Я болтологией занимаюсь.

\M
Нормально --- если ты с полседьмого здесь.

\I
Мы со Славой так же пока не ввели закон
тишины... --- так, пол занятия болтаешь.

\M
Нет, тогда теряешь время.
А вы друг другу-то деньги не платите?

\I
Нет. А он сколько раз, это, говорит... Допустим я уже
собираюсь...\ я уже всё, это, собрался, лыжи намазал, выходить уже --- смс-ка:
ой, чё-то я сегодня это, не хочу заниматься.

\M
То есть, он...\ у него ключ? От этого...\ да?

\I
Да. И вот он...\ как-бы иногда хорошо бывает --- я иногда встаю, думаю:
блин, хоть бы сегодня Слава отменил. Не отменяет --- ладно, пойду.
А иногда прям так совпадёт: что-то вот уставший такой, ну, не хочу.
Раз, Слава пишет: сегодня отменим. Он утром прям скидывает бывает, вообще, за час.
То отменит,
то не отменит.

\M
А что, с вечера неизвестно?

\I
Нет. Он буквально вот допустим в пол-пятого приходит смс: сегодня что-то
я себя плохо чувствую.
То есть он как-бы абсолютно свободный.
Свободный, да.
То есть, он не хочет --- не идёт, хочет --- идёт.
То есть он под настроение.

\M
Нет, регулярность важна.

\I
Мы регулярно. Мы очень много с ним суммарно...\
мы очень много занимались. Прям вообще реально.
Если просуммировать все дни, то дофига раз мы занимались.

\M
Судя по тому, как ты говоришь...

\I
Да, и вот эти как раз спонтанные...
Бывает он на полторы недели вообще пропадёт.
Ну, то есть там...
Говорит, я например скину когда в следующий раз заниматься.
Полторы недели тишина.
Потом скидывает: завтра в 5.
Ну то есть иногда вот эти вот прям...

\M
В обратную сторону надо? Или мы уже сделали?

\I
Сделали по-моему. Иногда вот эти промежутки,
они прям мне кстати приходятся, потому что у меня там какой-нибудь аврал.
Я прям...\ ну, они так получается классно у меня
иногда совпадают. Вот. Так, значит...\ да. Другую сторону. Теперь шоу-гун делаем, да.
Вот. И поэтому он как-бы просто напарник мой для компании. Вот. И я ему тоже для компании.

\M
Нет, это важно.

\I
Да. А перед Новым годом, я помню, мы вообще каждый день полторы недели занимались.
Каждый день! Я потом уже чувствую, что-то это...\ я уже реально...

\M
А еще я тут это...\ с этой... И вы ещё и там занимались? И ты на улицу?...

\I
Не-не, перед Новым годом. А! Это мы после Нового года встречались уже там, за Дворцом пионеров.
А это мы перед Новым годом как раз полторы недели, у нас такой марафонский забег был.
Я прям потом чувствовал, у меня каждая клеточка, прям тело своё чувствовал.
Реально так нагрузился. Я прям запомнил этот период.
Что-то прям так это...\ вымотался. Ну, а так вот, он себя как бы не стесняет, скажем так.

\M
Не напрягает.

\I
Да. Правое бедро в сторону от себя, потом левое бедро от себя, потом к себе.
Ну, Слава молодец в плане того, что много знает, во-первых, и пытается заниматься тоже.
Интересуется, продолжает.

\M
Девять, да?

\I
Да. Я вот по этому, по книжке прочитал программы занятия йогой, там для новичков,
продвинутых, которую книжку читаю. Там в конце таблицы есть.
Там есть значит, уже когда доходишь до какого-то уровня, раз в неделю голодать.
Потом там есть 15 минут в день там, это, слово ОМ повторять.
Потом там, это, смотреть там...

\M
Теперь к себе?

\I
Да. И я такой Славе говорю, там это слово ОМ.
Он говорит: а я это знаю, у меня эти есть специальные тексты, мантры.
Я говорю: принеси мне.
Он мне принёс. У него там они расписаны.
И там их просто поёшь, и там написано, эту мантру повторять 72 раза подряд каждый день.
Это надо только этим и заниматься. Они интересные звучания такие.
Они действительно... Я не знаю, как это всё медицински обосновывается, но для меня это просто...

\M
Вибрации.

\I
Да. Для меня это просто как некое весёлое занятие. То есть, я дома же один.

\M
Соседи ничего не думают?

\I
А я снимаю квартиру --- мне пофиг.
Я завтра-послезавтра отсюда съеду. Пусть думают что хотят.
Я развлекаюсь.

\M
Теперь по-моему переминаться надо, по логике.

\I
Тут что у нас...\ так, приподнимание стопы рукой, да.
Ну, я к тому, что у Славы богатые это..., как это сказать, знания, и он мне книжку поначалу
ещё давал «Пранаяма». Ну, брошюрка ещё в советских времён издания. Там цена, что-то рубль, там,
сколько-то. Буквы все такие пляшущие, кривые.

\M
Неизвестно на чём набирают.

\I
Да, да, да. Такая книжка корявенькая.
Афанасьев автор. Ну так нормально написана. Ещё советская.
Потом, кстати, он мне давал книжку по точкам, у него есть.
Я потом похожую книжку в
библиотеке Маяковского брал. Точки эти все пытался...\ но потом что-то...

\M
Ну это к специалистам надо идти.

\I
Ну. Я что-то пытался взять эти...\ суджок-терапия.

\M
Да, у меня знакомая одна училась и она применяет. Я всё никак не куплю эти, как они называются?
Есть у тебя? Сигаретки эти.

\I
А у меня есть. Я целую пачку выписал, я прижигал ими.

\M
Ну и чего, получается?

\I
Ну я вот этот...\ дзу-сань-ли этот прижигал.

\M
Я зашла посмотреть, их там столько мне вывалилось, что я не смогла выбрать.

\I
Я вам принесу, у меня целый мешок.

\M
Да ты их применяй.

\I
Я их выписал, я их не знаю куда девать.

\M
Применяй.

\I
Я по первости применял. У меня их там --- я говорю --- целый мешок.
Там я не знаю...

\M
Если у тебя есть какие-то проблемы, то ты их просто вычисли да и всё.
Куда... То есть если у тебя есть какая-то постоянная проблема...

\I
Да проблемы нет. Я это просто для прикола.
Я выписал, что эта типа...\ дзу-сань-ли --- это точка ста болезней,
самая популярная точка у жителей Азии. Они её любят прогревать, прижигать, растирать.

\M
Когда там что-то заболит, знаешь там что хочешь готов уже прижечь.

\I
У меня даже шрамы
по-моему до сих пор.

\M
Ты чё так прижёг, что?...

\I
Я сжёг с дури, ну, сдуру, вот.

\M
Ничего себе.

\I
По незнанке сначала давай туда жечь. У меня волдырь как вскочил.

\M
Блин. Это когда ты всё это сделал?

\I
Это я уже не помню, в феврале\footnote*{Это было ещё на Ленина 11, то есть до Нового года.} по-моему.

\M
А! Это вот за этот короткий период времени?
Господи.

\I
Ну вот, как со Славой познакомился.
Он мне говорит: у меня такая книжка есть. Я говорю: давай. У меня такая --- давай.
Все его книжки...

\M
Читать --- это одно, применять...
Я думаю... Вот это называется «каша в голове». Надо идти...\ выбрать
дорогу и по ней идти. Вот это всё, оно мешает.

\I
Ну. Да понятно.

\M
Это тут у нас есть
одна посетительница, ну, ходит ко мне в группу, а у неё подруга ---
она ездила в Китай и научилась по...\
я говорила, да?\ --- по этим там, по пульсу определять.

\I
А! По пульсу --- да.

\M
По пульсу, да.
Она училась во-первых у настоящего китайского специалиста, во-вторых там практиковали они.

\I
Так: от себя, к себе.
И что? Она тоже всё подряд?

\M
Да нет. А я к ней ходила. У меня
были проблемы тогда со здоровьем. Уже прям вот, плохо.
И она говорит: ``Вот это вот...'' Я говорю: ``Ну, да, да, я читала что это такое.''
Она говорит: ``Вот такие методы борьбы.''
Я каждый раз проявляю осведомлённость, вот о таком методе...
Она говорит: ``А зачем вы ко мне ходите?''
Я говорю ``В смысле?''
Она говорит: ``Вы же всё знаете.''
И рассказала мне такой анекдот, историю, что какой-то парень сильно хотел похудеть.
И диеты разные.
Много диет сейчас всяких есть.
И он каждую пытался пробовать. Пробовал, поделает пару недель.
Ничего не получается.
Другую. Опять ничего не получается.
Старайся чтобы ты не наклонялся.
Лучше тогда сделать амплитуду меньше.

\I
Включить ягодицы.

\M
Чтобы ты за счёт вот
этого...
Чтобы ты в вертикальной...
К себе.
Можно делать короче амплитуду пока.
Потом ты уже поймёшь что надо,
какие мышцы.
Короче, она говорит: ``А потом --- {\csc раз} и он вдруг похудел.''
И все спрашивают: ``А что, что ты сделал-то, как тебе удалось?''
А он говорит: ``Я просто перестал метаться.'' Одно что-то
выбрал, и всё. И оно сработало. И оно любое сработает. Понимаешь?

\I
Да.

\M
Не надо кидаться на все стороны.

\I
Совершенно согласен. Абсолютно согласен.

\M
Если куча времени и сил,
то можно кидаться.

\I
Я и говорю, я вот эту книжку взял из библиотеки, суджок-терапия, давай её читать
тоже. Параллельно эти две книги. Думаю, блин, у меня же
сейчас такая будет каша в голове. Я её пошёл обратно в библиотеку сдал.

\M
Правильно.
И такие книги дома не стоит иметь. Надо... Нет,
я пока вот ребёнок был маленький, а что делать?\ --- болеет.
И если...\ что-то же надо делать.
Эти все антибиотики, всё просто...

\I
Так, теперь, это сделали. А теперь гунбу. Круги
руками от себя, левая нога вперёд. А! Вот, вот так.

\M
Ага.

\I
Ага. Потом правая нога... Ну, ногу меняем, да.
А потом то же на каждую ногу к себе.

\M
Ты соблюдаешь вот это, что когда ты идёшь назад, у тебя задняя нога тянется, включается.
Когда вперёд --- передняя.
Задней не толкаем.

\I
А я почему-то думал, что энергию из земли мы передаем толканием.

\M
Ты втягиваешь энергию из земли.
Ну, видишь, версий много.
Но вот это вот я точно запомнила, потому что я на каком-то семинаре была, и
вот это нам сильно втолковывали прям.

\I
Ага. Ладно. Буду акцентировать.

\M
А, сейчас ещё...\ тоже на одном из семинаров он нас там пытался учить.
Ну, по-моему никто не научился, но я одно знаю, что нет. Я просто сильно удивилась,
что такое может быть. Короче... Он ещё и показывал. Можно было к нему
подойти, и он это...\ показывал. Короче он...\ как это там было сделано...
То есть, противник стоит вот тут вот, а ты мыслью уходишь вон туда и разворачиваешь
типа оглоблю какую-то, повод мысленный такой, и его сшибает туда. То есть, ты точку опоры,
откуда ты его бьёшь, выносишь вон аж туда. Туда мысленно выносишь. В неё
выгрызаешься, чё там, опираешься на неё и его
сдвигаешь. И причём это реально. Он это всё демонстрировал.
Ты подходишь и он тебя тянет, он к тебе почти не прикасается.
Он пытался нас научить, но бесполезно.

\I
Ну, это да... Который я ролик-то отправлял. Они его за пенёк, руку ему вот так держат.
Он не может оторвать, а он за рукой тянется.

\M
Ну это как бы самое...\ да. Это достаточно такое упрощённое.
Он посложнее нам давал. Я вообще не могла понять
что он хочет. Он объясняет всё подробно. А ты не можешь въехать, потому что ты...
У тебя нету знаний. И центр тяжести всё время
между ног, да, это ты помнишь. Сюда не наваливайся
и сюда не наваливайся. Всё время с середины...
То есть, ты как-бы ноги можешь вот так {\csc раз} оторвать, и ничего не изменится.
То есть положение твоего тела не меняется.
Ты центрирован.
Ты что!
Я ещё ходила, ездила на семинары по этим самым...
Ну, у меня тут родители разболелись оба.
Надо было что-то делать.

\I
Теперь меняем ногу.

\M
А там вообще. Там какие-то были
космические знания, которые я тоже не могла понять. Приехали куча массажистов
к этому дяденьке. «Биологическое центрирование», вот. И я поняла, что я после этого к массажистам
вообще не хожу. Я поняла что это за люди. Глядя на них там.
Они вот очень такие...\ ну может
быть есть какие-нибудь приличные. Они беззастенчиво экспериментируют на тебе, и они могут тебе
просто сильно навредить.

\I
Я одного встречал такого. Фамилия Савяк. Красноярский. Слышали?

\M
Слышала фамилию такую.

\I
Ой! Такой прохиндей.

\M
Да? А все говорят к нему в очередь записываются.

\I
Я к нему ходил один раз. И я посмотрел кто это такой. Я оттуда
убежал, как...

\M
Не, ну...\ что он там такое делал?

\I
Во-первых, он циничный очень.
Во-вторых, он играет
на психике очень сильно. А реально лечить --- он не лечит.
Он делает облегчение и человеку делает внушение,
что он ему помог. Может он просто...

\M
Ну может это кому-то и помогает.

\I
Помогает --- тем людям, которые не хотят сами...

\M
Сами ничего делать.

\I
Да. Они думают, что придут к дяде, он им сделает.

\M
И через полгода снова придут.

\I
Да! Да, да. То есть, он им симптомы облегчает.

\M
Люди думают: я вышла --- выпрямилась.

\I
Да, да, да. Вот.
А базы-то нету, фундамента. Человек должен сам работать.

\M
Опять меняем ноги, меняем направление?

\I
Ноги меняем, а направление оставляем. То есть к себе. Вот.
И он просто пользуется тем, что человек идёт, деньги платит, ему всё устраивает,
и он на этом процветает. Симптомы облегчает.

\M
То, что нам преподавали оно
было конечно всё круто.
Очень круто. Но я не всё опять же могла понять, потому что у меня не было никакого понятия
о внутреннем строении человека.
То есть очень примитивные какие-то вещи.
Сейчас бы я гораздо больше взяла оттуда. И мы там практиковали друг на друге.
Ну, то есть, на кушетку ложится один и другой начинает делать. В самом деле идут
какие-то эти...
Как будто тебя зацепили вот изнутри.
Вот натурально.
Я человек не впечатлительный.
Но я вот это прям ощущала. А потом когда ты делаешь, и ты тоже...
Как будто ты за что-то зацепился там.
А ты не знаешь за что, и страшно так.

\I
За что-то в костях?

\M
Нет, там как-то рукой, знаешь, надо рукой водить, и ты понимаешь, там
ты ощущаешь напряжение в теле.
То есть, они тебя вводят в такое состояние. Кстати, вот это сильно помогало,
вот это вот.
Я как раз тут ходила занималась, и я поняла,
что принципы очень схожи.
То есть ты должен...

\I
Надо попробовать.

\M
Не надо. Ты вначале в одном преуспей, а потом расширяйся. Но я тебе хочу сказать,
что если ты влезешь вот в эту тайцзицюань тему --- там
бесконечно, там бесконечно, там можно до самой смерти развиваться.

\I
Хотя бы вообще попробовать. Я просто ни разу никого не массажировал.

\M
А там не надо даже прикасаться. Ты
там вот просто идёшь и понимаешь, что в
этом месте затык какой-то, ну... То есть... Но ты должен быть сам расслаблен, вот.
Что очень трудно.

\I
Так. Теперь в гунбу, значит, сделали. Так теперь в гунбу...\ сделали. Теперь фронтальные в мабу
делаем. Влево.
Носки чуть-чуть наружу, мы по-моему так делали.
И вот эти вот большие круги.

\M
И до пола или?... Не, не наклоняясь, да?

\I
Наклоняясь, вот, да. То есть получается...

\M
А там руками так вот надо делать? Вот так. Нет?

\I
Нет, вот он просто вот так. Большие такие
круги он делает. В мабу.
Да, то есть...\ так, вот правильно, да.
В одну сторону и в другую сторону.
Почти дошли до конца.
А, и у него когда смотришь вот на него, у него видно прям выстраивается диагональ, то
есть вот от ноги и пошла вот.

\M
Да. Вот то
есть, когда нога вот в этом крайнем положении, она вот здесь, да.

\I
Да, и вот там прям видно прям четко диагональ вот.

\M
Да. Он столько этих кругов накрутил.

\I
Теперь в гунбу переходим, и которая нога вперёд, на неё выкидываем и крутим.

\M
Ну да, тут шпаргалка нужна на первое время.

\I
Да.

\M
Вот надо же, отдала Анне Вадимовне диск. Говорю: давайте смотрите и будете нам это...\
рассказывать.
А она так вернула его через какое-то время
и говорит: ну, что там.
Я говорю: в смысле, я хотела тренировку построить на его...\ ну...
Система какая-то! У него явно она есть.

\I
Да, он вначале говорит: по системе мастера...
Он прям в начале говорит по чьей\footnote*{Фэн Чжицян} он системе
занимается.

\M
Ну понятно, да. То есть, он этой системы придерживается, а тут куда кривая выведет.
Это не правильно ведь.
А она говорит: ``Ну как-то вот, скучно.''

\I
Ну. А я на семинар когда съездил и я решил, что... На меня, да, ноги?

\M
Я думала --- в другую сторону.

\I
Нет. Пока вот. На семинар когда съездил и мы когда поделали, вот это всё
регулярно. Даже хотя бы тех же три дня, да.
Я уже как бы на это дело подсел. То есть мне это уже на третий день казалось
прям нечто само собой разумеющееся. Начинаем, вот это, крутим...

\M
Да, да, ты втягиваешься.

\I
Да.

\M
И потом, когда уехал, и думаешь: вот сейчас я буду каждый день...

\I
Да, да, да.

\M
И нифига.

\I
Потом, да. Я вот решил, вот с вами вот это...

\M
Освежить.

\I
Возобновить.

\M
Освежить, да.
На самом деле, это... Я поэтому диск-то и
купила, но я его первое
время там что-то читала, смотрела, а потом что-то у меня дисковода не стало.
Поменялся компьютер, дисковода нет.
Вставить некуда.

\I
По-моему последний раз, да?
Так, дальше у нас идёт...

\M
Шоу-гун должен быть уже.

\I
Сейчас, сейчас так...
обе сделали, шоу-гун и всё, и конец.
Шоу-гун и заканчиваем.
Я вспоминаю, как я этот 18 форм изучал.
Это мы его со Славой месяца наверное 4 делали.
А я потом захотел один сделать ---
не смог.

\M
Так же как я 24 формы не
могла сделать.

\I
Ну 24 формы, я уже к концу...\ мы начали в начале октября, я уже к концу ноября
уже все как бы знал, полностью весь комплекс. Ну, внешне хотя бы.
А почему-то цигун 18 форм, я его целенаправленно
не изучал. Просто со Славой делали, я за ним
повторял. Потом...\ ну, долго делали, я --- уже месяца четыре --- я захотел дома сам сделать,
и не смог. А потом уже, когда стал это...\ целенаправленно
запоминать, тогда уже выучил.

\M
Этих комплексов просто --- миллион. Мне говорят:
``А давайте вот это, а там каких-то 8 кусков парчи какой-то.''
Я говорю: ``Слушайте, ну, вы пожалуйста изучайте, приходите, будем...
Я даже рада, что вы мне тут показываете.''
``Нет, давайте вы изучите.'' Я говорю: ``Я не хочу.''
Отлично.
Спасибо тебе за это, за встречу.
Ещё раз тебя с днём рождения.

\I
Спасибо.

\M
Пусть у тебя...\ как это, как...\ этап в жизни...\ какие там --- по 12 лет считаются, да?

\I
Не знаю.

\M
По 12, там кто-то считает по 7, кто-то по 12.

\I
Ну я считаю, что новый этап начался. Я нашёл для себя интересную тему,
где можно не думать и при этом развиваться.

\M
А в смысле «не думать и развиваться»?

\I
Ну, то есть, в йоге --- там надо мозг отключить, но при этом развитие идёт.

\M
А развитие идёт по-любому.
Вот смотри, ты даже запоминаешь какое-то движение, включаешь какие-то мышцы,
уже там, какие-то ответы идут туда-сюда.
Связи, как это сейчас модно говорить.
Связи...
Связь... Нейро там какие-то связи новые в мозгу начинаются, и от этого идёт его развитие.
У нас-то у всех отодвинуть подальше деменцию и Альцгеймера, чтобы не докучать своим детям.
В полутрупном состоянии.

\I
Да, и ещё такой интересный момент: все хотят, вот почему это так, почему вот мы это делаем,
почему это делаем. А я что-то сейчас для себя решил, что не надо это даже себе пытаться объяснить.
Просто делать и всё.

\M
Да, начать делать.
Вот как этот, Балашов, как раз спрашивает на семинаре.
У нас было...\ занятия вот эти вот раз, два, три, а один день был как типа...\
задаём вопросы, а он отвечает.
И вот там все задают вопросы.
Я конечно там ничего не задавала, потому что у меня собственно и не было вопросов.
И вообще хоть бы понять, что тебе сказали. Нету вопросов.
Ну все там задают, они уж не первый раз.
И один говорит: ``А вот каждый день надо заниматься?
Каждый день?'' Он говорит: ``Да, каждый день.
Надо просто каждый день вставать и делать.''
``Ну --- говорит --- это...\
как --- говорит --- делал я.
Вот ты просыпаешься: а может не делать?
Вот сегодня как-то вот что-то идти куда-то надо, там ещё...
Поспать бы.
Нет, а вот так встаёшь просто и «на отвяжись».
Не надо делать правильно.
Ты просто встаёшь и начинаешь размахивать руками.
И ты постепенно входишь вот в этот ритм,
и начинаешь делать правильно.
И...\ ну как...\ «на отвяжись».
Потому что...\ ну вот, постою вот так просто поболтаю.
Ну что, раз уж встал, так что-то надо...''
И я вот подумала, что в этом он прав, конечно, сильно прав.
Но тот крутой сильно.
Я ни разу с ним непосредственно не занималась, потому что он
с теми, кто получше.
Но один раз мне дама там одна рассказывала.
В общем, тот тренер который нас тренирует, по каким-то причинам не смог и он вёл семинары.
И говорит: ``Это вообще просто несравнимое совершенно ощущение.''
Они его попросили, чтобы он показал вот эти все бесконтактные какие-то воздействия.
Говорит: ``Мальчики летали просто по краям площадки.
Девочек он щадил, девочек: вот, как будто тебя просто вот так подошёл кто-то и толкнул.''
Я говорю: ``Господи, как так вообще бывает.''
Ну, это не...\ это как-бы трюки такие, которые завлекалки.
А на самом деле это очень серьёзная работа, которая может показаться очень нудной.
Но результата можно такого достигнуть.
Так что это...\ дерзай.
У тебя всё впереди.
Я кстати начала заниматься вот этой штукой --- да, мне было 40 лет, точно.
40 лет мне было. Я пришла в Олимпиец, вёл Юрий Николаевич.
И он значит...
И там толпа просто женщин.
В разных одеждах, в разных комплекциях.
И он впереди идёт.
Я себе представляю, какой он ощущал...\ как это...\ повелитель такой.
Потому что все ему в рот смотрят.
А ему нельзя такое --- крышу сносит.
И все эти тётеньки стараются вот так, идут за ним, назад там...
И он так...\ каких-нибудь покраше выберет, там трёх-четырёх, на них обращает внимание.
А остальные там где-то в углу... ``А посмотрите как я делаю!''
И, значит, мы всё делали, делали. Я год проходила, я вообще ничего не понимала, что там происходит.
Я тогда заболела сильно, и мне что-то надо было делать.
Мне говорили походить.
И я пошла.
И я натурально просто три раза в неделю ходила, или два.
И ходила, просто, вот честно ходила.
Ничего не понимала вообще.
Заходишь в эту...\ в раздевалку.
А эти дамы, значит, которые с ним или начинали или что-то такое:
``Ой, я прямо вот чувствую у меня энергия прям побежала.''
Господи, где я вообще?
Что это такое?
И они по-натуральному серьёзно вот так вот разговаривают.
К концу учебного года, то есть вот в мае, осталось наверное человек пять от этой вот
большой группы.
То есть все тихо-тихо поотсеялись.
А в конце они делали 24 формы.
Я вот так вот посмотрела.
Во-первых, у меня не возникло ни малейшего желания это повторить.
Вообще, вот просто нет.
Второе, я ничего не поняла.
Я не могла понять даже последовательность движения, что они там делают.
Мне казалось, каждый раз они что-то разное делают.
Но они все делали по-разному.
Сейчас я представляю, мы делаем лучше гораздо сейчас, чем они тогда.
И они как бы соревновались между собой, кто лучше делает.
Понимаешь, вот это {\font\F=omu12 \F правильно}. Они все делали неправильно, это я сейчас понимаю.
А ему было вообще пофиг, он там своё делал, у него растяяяяжка такая.
Потом выяснилось, что он многие формы давал неправильно, потому что я привезла на семинар
одну форму.
Мне говорят: ``Ну покажите, там, что вы делаете.'' Я делаю.
``А что вы делаете, я не пойму.'' Я говорю: ``Ну вот, там...\ облака --- как это называется
--- облачные движения.''
``Это всё что угодно, только не облачные движения.''
И учила я весь семинар, я учила облачные движения.
В конце он так посмотрел: ну ладно, пойдёт, пойдёт.
И приедете в свою группу, покажете, как их надо делать.
Из всех 24 форм, понимаешь, он выбрал одну только: облачные движения, которую я считала
я делаю прекрасно.
То есть, мне казалось, что я всё делаю нормально.
Приезжаю сюда и этому говорю, что вот...
Он говорит: ``Вы никому не говорите что...\ вы мне покажите.'' Вот он как...\
что он у меня снимет это всё, что там все косяки, которые он допустил.
И он потом будет от себя давать на группу.
Ну понимаешь?
Вот. И я тогда начала сомневаться.
Ну, что-то какой-то тухлый этот товарищ в общем-то.
Но он за счёт своей растяжки, бесконечного занятия каждый день, у него всё красиво получалось.
Возможно в каких-то местах наполнено.
Ну, он сильно отличался от нас в лучшую сторону.
Так что...\ я думаю, у меня скорость набора информации была значительно меньше чем у тебя.
Я думаю, что ты к моим годам будешь гораздо продвинутее.
Так что дерзай.
Но не торопись.
Вот торопиться здесь точно нельзя.
Оно может это...\ крышу снести.
Я тоже тогда думаю: а почему детям?
Почему детям не даёте вот такое?
Им, говорит, нельзя.
Почему нельзя?
Я до сих пор не понимаю.

\I
А знаете, вот...

\M
У них в общем-то какие-то знания есть. То есть они ещё гармонично достаточно двигаются.

\I
Почему я... Я вот...\
такая же история как я на йогу ходил.
Я тоже ходил, просто ходил.
Для галочки.
Пришёл, поделал. Что поделал? Сам не понял.
Но ходил.
И вот эти притчи-то восточные, сказки, всё --- они же специально не объясняют.
Потому что эти знания могут попасть в руки неподготовленных.
Они зашифрованы.
Когда время придёт --- ты поймешь.

\M
Да-да-да, совершенно верно.
Так что, жди когда время придёт, подготовься заранее, но не лезь сильно в глубину.
А вот на самом деле: делаешь движение, делаешь, делаешь, и вот так каждый раз.
Ну, если есть кто смотрит: ``Ну что я, правильно, да? Вот здесь вот как, руку сюда или сюда?''
Ну то есть, ты всё время сверяешься с какой-то картинкой.
А потом так {\csc раз} и ты понимаешь, и тебе не надо спрашивать.
Ты просто сам знаешь, что ты уже делаешь как надо.
Потому что ты устойчивым становишься в этом положении.
Как только ты понял, что ты как такой тяжёлый шар, который сдвинуть можешь только ты изнутри.
И тогда просто легко можно понять, что ты всё делаешь правильно. Так что давай.
Ну, это интересная тема.
И я рада, что кто-то подцепился из молодых.

\I
Я рад, что вас встретил.

\M
Потому что... Кстати, это конечно здорово, что можно
поделиться. Давай, удачи. Я думаю, что ты будешь теперь собираться и всё такое.
Потом приедешь --- как-нибудь соберёмся.

\I
Обязательно встретимся!

\kern2cm
\it
\lineskip=7pt
\hskip6cm Утро 14 июня 2025 года \par
\hskip3.8cm Спортивная площадка возле хоккейного \par
\hskip7.3cm корта за «Восходом» \par
\bye
