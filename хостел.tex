%&12pt
\pdfpagewidth=297mm
\pdfpageheight=210mm
\pdfhorigin=1in
\pdfvorigin=0pt
\input quire
\shhtotal=\pdfpagewidth
\htotal=.5\shhtotal
\vtotal=\pdfpageheight
\shoutline=0pt
\shstaplewidth=0pt
\shcrop=0pt
\shfootline={}
\shthickness=0pt % we want to use staples, but do not use .27mm (and do not do corresponding manipulations with \horigin and \hsize) until you do TODO in ~/term/стихи.tex
                 % (this document consists only of 3 sheets, so protrusions are almot imperceptible)
\quire{4}

\horigin=9mm
\hoffset=-9mm
\hsize=\htotal \advance\hsize by-2\horigin \advance\hsize by\hoffset
\advance\horigin by-\hoffset
\output={\ifodd\pageno\hoffset=0pt\fi \plainoutput}

\vorigin=15mm
\vsize=\topskip \advance\vsize by35\baselineskip

\font\TENSL=OMSL10
\headline={\rlap{\vbox to0pt{\vss\hbox{\raise5pt\line{\TENSL
  \ifodd\pageno\hfil Разговор в хостеле\kern.5pt\else
  Возвращение с летнего интенсива по тайцзицюань\hfil\fi}}}}\line{\hrulefill}}
\footline={\raise3pt\line{\hss\tenrm\folio\hss}}

\font\speakerF=omssbx10
\def\A{\item{\speakerF А.}}
\def\I{\item{\speakerF И.}}
\setbox0=\hbox{\speakerF А.\enskip}
\parindent=\wd0

\hyphenation{тай-цзи-цю-ань}

\A
И потом начинаешь замечать, что мозг сам уже просит, просится.
То есть вот уже даже независимо от тебя уже сам эти все прокручивает форму уже даже во сне,
как эти все движения делаются.
И вот этот процесс именно обучения как раз для меня именно занимателен, интересен.
То есть вот как изучать что-то, новый предмет.
И чем больше ты смотришь на какую-то вещь, интересуешься, неважно чем,
тем больше она раскрывается и становится интереснее.
Вот это я заметил.
То есть вот именно сам принцип обучения, он как бы в этом, я понял, что он в этом состоит.
То есть бери любой предмет, смотри на него долго, и он начнет раскрываться.
Вот основной принцип.
Неважно что.
Чем подробнее, тем оно богаче становится.
Принцип созерцания и погружения.
Да.
Так, чтобы любая вещь, любая вещь может стать интересной.
Для этого надо просто на нее достаточно пристально и долго смотреть.
Даже вот есть притча о каком-то зернышке, что вот мудрец смотрит зернышко,
а в нем он видит всю картину мира в этом зернышке.
Как оно там созревает.
Как оно все это идет.
Как все эти процессы природы, они все там отражаются в этом всем.
То есть как бы вся вселенная, грубо говоря, заключена в маленьком зернышке.
Если на него сильно долго смотреть.
Это же ребята, которые давно на этом ездят, занимаются.
Да, они уже 15 лет проводят.
Андрей Владимирович Ширай есть.
Такой мастер боевых искусств.
Вот.
Он организует.
Он у нас был тренером.
Ну, очень много он нам дал информации.
Комплексы там всякие интересные.
То есть у нас было просто как закачка информации прям.
Набор такой информации прям.
Объемов таких.
Накачка информации.
То есть мы не разучивали что-то определенное несколько раз подряд.
А мы вот именно каждый день у нас что-то новое.
То есть обзорное было у нас.
Загрузка вот именно информационная, обзорная.
Мы там и багуа эти, разминку проходили по кругу.
Они там хороводы водят.
Багуа это.
У них стиль такой.
Они за спину заходят, атакуют со спины.
И у них такая разминка есть.
Они хороводом ходят в одну сторону, в другую сторону.
Вот у нас перед отъездом эта последняя разминка была как раз багуа стиль.
Интересно.
Ну там много он и всякие школы боевых искусств разных у них.
Какие у них разминки есть.
Мне понравилось там этот крокодил машет лапами.
Разминка.
Потом ласточка касается крыльями воды.
Тоже интересно.
Там такие движения.
Особенно мне понравилось наматывание шелковой нити в движении.
Вот.
То есть при ходьбе.
Я не совсем понял как это делается.
Я запутался.
Там просто сложная синхронизация движения рук и ног.
Но мне прям это.
Прям запало.
Надо более подробно это все дело ознакомиться.
Но у вас вроде есть так дети?
Нет.
Я один человек только этим занимаюсь.
Ну у меня есть.
Его просто не нашли пока.
Я как бы начал заниматься в 2024 году.
Первое знакомство мое с Тайцзи Цуань.
В январе.
2024 года по интернету просто увидел там мужчина ну китаец делает всякие такие движения там такие ну я полез в Тайцзи Цуань полез по ней поискал там ролики стал смотреть ну там потом больше больше потом просто в интернете набрал ну название города там я под Красноярском город маленький Зеленогорск да и я набрал Зеленогорск Тайцзи Цуань.
Ну там высветил сердце.
Уже закрыто было?
Закрыто и закрыто вон сейчас.
Ребята приезжали оттуда Зеленогорска каждый год типа аналог что где когда интеллектуал с вашего города приезжали молодые.
Так так и что дальше там приходилось?
Ну да и вот мне поисковик нашел что есть секция я звоню в эту секцию а мне говорят ну у нас уже это не преподается это сайт старый там информация тот человек который преподавал номер уже.
Ну это с 90-х годов когда у ШУ вот это расцвет был еще это было все конец 80-х 90-х вот там у нас был человек он вел а потом вот он умер как раз года три назад и как-то вообще интерес к УШУ как-то пропал вот и как бы ну а информация на сайте осталась.
Ну я когда звоню ну мне сказали там еще вот ходят ребята которые к этому ходили которые вот умер.
Занимаются сами просто?
Я ну телефон дали я позвонил ну и там один человек он как бы ну с ними как бы тоже занимался и остальные они пенсионеры уже все и они занимаются в будний день а я работаю.
Я никак с ними не попадаю один он посменно работает мы с ним договорились что перед работой будем с ним тренироваться просто как за компанией.
Угу.
Я говорю давай вместе ты будешь меня учить там перед работой как раз разминочка там все.
Ну и мы стали тренироваться вдвоем по утрам перед работой.
А потом я это где-то с октября ну он стал меня учить по моряку там он уже как бы ну кое-что там знал там все.
А потом я в интернете просто забил ну уж когда нам отпуск сказали что график отпусков делать.
Угу.
Ну надо вот какие-то типа курсы и куда-то съездить.
Конечно все надо обязательно.
Да совместить то есть и отдых и обучиться получше вот этой всей Тайзи Цуань.
Ну я набрал да в интернете Тайзи Цуань летние летняя школа там это ну и он мне вконтакте нашел я тут же позвонил записался и вот все удачно съездил.
То есть у вас получается все уроки все удачно?
Вообще удачно я съездил.
Я вижу что вы как вдохновленный.
Это сейчас еще у вас представляет кто-то переосмысление сейчас пока еще только еще как сказать не выключились процессы.
Да ну то есть грубо говоря у нас в городе есть там кто занимается но они уже все за 60 за 80 а один вот этот с которым я человеком занимаюсь ему 55.
А молодых вот чтоб 40 там ну может школьники там какие-то секции борьбы ходят там это а именно Тайзи Цуань и именно моего возраста вообще нет народа.
То есть у нас город маленький там все все.
Я очень могу сказать что я давно был в Красноярске.
Да.
В Красноярске это должно быть.
В Красноярске есть да.
И там ну народ не занимается.
Я фактически со своего возраста один человек только один занимаюсь.
Поэтому мне вот сейчас я приехал там с ребятами потренировались и я сейчас приеду а с кем я буду.
Мы же там у парня работу делали с палками ту шоу делали только сейчас понял.
То есть ты стоишь.
С человеком рядом.
Он пытается тебя спихнуть с места.
Вот просто толкает.
И там интересно так что человек как сделан интересно что если он правильно синхронизирует движение его не подкинешь.
То есть вот реально я по всякому пробовал человек опытный.
Бесполезно.
Все он тебя ставит и ты его никак не можешь с места строить.
И вот это как раз тут шоу и заключается.
То есть ты давишь на него.
Он твою силу маленечко принимает.
И уводит в сторону.
И ты своей силой мимо него.
И вот он стоит так играет.
И ты никак не можешь до него добраться.
Интересно так.
И вот я первый раз только прочувствовал что это как это.
Потому что партнера у меня не было до этого.
А тут мы в парах работали.
Конечно.
Практика когда это одно.
Когда даже самое.
Когда вот эти семинары.
Когда другие люди.
Новый опыт.
Да.
Поэтому я думаю где вот мне сейчас искать партнера.
Ну ищите Красноярск, Новосибирск.
Я думаю здесь семинары будут какие-то.
Ну да.
Семинары.
Прослежащие города.
Крупные.
Ну а у вас Красноярск крупнейший.
Да.
Ну и Новосибирск.
Потому что даже вот смотрите.
У нас ко мне приезжают очень часто те допустим ведущие
заинтересованные вот в данном направлении люди.
Они же приезжают.
Где делают обычно?
В центральной части.
Где ведь?
Москва, Питер.
А в Сибири они здесь едут.
Самый крупный город Новосибирск.
И он как бы.
И он в центре находится.
То есть приедешь в Иркутск.
Вы понимаете.
С Екатеринбурга не поедешь.
А к нам сюда.
И с Екатеринбурга поедешь.
И с Красноярска.
И с Иркутска.
Как бы разумное расстояние.
Да.
Поэтому нас вы точно найдете.
То есть то что у нас в РУШУ есть.
Это я вам точно говорю.
Тайцзи Цюань.
Я.
Боюсь вам сказать.
Я просто у меня зуброга как-то тоже ходила.
Я знаю что есть.
Занималась.
Я сейчас просто не готов сказать.
А да.
Я думаю что есть.
Это точно.
Есть.
В Новосибирске 5 секций есть.
А.
Ну вы уже знаете.
И я думаю что они какие-то мастер-классы.
Семинары.
Проводят.
Да.
И вам просто на них выйти.
Графики.
Вот.
Вот.
Вот.
Вот.
Вот.
Вот.
Вот.
Вот.
Вот.
Вот.
Вот.
Вот.
Вот.
Какие-то мероприятия.
Но да что под себя?
Вам кто еще.
Утром сел.
Вечером сел утром здесь.
Или.
Ну вечером сел.
Ну да.
Это понятно.
Но а кто-то был в Новосибирске хотя бы там несколько раз
неделю.
Но по мнению.
Нет.
Это.
Это.
Прикольно.
Не.
Это да.
Само собой.
Так найти единомышленников.
А в качестве развития.
То есть даже если у вас будет там 5 единомышленников.
Но развитие.
Это всегда должен быть мастер выше.
Да.
Замастера выше.
Да.
ежедневные практики, назовём так, а, чтобы у вас был рост, это нужно, опять же, это обязательно, это обязательно развитие, это так же, как спортсмен, ты можешь тренироваться для себя в подвале 100 лет, если соревнований нет, ты не знаешь свой уровень.
Да, я прав, если ты съездил на семинар, а всю эту информацию выкопали, ты не, да, ты будешь, она бесполезна, я говорю, если вам нужна единомышленники, а вы для развития нашли, ну, я думаю, вы найдёте, в ближайших городах здесь у нас будет точно, ну, не надо в своём городе единомышленников, найдёте, я не сомневаюсь, либо переехать в другой город, что будет, наверное, нелегко, ну, это уже другой вопрос, да.
Хотя, знаете...
Вы же понимаете, ставишь цель...
Да, это так.
Если у вас будет цель, и вопрос нелегко отпадёт, если цель есть, нет, ну, сложности, они, понимаете, они в любом случае будут.
Да, это правильно.
Но будут, а, сложности, либо они, как сказать, катастрофические, или сложности, которые, как этап, как ступенька.
Ну, сложность, ступенька, сложность, ну, ты напрягся, поднялся, ну, ты идёшь дальше.
Да, да, да.
Ты не стоишь на месте.
Это так.
Ну, всё, найдёте, не переживайте.
Ну, опять же...
Ну, опять же, что цель ставить, да, думаешь, а, ну, если цель поставить, её можно добиться, это да.
Но, с другой стороны, а нужна ли тебе эта цель?
Нет.
Такие мысли зарождаются.
Опять же.
А это постоянный процесс.
Это смысл жизни уже.
Это, это, это у вас, если вы ставите такие вопросы, то вы правильно делаете эти вопросы, вы будете ставить до конца жизни.
Плохо тот, кто не думает ни о смысле жизни, а нужна ли эта цель.
Бесценная жизнь, она хуже.
Лучше работать постоянно, я делаю правильную идею, нет, задавать вопросы.
Когда ты задаёшь вопросы, ты находишь ответ.
А когда ты не задаёшь вопросы, нужна ли эта цель, нет, это в лучшем случае вот так, мы, скорее всего, это деградация.
У меня вот сейчас просто я, как бы, на распутье в таком жизни, не знаю, где моё, скажем так...
А вы же знаете.
Вы задавайте вопросы, ответы придут.
Ты вопросы задаёшь, но, как, знаете, говорят мудрецы, всему своё время, и всё, то, что вы должны узнать, получить знания в 10-м классе, вы в 1-м никогда не получили.
Ну, как бы вам не хотелось в 1-м классе знания получить 10-классника, вы должны пройти ещё 9 лет.
Поэтому ваши, ответы на ваши вопросы придут именно к вам в то время, когда это будет.
Должно быть.
Одному это приходит в 20, одному к другому в 70, третьему в 40, кому-то вообще не дано в этой жизни, но не в следующем, только в реинкарнациях получит.
Да.
Поэтому вы же на этом пути.
Терпеливо, как вам?
Задали вопрос, смотрите, и терпеливо ждите, и вы дождётесь.
Но не то, что терпеливо сиди, это, знаете, как сиди, рыбку не поймаешь.
Возьми удочку.
И сиди, и лови, и всё, и как.
И вот, даже если так сейчас, вот, оглянуться назад, где я был полтора года назад, на нулевом уровне, и как я за полтора года уже, что со мной произошло, я туда съездил, того, с кем познакомился, то узнал, то, то есть, путь уже пройден, вы поняли?
Ну, и дальше будет так же.
Ну, то есть, я понимаю, что вам хочется, а я хочу сразу туда, но вы посмотрите на тех учителей, которые, они к этому пути ежедневными.
Занятием, упорством, трудовым, понимаете, и у вас нужно просто путь.
А распутье у меня заключается именно в том, где мне будет лучше, то есть, а может, вот я сейчас туда, а может, там ещё хуже, чем здесь будет.
Здесь вроде как, как говорят китайцы, народная мудрость китайская, не попробуешь, не узнаешь.
Поэтому, это нормально, что вы задаёте вопросы, пробуйте.
С другой стороны, попробовали, это как, знаете, пошёл, нет, не моя дорога.
Вернулся, пошёл в другую, моя, или опять, ну, опять не моя, значит, другая.
А у меня уже в жизни так, кстати, было.
Так и на раз.
Резкая перемена прям образа жизни, и оглядываясь назад, я понимаю, что...
Если бы не сделали...
Если бы не сделал, я бы вообще...
Ну, могу это только что говорить, я говорю, вот я с чего мы начали.
В начале я говорю, если вы задаёте вопросы это, вы...
Вода найдёт дорогу.
А если вы...
О, нет, наверное...
То есть, если вы сами себе задаёте вопросы, ищите ответы, если не будете, будете, говорю, в лучшем случае, а скорее, а сейчас вы же сами себе задаёте, да я уж так, как правило, дорогу туда, а то вам туда не надо, туда никому не надо.
Но для этого нужно, так сказать, искать, не бояться, пробовать, в том числе и ошибки.
Это не отдельная моя часть.
Это так же, как в упражнении.
Вы сделали...
Не не так, а так, получили опу.
Сделали так, вы защитились.
Понимаете?
Ошибка была, но ошибка это опа.
То есть, в ошибке правильно воспринимать как?
Окей, это так, я теперь знаю, как не делать.
Пока ты её не сделал, ты не понимаешь.
В внимательной ситуации, когда ты знаешь, как не делать, шансы победить есть.
Когда ты не ошибался, бесполезно.
Без проигрыша.
Да, и тут ещё такой важный фактор, уверенность в себе.
То есть, вот я понял со временем, что есть, как бы, условия, в которых человеку не подходят абсолютно.
То есть, одному они подходят, а другому, для другого они неестественны.
И многие люди говорят так, вот аж ты чего там захотел, сразу туда.
И как бы людей, как бы, задавливают, да.
А человек должен искать своё, своё.
И для этого нужна некая уверенность в себе, чтобы противостоять.
Противостоять этим людям, которые пытаются, вот, ишь, ты чего захотел, ты вот, значит, ещё вот это не прошёл, а уже вот туда захотел, и как бы они тебя придавливают.
И надо им как-то уметь противостоять.
А здесь, как сказать, здесь не то, что противостоять, я бы даже сказал, здесь просто двигаться своей дорогой.
Вам зачем тратить время на противоэнергию, на противостояние, это когда вы говорите, да так, если нет, ну идёте своей дорогой.
Ну вот он говорит, нет, не ходи туда, не пробуй, потому что тебе там будет плохо, а кто-то узнает.
Вы задайте вопрос, вот он предполагает, а другой, вот, допустим, будет нас два человека, вы скажете, вот, слушайте, переехать мне из Селеногорска, условно говоря, там, в Химки, под Москву, круто, только резко, резко, оттуда туда, даже не в Новосибирск.
Они скажут, ой, да ты что, в Москву, дорогой город, думаешь, там будет...
А другой скажет, да не знаю, попробуйте, перспектив больше, тем более, там есть школы эти, понимаете.
Откуда знает один, что там будет, никто не знает, а у вас своя дорога.
Если у вас есть цель, сказать, ну, здесь я как бы это, а там я могу попробовать.
Вернуться вы всегда можете.
Отчего вы, если это неправильно, вы...
Поэтому не знает не первый, который говорит, вперёд идти, не знает не второй.
Так.
Но если у вас, вы знаете, что надо что-то изменить, вам стоит лучше попробовать, чем...
Ну, опять же, это же всё не то, что там, засунь руку в вагоне, прыгни с двадцать пятого этажа.
Мы же говорим о таких, если у вас цель.
Пробуйте.
В любом случае надо идти.
Потому что на месте это...
Цели нет.
А когда есть цель, лучше попробовать и понять, что это, допустим, не ваша дорога, чем не попробовать и потом сомневаться, что работает.
Господи, а вы как не поломаете.
Ищите, ищите, и обрежьте.
И обретёте.
Но идёте.
Пробуйте.
Ну, ладно, да.
Даст Бог, у вас ещё и всё-то может получиться.
С массажем вообще будете довольны.
И я вот ещё заметил такое, что в жизни бывают какие-то установки у тебя есть, и ты прям идёшь, и пытаешься этого достичь, и у тебя вот ничего не получается.
Вот не получается всё упорно.
Ты вроде стараешься.
А в какой-то момент я задумался.
А может просто всё это бросить, даже несмотря на то, что я уже какие-то усилия вложил.
Просто жизнь говорит.
Это не твой путь, не иди туда.
Только чтобы это грех.
Ты попробовал и понимаешь, что это дорога.
Что-то я думал, там это, она-то ведёт в Гавилону какие-то.
Там же была другая.
Да.
Я упёртый вот шёл, шёл по ней.
А в какой-то момент потом не счёл.
Ты почему не смотришь?
Жизнь тебе даёт подсказки.
У тебя там плохо.
Зачем ты туда идёшь?
Не иди туда.
Сверни на другую дорогу.
Ну мы только что про это, про то, что вы видите.
Жизненный опыт.
Теоретически это.
Ладно, приятного аппетита.
Здравствуйте.
Спасибо.
С массажём.
Опять же, если массаж не здесь, значит он всё же не в головах.
Вы постоянно это делаете.
Это вот так относитесь.
Зато это у вас опыт.
В следующий раз, если у вас будет это, вы будете знать.
После Алтая?
После Алтая.
Это когда вы уже там.
Вы говорите, знаете, когда.
И это для вас уже.
Вот вы сделали.
О!
Хотел.
Почему?
Потому что позавтракали.
Реально всё.
Опыт.
Опыт.
Не получили, да.
Попалят ещё в деятельности.
Но это если случится, да.
Ну, может быть.
Попал.
Так же вот на память, так сказать, вот Света сегодня
вам благоприятствует, на самом деле, помните.
Света.
Да.
Всё.
Спасибо.
\bye
Будьте здоровы.
До встречи.
Будьте здоровы.
Не забудьте.
Пока.
Пока.
Пока.
Пока.
Пока.
Пока.
Пока.
Пока.
Пока.
Пока.
Пока.
Пока.
Пока.
Пока.
Кто?
Куда?
Ван Лин.
Вонлун.
