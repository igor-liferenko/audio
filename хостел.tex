%&12pt
\pdfpagewidth=297mm
\pdfpageheight=210mm
\pdfhorigin=1in
\pdfvorigin=0pt
\input QUIRE
\shhtotal=\pdfpagewidth
\htotal=.5\shhtotal
\vtotal=\pdfpageheight
\shoutline=0pt
\shstaplewidth=0pt
\shcrop=0pt
\shfootline={}
\shthickness=.27mm
\quire{12}

\horigin=9mm
\hoffset=9mm
\hsize=\htotal \advance\hsize by-2\horigin \advance\hsize by-\hoffset
\advance\hsize by-\QUIRE
\output={\ifodd\pageno\else\hoffset=\QUIRE\fi \plainoutput}

\vorigin=15mm
\vsize=\topskip \advance\vsize by35\baselineskip

\font\TENSL=OMSL10
\headline={\rlap{\vbox to0pt{\vss\hbox{\raise5pt\line{\TENSL
  \ifodd\pageno\hfil Возвращение с летнего интенсива по тайцзицюань\kern.5pt\else
  Разговор в хостеле\hfil\fi}}}}\line{\hrulefill}}
\footline={\raise3pt\line{\hss\tenrm\folio\hss}}

\bgroup
\headline={\hfil}
\footline={\hfil}
\topglue 0pt plus7fil
\centerline{\tt Х\qquad О\qquad С\qquad Т\qquad Е\qquad Л}
\vskip 0pt plus10fil
\eject
\egroup
\def\folio{{\advance\pageno by-1 \number\pageno}}

\font\speakerF=omssbx10
\def\A{\item{\speakerF А.}}
\def\I{\item{\speakerF И.}}
\setbox0=\hbox{\speakerF А.\enskip}
\parindent=\wd0

\hyphenation{тай-цзи-цю-ань}

\I
... И потом начинаешь замечать, что мозг сам уже просит, просится.
То есть вот уже даже независимо от тебя уже сам эти все прокручивает формы уже даже во сне,
как эти все движения делаются.
И вот этот процесс именно обучения как раз для меня именно занимателен, интересен.
То есть вот как изучать что-то, новый предмет.
И чем больше ты смотришь на какую-то вещь, интересуешься, неважно чем,
тем больше она раскрывается и становится интереснее.
Вот это я заметил.
То есть вот именно сам принцип обучения, он как бы в этом, я понял, что он в этом состоит.
То есть бери любой предмет, смотри на него долго, и он начнет раскрываться.
Вот основной принцип.
Неважно что.
Чем подробнее, тем оно богаче становится.

\A
Принцип созерцания и погружения.

\I
Да.
Так, чтобы любая вещь, любая вещь может стать интересной.
Для этого надо просто на неё достаточно пристально и долго смотреть.
Даже вот есть притча\footnote*{Что-то про это вроде Рудольф Штайнер писал.} о каком-то зернышке, что вот мудрец смотрит зернышко,
а в нем он видит всю картину мира в этом зернышке.
Как оно, там, созревает.
Как всё это идёт.
Как все эти процессы природы, они все там отражаются в этом всём.
То есть как бы вся вселенная, грубо говоря, заключена в маленьком зернышке.
Если на него сильно долго смотреть.

\A
Есть же ребята, которые давно на этом --- ездят, занимаются?

\I
Да, они уже 15 лет проводят.
Андрей Владимирович Ширай --- есть
такой мастер боевых искусств.
Вот.
Он организует.
Он у нас был тренером.
Ну, очень много он нам дал информации.
Комплексы там всякие интересные.
То есть у нас было просто как закачка информации прям.
Набор такой информации прям.
Объёмов таких.
Накачка информации.
То есть мы не разучивали что-то определённое несколько раз подряд,
а мы вот именно каждый день у нас что-то новое.
То есть обзорное было у нас.
Загрузка вот именно информационная, обзорная.
Мы там и багуа, эти, разминку, проходили. По кругу
они там «хороводы» водят.
Багуа --- это у них стиль такой.
Они за спину заходят, атакуют со спины.
И у них такая разминка есть.
Они хороводом ходят в одну сторону, в другую сторону.
Вот у нас перед отъездом последняя разминка была как раз багуа стиль.
Интересно.
Ну там много он... И всякие школы боевых искусств разных --- какие у них разминки есть.
Мне понравилось там этот, крокодил машет лапами, разминка.
Потом ласточка касается крыльями воды.
Тоже интересно.
Там такие движения.
Особенно мне понравилось наматывание шёлковой нити в движении.
Вот.
То есть при ходьбе.
Но я не совсем понял как это делается.
Я запутался.
Там просто сложная синхронизация движения рук и ног.
Но мне прям это, прям запало.
Надо более подробно с этим всем делом ознакомиться.

\A
Но у вас в городе есть, да, кто этим занимается?

\I
Нет.
Я один человек только этим занимаюсь.

\A
Неужели.

\I
Ну, у меня есть...

\A
Вы просто не нашли пока.

\I
Я как бы начал заниматься в 2024 году.
Первое знакомство моё с тайцзицюань.
В январе
2024 года по интернету просто увидел, там,
мужчина, ну китаец, делает всякие такие движения, там, такие.
Ну я полез --- «тайцзицюань» --- полез поискал, там, ролики. Стал смотреть.
Ну, там, потом больше, больше... Потом просто в интернете набрал, ну название города, там я под
Красноярском город маленький --- Зеленогорск...

\A
Угу.

\I
Да. И я набрал «Зеленогорск тайцзицюань».
И он, там, высветил секцию.

\A
Он же закрытый по-моему город?

\I
Закрытый, закрытый он и сейчас.

\A
Ребята приезжали оттуда, из Зеленогорска --- каждый год здесь проходит у нас
типа аналог «что где когда». Интеллектуалы с вашего города приезжали молодые.
Так, так, и что дальше там появилось?

\I
Ну да, и вот мне поисковик нашёл, что есть секция. Я звоню в эту секцию, значит, а мне говорят:
``Ну, у нас уже это не преподаётся. Это сайт старый, там информация...
Тот человек, который преподавал, он умер уже.''
Ну это с 90-х годов когда ушу вот это расцвет был, ещё это было всё.
Конец 80-х, 90-е. Вот там у
нас был человек\footnote*{Юрий Вячеславович Андросов} --- он вёл, а потом вот он умер как раз года
три
назад. И как-то вообще интерес к ушу как-то пропал. Вот. И как-бы...\ ну, а информация на сайте
осталась.
Ну я когда звоню...\ ну, мне сказали: ``Там ещё вот ходят люди которые к этому ходили который
вот умер. Занимаются сами просто.''
Я...\ ну, телефон дали --- я позвонил. Ну и там один он как-бы, ну с ними как-бы тоже
занимался. А остальные, они пенсионеры уже все и они занимаются в будний день, а я работаю.
Я никак с ними не попадаю. Один --- он посменно работает. Мы с ним договорились, что перед
работой будем
с ним тренироваться просто --- как «за компанию».

\A
Угу.

\I
Я говорю: ``Давай, вместе. Ты будешь меня учить, там, перед работой как раз разминочка, там всё.
Ну и мы стали тренироваться вдвоём по утрам перед работой.
А потом я --- это где-то с октября --- ну, он стал меня учить помаленьку, там.
Он уже как-бы...\ ну, кое-что там знал, там всё.
А потом я в интернете просто забил --- ну, когда нам отпуск...\ сказали, что график отпусков
делать.

\A
Угу.

\I
Я думаю: надо вот какие-то типа курсы и куда-то съездить.

\A
Конечно, съездить надо обязательно.

\I
Да, совместить. То есть, и отдых и обучиться получше вот этой всей тайцзицюань.

\A
То есть...

\I
Ну я набрал, да, в интернете «тайцзицюань летняя школа» там это...
Ну и он мне вконтакте нашёл. Я тут же позвонил записался. И вот --- всё, удачно съездил.

\A
То есть у вас получается все уроки, всё удачно.

\I
Вообще удачно я съездил.

\A
Я вижу что вы как...\ вдохновлённый.
Это сейчас ещё у вас предстоит какое-то переосмысление.
Сейчас вы пока ещё только, как сказать...\
не выключились процессы.

\I
Да. Ну то есть грубо говоря у нас в городе есть там кто занимается, но они уже все за 60, за 80.
А
один вот этот с которым я человеком занимаюсь ему 55.
А молодых вот чтоб 40, там. Ну может школьники там какие-то секции борьбы ходят, там...\ это. А
именно тайцзицюань и именно моего возраста вообще нет народа.
То есть у нас город маленький там всё, все...

\A
Может попробовать в Красноярск?

\I
Да.

\A
В Красноярске это должно быть.

\I
В Красноярске есть, да.
А у нас, ну, народ не занимается.
Я фактически со своего возраста один человек только этим занимаюсь.
Поэтому мне вот сейчас я приехал, там с ребятами потренировались и я сейчас приеду ---
а с кем я буду?
Мы же там и парную работу делали, и с палками, и туйшоу делали...
Я только сейчас понял что такое туйшоу.
То есть ты стоишь с человеком рядом.
Он пытается тебя спихнуть с места.
Вот просто толкает.
И там интересно так, что человек так сделан интересно, что его
--- если он правильно синхронизирует движение --- его не опрокинешь.
То есть вот реально я по всякому пробовал --- человек опытный.
Бесполезно.
Всё. Он тебя сталкивает и ты его никак не можешь с места стронуть.
И вот в этом как раз туйшоу и заключается.
То есть ты давишь на него, он твою силу маленечко принимает, и уводит в сторону.
И ты своей силой мимо него.
И вот он стоит так, играет.
И ты никак не можешь до него добраться.
Интересно так.
И вот я первый раз только как-бы...\ прочувствовал что это, как это.
Потому что партнера у меня не было до этого.
А тут мы в парах работали.

\A
Конечно.
Видите, практика когда это одно, когда то-же самое.
А когда вот эти семинары, когда другие люди, новый опыт.

\I
Да.

\A
Когда ты пробуешь... Конечно.
Если вы этим интересуетесь, это надо прям на регулярной основе.

\I
Да.
Поэтому я думаю где вот мне сейчас искать партнера.

\A
Ну ищите Красноярск, Новосибирск.
Я думаю здесь семинары будут какие-то.

\I
Ну да, семинары...

\A
Близлежащие города крупные.
Ну а у вас Красноярск крупнейший?

\I
Да.

\A
Ну и Новосибирск.
Потому что даже вот смотрите.
У нас... Ко мне приезжают очень часто те, допустим...\ ведущие,
заинтересованные вот в данном направлении люди.
Они же приезжают --- где делают обычно?
В центральной части.
Москва, Питер.
А в Сибири они здесь едут в
самый крупный город --- Новосибирск.
И он как бы...
И он в центре находится.
То есть, поедешь в Иркутск --- вы понимаете --- с Екатеринбурга не поедешь.
А к нам сюда и с Екатеринбурга поедешь, и с Красноярска, и с Иркутска.
Разумное расстояние.

\I
Да.

\A
Поэтому у нас вы точно найдёте.
То есть то что у нас ушу есть
это я вам точно говорю.
Тайцзицюань... Я боюсь вам сказать.
Я просто...

\I
Я знаю, что есть.
В Новосибирске 5 секций есть.

\A
А! Ну вы уже знаете.
И я думаю что они какие-то мастер-классы, семинары \hbox{п-р-о-в-о-д-я-т}.
И вам просто на них выйти.
Графики их мероприятий, как вам лучше --- под себя.
Вам то что, утром сел...\ вечером сел --- утром здесь.
И здесь вечером сел.

\I
Ну да, это понятно.
Но охота было бы вот именно хотя бы там несколько раз в неделю тренироваться.

\A
Это да. Это само собой.
Там найти единомышленников одно.
А в качестве развития...
То есть даже если у вас будет там 5 единомышленников,
но развитие --- это всегда должен быть мастер выше.
А мастера выше, то есть, это понятно:
когда там --- ежедневные практики, назовём так, а чтобы у вас был рост, это нужно,
опять же, это обязательно, развитие это обязательно. Это так же, как спортсмен: ты можешь
тренироваться для себя в подвале 100 лет, а если соревнований нет, ты не знаешь свой уровень.

\I
Да. И если ты съездил на семинар, а всю эту информацию накопленную ты не практикуешь
--- она бесполезна.

\A
Вам нужны единомышленники, а для развития вы нашли.
Ну, я думаю вы найдёте. В близлежащих городах здесь у нас будет точно.

\I
Мне надо в своём городе единомышленников.

\A
Найдёте, я не сомневаюсь.

\I
Либо переехать в другой город,
что будет, наверное, нелегко.

\A
Ну, это уже другой вопрос, да.
Хотя, знаете...
Вы же понимаете: ставишь цель...
Если у вас будет цель --- вопрос ``нелегко'' отпадёт.
Если цель есть... Нет, ну, сложности,
они, понимаете, они в любом случае будут.
Но будут сложности, либо они, как сказать, катастрофические, или сложности, которые
как этап, как ступенька.
Ну, сложность, ступенька, сложность, ну, ты напрягся, поднялся, ну, ты идёшь дальше.
Ты не стоишь на месте.
Ну вот и всё. Найдёте, не переживайте.

\I
Ну, опять же, чтоб цель ставить, иногда, думаешь: а...\ ну... 
Если цель поставить, её можно добиться, это да.
Но с другой стороны: а нужна ли тебе эта цель?
Тут такие мысли зарождаются.

\A
Опять...
А это постоянный процесс.
Это смысл жизни уже.
Это, это, это у вас...
Если вы ставите такие вопросы, то вы правильно делаете. Эти вопросы вы будете ставить
до конца жизни.
Плох тот, кто не думает ни о смысле жизни, `а нужна ли эта цель'...
Бесцельная жизнь --- она хуже.
Лучше работать постоянно: я делаю правильно или нет --- задавать вопросы.
Когда ты задаёшь вопрос, ты находишь ответ.
А когда ты не задаёшь вопросы `нужна эта цель или нет' --- это в лучшем случае вот
так, ну а скорее всего это деградация.

\I
У меня вот сейчас...\ просто я...\ как-бы на распутье в таком в жизни.
Не знаю где моё, скажем так...

\A
А вы же знаете, вы задавайте вопросы, а ответы придут.
Ты вопросы задаёшь, но...\ как знаете говорят мудрецы:
«Всему своё время». И всё. То, что вы должны узнать, получить знания в 10-м классе,
вы в 1-м их никогда не получили бы.
Ну как бы вам не хотелось в 1-м классе знания получить 10-классника, вы должны пройти ещё 9 лет.
Понимаете, да?
Поэтому ваши...\ ответы на ваши вопросы придут именно к вам в то время, когда это должно быть.
Одному это приходит в 20, другому в 70, третьему в 40, кому-то вообще не дано в этой жизни.
Он их в следующих только реинкарнациях получит.

\I
Ну да.

\A
Поэтому тут же... Вы же на этом пути!
Терпеливо, как вам...
Задали вопрос. Смотрите, и терпеливо ждите, и вы дождётесь.
Ну, не то что терпеливо сиди --- это, знаете как, сидя рыбку не поймаешь.
Возьми удочку.
И сиди лови.

\I
И вот, даже если так сейчас оглянуться назад.
Где я был полтора года назад --- на нулевом уровне --- и как я, за полтора года уже,
что со мной произошло: я туда съездил, с тем познакомился, то узнал, то.
То есть, путь уже пройден такой...

\A
Ну, и дальше будет так же.
Ну, то есть я понимаю, что вам хочется `А я хочу сразу туда', но вы посмотрите
на тех учителей, которые... Они к этому пути ежедневными
занятиями, упорством, трудом --- вы понимаете.
И у вас нужно...\ просто путь.

\I
А распутье у меня заключается именно в том, где мне будет лучше, то есть, а может, вот я
сейчас туда, а может там ещё хуже, чем здесь будет.
Здесь вроде как...

\A
Как говорят китайцы, народная мудрость китайская: «Не попробуешь --- не узнаешь».
Поэтому, это нормально, что вы задаёте вопросы. Пробуйте!
С другой стороны, попробовали... Это как, знаете, пошёл --- нет, не моя дорога.
Вернулся, пошёл в другую --- моя. Или опять `Ну, опять не моя', значит другая.

\I
А у меня уже в жизни так, кстати, было.
Резкая перемена прям образа жизни. И оглядываясь назад, я понимаю, что...

\A
Если бы не сделали...

\I
Если бы не сделал, я бы вообще...

\A
Ну, мы про это только что говорили.
С чего мы начали.
В начале: я говорил, что если вы задаёте вопросы, вы...
Вода найдёт дорогу.
А если вы...
`О, нет, наверное это...'
То есть, если вы сами себя... Задавайте вопросы, ищите ответы.
Если не будете... Будете, говорю, в лучшем случае, а скорее...
А сейчас вы же сами себе... `Да я уж так.' Как правило, это дорога туда, а вам туда не надо.
Туда никому не надо.
Но для этого нужно, как сказать, искать, не бояться, пробовать.
В том числе и ошибки будут.
Это неотъемлемая часть.
Это так же, как в упражнении:
вы сделали не так, а так --- получили опу.
Сделали так --- вы защитились.
Понимаете.
Ошибка была, но ошибка это опа.
То есть, ошибки правильно воспринимать как:
`Окей, это так, я теперь знаю, как не делать.'
Пока ты её не сделал, ты не понимаешь.
В экстремальной ситуации, когда ты знаешь как не делать, шансы победить есть.
А когда ты не ошибался --- бесполезно.
Без проигрыша нет победы.

\I
Да. И тут ещё такой важный фактор --- уверенность в себе.
То есть, вот я понял со временем, что есть, как бы, условия, которые человеку не
подходят абсолютно.
То есть, одному они подходят, а другому, для другого они неестественны.
И многие люди говорят так: `Вот аж ты чего, там, захотел --- сразу то и то.'
И как бы людей, как бы, задавливают. Да?
А человек должен искать своё, своё.
И для этого нужна некая уверенность в себе, чтобы
противостоять этим людям, которые пытаются... `Вот, ишь ты чего захотел, ты вот, значит,
ещё вот это не прошёл, а уже вот туда захотел.' И как-бы они тебя придавливают.
И надо им как-то уметь противостоять.

\A
А здесь, как сказать, здесь не то что противостоять, я бы даже сказал...
Здесь просто двигаться своей дорогой.
Вам зачем тратить время на противо...\ энергию на ``противостояние,'' как вы говорите.
А так, если они... Ну, идёте своей дорогой. Идите своей дорогой.
Ну вот он говорит, `Нет, не ходи туда, не пробуй, потому что тебе там будет плохо.'
А кто-то знает?
Вы задайте вопрос. Вот он предполагает. А другой... Вот, допустим, будет нас два человека.
Вы скажете: `Вот, слушайте, переехать мне из Зеленогорска, условно говоря, там, в Химки,
под Москву?' Круто? Резко? Резко. Оттуда туда. Даже не в Новосибирск.
Они скажут: `Ой да ты что, Москва --- дорогой город. Сдохнешь, там будешь это...\
прозябать под этим...'
А другой скажет: `Даже не знаю, попробуйте, перспектив больше. Тем более, там есть школы, эти...'
Понимаете.
Откуда знает один, что там будет? Никто не знает. А у вас своя дорога.
Если у вас есть цель, сказать: `Ну, здесь я как бы это, а там я могу попробовать.'
Вернуться вы всегда можете.
Отчего вы, если это неправильно, вы...
Поэтому не знает не первый, который говорит, вперёд иди, не знает не второй --- там.
Но если у вас, вы знаете, что надо что-то изменить, вам стоит лучше попробовать, чем...
Ну, опять же, это же всё не то, что там, засунь руку в огонь и прыгни с двадцать пятого этажа.
Мы же говорим о таких, если у вас цель.
Пробуйте.
В любом случае надо идти.
Потому что на месте это...
Цели нет.
А когда есть цель, лучше попробовать и понять, что это, допустим, не ваша дорога, чем не
попробовать и потом сомневаться.
Найдёте.
Ищите и обрящете.
И обретёте.
Но идёте.
Пробуйте.
Ну, ладно, да.
Даст Бог, у вас ещё сегодня может получится.
С массажем вообще будете довольны.

\I
И я вот ещё заметил такое, что в жизни бывают какие-то установки у тебя есть, и ты прям идёшь,
и пытаешься этого достичь, и у тебя вот ничего не получается.
Вот не получается --- и всё. Упорно.
Ты вроде стараешься.
А в какой-то момент я задумался:
а может просто всё это бросить, даже несмотря на то, что я уже какие-то усилия вложил.
Просто жизнь говорит:
`Это не твой путь, не иди туда.'

\A
Только что про это говорили.
Ты попробовал и понимаешь: не та дор\'ога...
Что-то я думал там это, а она-то ведёт в буреломы какие-то.
Там же была другая.

\I
Да. И я упёртый вот шёл и шёл по ней.
А в какой-то момент потом мне щёлкнуло:
`Ты почему не смотришь?
Жизнь тебе даёт подсказки.
У тебя там плохо.
Зачем ты туда идёшь?
Не иди туда.
Сверни на другую дорогу.'

\A
Ну мы только что про это, про всё проговорили. То, что вы и видите.

\I
Жизненный опыт.

\A
Ладно, приятного аппетита.

\I
Спасибо.

\A
Может ещё с массажем повезёт.
Опять же, если массаж не сделаете, у вас будет опыт.
В следующий раз вы будете знать, что надо позвонить заранее записаться.

\I
После Алтая?

\A
После Алтая.
Это когда вы ещё там.
И это для вас уже...
Вот вы сделали.
О! Хотел. Не получилось.
Опыт?
Опыт.
Не получили, да.
Поболят ещё мышцы.
Но это если не получится.
Ну может быть над вами,
так сказать, луч света сегодня...\
вам благоприятствует.

\kern2cm
\it
\lineskip=7pt
\hskip6cm Утро 30 июня 2025 года \par
\hskip5.1cm Новосибирск \par
\hskip4cm Хостел возле Речного вокзала \par
\vfil
\eject
\shipout\vbox{}
\bye
