0:00
И потом начинаешь замечать, что мозг сам уже просит, просится.

0:05
То есть вот уже даже независимо от тебя уже сам эти все прокручивает форму уже даже во сне,

0:13
как эти все движения делаются.

0:15
И вот этот процесс именно обучения как раз для меня именно занимателен, интересен.

0:22
То есть вот как изучать что-то, новый предмет.

0:27
И чем больше ты смотришь на какую-то вещь, интересуешься, неважно чем,

0:33
тем больше она раскрывается и становится интереснее.

0:37
Вот это я заметил.

0:38
То есть вот именно сам принцип обучения, он как бы в этом, я понял, что он в этом состоит.

0:46
То есть бери любой предмет, смотри на него долго, и он начнет раскрываться.

0:58
Вот основной принцип.

0:59
Неважно что.

1:02
Чем подробнее, тем оно богаче становится.

1:05
Принцип созерцания и погружения.

1:07
Да.

1:12
Так, чтобы любая вещь, любая вещь может стать интересной.

1:16
Для этого надо просто на нее достаточно пристально и долго смотреть.

1:20
Даже вот есть притча о каком-то зернышке, что вот мудрец смотрит зернышко,

1:26
а в нем он видит всю картину мира в этом зернышке.

1:32
Как оно там созревает.

1:34
Как оно все это идет.

1:36
Как все эти процессы природы, они все там отражаются в этом всем.

1:40
То есть как бы вся вселенная, грубо говоря, заключена в маленьком зернышке.

1:49
Если на него сильно долго смотреть.

1:52
Это же ребята, которые давно на этом ездят, занимаются.

2:13
Да, они уже 15 лет проводят.

2:16
Андрей Владимирович Ширай есть.

2:19
Такой мастер боевых искусств.

2:23
Вот.

2:24
Он организует.

2:26
Он у нас был тренером.

2:30
Ну, очень много он нам дал информации.

2:35
Комплексы там всякие интересные.

2:37
То есть у нас было просто как закачка информации прям.

2:41
Набор такой информации прям.

2:43
Объемов таких.

2:45
Накачка информации.

2:46
То есть мы не разучивали что-то определенное несколько раз подряд.

2:52
А мы вот именно каждый день у нас что-то новое.

2:54
То есть обзорное было у нас.

2:57
Загрузка вот именно информационная, обзорная.

3:02
Мы там и багуа эти, разминку проходили по кругу.

3:07
Они там хороводы водят.

3:09
Багуа это.

3:10
У них стиль такой.

3:12
Они за спину заходят, атакуют со спины.

3:16
И у них такая разминка есть.

3:17
Они хороводом ходят в одну сторону, в другую сторону.

3:22
Вот у нас перед отъездом эта последняя разминка была как раз багуа стиль.

3:28
Интересно.

3:30
Ну там много он и всякие школы боевых искусств разных у них.

3:34
Какие у них разминки есть.

3:37
Мне понравилось там этот крокодил машет лапами.

3:41
Разминка.

3:41
Потом ласточка касается крыльями воды.

3:45
Тоже интересно.

3:48
Там такие движения.

3:49
Особенно мне понравилось наматывание шелковой нити в движении.

3:56
Вот.

3:56
То есть при ходьбе.

3:58
Я не совсем понял как это делается.

4:01
Я запутался.

4:02
Там просто сложная синхронизация движения рук и ног.

4:07
Но мне прям это.

4:09
Прям запало.

4:10
Надо более подробно это все дело ознакомиться.

4:14
Но у вас вроде есть так дети?

4:16
Нет.

4:17
Я один человек только этим занимаюсь.

4:22
Ну у меня есть.

4:23
Его просто не нашли пока.

4:24
Я как бы начал заниматься в 2024 году.

4:29
Первое знакомство мое с Тайцзи Цуань.

4:32
В январе.

4:33
2024 года по интернету просто увидел там мужчина ну китаец делает всякие такие движения там такие ну я полез в Тайцзи Цуань полез по ней поискал там ролики стал смотреть ну там потом больше больше потом просто в интернете набрал ну название города там я под Красноярском город маленький Зеленогорск да и я набрал Зеленогорск Тайцзи Цуань.

5:02
Ну там высветил сердце.

5:03
Уже закрыто было?

5:04
Закрыто и закрыто вон сейчас.

5:05
Ребята приезжали оттуда Зеленогорска каждый год типа аналог что где когда интеллектуал с вашего города приезжали молодые.

5:18
Так так и что дальше там приходилось?

5:20
Ну да и вот мне поисковик нашел что есть секция я звоню в эту секцию а мне говорят ну у нас уже это не преподается это сайт старый там информация тот человек который преподавал номер уже.

5:36
Ну это с 90-х годов когда у ШУ вот это расцвет был еще это было все конец 80-х 90-х вот там у нас был человек он вел а потом вот он умер как раз года три назад и как-то вообще интерес к УШУ как-то пропал вот и как бы ну а информация на сайте осталась.

5:57
Ну я когда звоню ну мне сказали там еще вот ходят ребята которые к этому ходили которые вот умер.

6:05
Занимаются сами просто?

6:08
Я ну телефон дали я позвонил ну и там один человек он как бы ну с ними как бы тоже занимался и остальные они пенсионеры уже все и они занимаются в будний день а я работаю.

6:28
Я никак с ними не попадаю один он посменно работает мы с ним договорились что перед работой будем с ним тренироваться просто как за компанией.

6:39
Угу.

6:39
Я говорю давай вместе ты будешь меня учить там перед работой как раз разминочка там все.

6:47
Ну и мы стали тренироваться вдвоем по утрам перед работой.

6:52
А потом я это где-то с октября ну он стал меня учить по моряку там он уже как бы ну кое-что там знал там все.

7:02
А потом я в интернете просто забил ну уж когда нам отпуск сказали что график отпусков делать.

7:09
Угу.

7:10
Ну надо вот какие-то типа курсы и куда-то съездить.

7:13
Конечно все надо обязательно.

7:15
Да совместить то есть и отдых и обучиться получше вот этой всей Тайзи Цуань.

7:22
Ну я набрал да в интернете Тайзи Цуань летние летняя школа там это ну и он мне вконтакте нашел я тут же позвонил записался и вот все удачно съездил.

7:36
То есть у вас получается все уроки все удачно?

7:38
Вообще удачно я съездил.

7:40
Я вижу что вы как вдохновленный.

7:42
Это сейчас еще у вас представляет кто-то переосмысление сейчас пока еще только еще как сказать не выключились процессы.

7:50
Да ну то есть грубо говоря у нас в городе есть там кто занимается но они уже все за 60 за 80 а один вот этот с которым я человеком занимаюсь ему 55.

8:00
А молодых вот чтоб 40 там ну может школьники там какие-то секции борьбы ходят там это а именно Тайзи Цуань и именно моего возраста вообще нет народа.

8:10
То есть у нас город маленький там все все.

8:13
Я очень могу сказать что я давно был в Красноярске.

8:16
Да.

8:17
В Красноярске это должно быть.

8:18
В Красноярске есть да.

8:19
И там ну народ не занимается.

8:21
Я фактически со своего возраста один человек только один занимаюсь.

8:25
Поэтому мне вот сейчас я приехал там с ребятами потренировались и я сейчас приеду а с кем я буду.

8:32
Мы же там у парня работу делали с палками ту шоу делали только сейчас понял.

8:38
То есть ты стоишь.

8:41
С человеком рядом.

8:42
Он пытается тебя спихнуть с места.

8:44
Вот просто толкает.

8:47
И там интересно так что человек как сделан интересно что если он правильно синхронизирует движение его не подкинешь.

8:55
То есть вот реально я по всякому пробовал человек опытный.

8:59
Бесполезно.

9:00
Все он тебя ставит и ты его никак не можешь с места строить.

9:04
И вот это как раз тут шоу и заключается.

9:07
То есть ты давишь на него.

9:09
Он твою силу маленечко принимает.

9:11
И уводит в сторону.

9:13
И ты своей силой мимо него.

9:15
И вот он стоит так играет.

9:17
И ты никак не можешь до него добраться.

9:19
Интересно так.

9:21
И вот я первый раз только прочувствовал что это как это.

9:23
Потому что партнера у меня не было до этого.

9:26
А тут мы в парах работали.

9:28
Конечно.

9:31
Практика когда это одно.

9:33
Когда даже самое.

9:35
Когда вот эти семинары.

9:37
Когда другие люди.

9:39
Новый опыт.

9:41
Да.

9:42
Поэтому я думаю где вот мне сейчас искать партнера.

9:46
Ну ищите Красноярск, Новосибирск.

9:48
Я думаю здесь семинары будут какие-то.

9:50
Ну да.

9:51
Семинары.

9:52
Прослежащие города.

9:53
Крупные.

9:54
Ну а у вас Красноярск крупнейший.

9:55
Да.

9:56
Ну и Новосибирск.

9:57
Потому что даже вот смотрите.

9:58
У нас ко мне приезжают очень часто те допустим ведущие

10:05
заинтересованные вот в данном направлении люди.

10:08
Они же приезжают.

10:09
Где делают обычно?

10:11
В центральной части.

10:12
Где ведь?

10:13
Москва, Питер.

10:14
А в Сибири они здесь едут.

10:15
Самый крупный город Новосибирск.

10:16
И он как бы.

10:17
И он в центре находится.

10:18
То есть приедешь в Иркутск.

10:19
Вы понимаете.

10:20
С Екатеринбурга не поедешь.

10:21
А к нам сюда.

10:22
И с Екатеринбурга поедешь.

10:23
И с Красноярска.

10:24
И с Иркутска.

10:25
Как бы разумное расстояние.

10:26
Да.

10:27
Поэтому нас вы точно найдете.

10:28
То есть то что у нас в РУШУ есть.

10:29
Это я вам точно говорю.

10:30
Тайцзи Цюань.

10:31
Я.

10:32
Боюсь вам сказать.

10:33
Я просто у меня зуброга как-то тоже ходила.

10:34
Я знаю что есть.

10:35
Занималась.

10:36
Я сейчас просто не готов сказать.

10:37
А да.

10:38
Я думаю что есть.

10:39
Это точно.

10:40
Есть.

10:41
В Новосибирске 5 секций есть.

10:42
А.

10:43
Ну вы уже знаете.

10:44
И я думаю что они какие-то мастер-классы.

10:45
Семинары.

10:46
Проводят.

10:47
Да.

10:48
И вам просто на них выйти.

10:49
Графики.

10:50
Вот.

10:51
Вот.

10:52
Вот.

10:53
Вот.

10:54
Вот.

10:55
Вот.

10:56
Вот.

10:57
Вот.

10:58
Вот.

10:59
Вот.

11:00
Вот.

11:01
Вот.

11:02
Вот.

11:03
Какие-то мероприятия.

11:04
Но да что под себя?

11:05
Вам кто еще.

11:06
Утром сел.

11:07
Вечером сел утром здесь.

11:08
Или.

11:09
Ну вечером сел.

11:10
Ну да.

11:11
Это понятно.

11:12
Но а кто-то был в Новосибирске хотя бы там несколько раз

11:15
неделю.

11:16
Но по мнению.

11:17
Нет.

11:18
Это.

11:19
Это.

11:20
Прикольноह.

11:21
Не.

11:22
Это да.

11:23
Само собой.

11:24
Так найти единомышленников.

11:25
А в качестве развития.

11:26
То есть даже если у вас будет там 5 единомышленников.

11:27
Но развитие.

11:28
Это всегда должен быть мастер выше.

11:29
Да.

11:30
Замастера выше.

11:31
Да.

11:32
ежедневные практики, назовём так, а, чтобы у вас был рост, это нужно, опять же, это обязательно, это обязательно развитие, это так же, как спортсмен, ты можешь тренироваться для себя в подвале 100 лет, если соревнований нет, ты не знаешь свой уровень.

11:51
Да, я прав, если ты съездил на семинар, а всю эту информацию выкопали, ты не, да, ты будешь, она бесполезна, я говорю, если вам нужна единомышленники, а вы для развития нашли, ну, я думаю, вы найдёте, в ближайших городах здесь у нас будет точно, ну, не надо в своём городе единомышленников, найдёте, я не сомневаюсь, либо переехать в другой город, что будет, наверное, нелегко, ну, это уже другой вопрос, да.

12:20
Хотя, знаете...

12:22
Вы же понимаете, ставишь цель...

12:24
Да, это так.

12:26
Если у вас будет цель, и вопрос нелегко отпадёт, если цель есть, нет, ну, сложности, они, понимаете, они в любом случае будут.

12:34
Да, это правильно.

12:35
Но будут, а, сложности, либо они, как сказать, катастрофические, или сложности, которые, как этап, как ступенька.

12:42
Ну, сложность, ступенька, сложность, ну, ты напрягся, поднялся, ну, ты идёшь дальше.

12:46
Да, да, да.

12:47
Ты не стоишь на месте.

12:48
Это так.

12:49
Ну, всё, найдёте, не переживайте.

12:52
Ну, опять же...

12:52
Ну, опять же, что цель ставить, да, думаешь, а, ну, если цель поставить, её можно добиться, это да.

13:01
Но, с другой стороны, а нужна ли тебе эта цель?

13:05
Нет.

13:06
Такие мысли зарождаются.

13:08
Опять же.

13:08
А это постоянный процесс.

13:10
Это смысл жизни уже.

13:11
Это, это, это у вас, если вы ставите такие вопросы, то вы правильно делаете эти вопросы, вы будете ставить до конца жизни.

13:18
Плохо тот, кто не думает ни о смысле жизни, а нужна ли эта цель.

13:22
Бесценная жизнь, она хуже.

13:24
Лучше работать постоянно, я делаю правильную идею, нет, задавать вопросы.

13:29
Когда ты задаёшь вопросы, ты находишь ответ.

13:31
А когда ты не задаёшь вопросы, нужна ли эта цель, нет, это в лучшем случае вот так, мы, скорее всего, это деградация.

13:40
У меня вот сейчас просто я, как бы, на распутье в таком жизни, не знаю, где моё, скажем так...

13:51
А вы же знаете.

13:53
Вы задавайте вопросы, ответы придут.

13:56
Ты вопросы задаёшь, но, как, знаете, говорят мудрецы, всему своё время, и всё, то, что вы должны узнать, получить знания в 10-м классе, вы в 1-м никогда не получили.

14:10
Ну, как бы вам не хотелось в 1-м классе знания получить 10-классника, вы должны пройти ещё 9 лет.

14:18
Поэтому ваши, ответы на ваши вопросы придут именно к вам в то время, когда это будет.

14:24
Должно быть.

14:25
Одному это приходит в 20, одному к другому в 70, третьему в 40, кому-то вообще не дано в этой жизни, но не в следующем, только в реинкарнациях получит.

14:36
Да.

14:37
Поэтому вы же на этом пути.

14:41
Терпеливо, как вам?

14:43
Задали вопрос, смотрите, и терпеливо ждите, и вы дождётесь.

14:48
Но не то, что терпеливо сиди, это, знаете, как сиди, рыбку не поймаешь.

14:52
Возьми удочку.

14:54
И сиди, и лови, и всё, и как.

14:57
И вот, даже если так сейчас, вот, оглянуться назад, где я был полтора года назад, на нулевом уровне, и как я за полтора года уже, что со мной произошло, я туда съездил, того, с кем познакомился, то узнал, то, то есть, путь уже пройден, вы поняли?

15:14
Ну, и дальше будет так же.

15:16
Ну, то есть, я понимаю, что вам хочется, а я хочу сразу туда, но вы посмотрите на тех учителей, которые, они к этому пути ежедневными.

15:25
Занятием, упорством, трудовым, понимаете, и у вас нужно просто путь.

15:30
А распутье у меня заключается именно в том, где мне будет лучше, то есть, а может, вот я сейчас туда, а может, там ещё хуже, чем здесь будет.

15:39
Здесь вроде как, как говорят китайцы, народная мудрость китайская, не попробуешь, не узнаешь.

15:45
Поэтому, это нормально, что вы задаёте вопросы, пробуйте.

15:50
С другой стороны, попробовали, это как, знаете, пошёл, нет, не моя дорога.

15:56
Вернулся, пошёл в другую, моя, или опять, ну, опять не моя, значит, другая.

16:00
А у меня уже в жизни так, кстати, было.

16:02
Так и на раз.

16:03
Резкая перемена прям образа жизни, и оглядываясь назад, я понимаю, что...

16:09
Если бы не сделали...

16:10
Если бы не сделал, я бы вообще...

16:12
Ну, могу это только что говорить, я говорю, вот я с чего мы начали.

16:15
В начале я говорю, если вы задаёте вопросы это, вы...

16:21
Вода найдёт дорогу.

16:23
А если вы...

16:25
О, нет, наверное...

16:26
То есть, если вы сами себе задаёте вопросы, ищите ответы, если не будете, будете, говорю, в лучшем случае, а скорее, а сейчас вы же сами себе задаёте, да я уж так, как правило, дорогу туда, а то вам туда не надо, туда никому не надо.

16:43
Но для этого нужно, так сказать, искать, не бояться, пробовать, в том числе и ошибки.

16:52
Это не отдельная моя часть.

16:54
Это так же, как в упражнении.

16:56
Вы сделали...

16:57
Не не так, а так, получили опу.

16:59
Сделали так, вы защитились.

17:02
Понимаете?

17:02
Ошибка была, но ошибка это опа.

17:06
То есть, в ошибке правильно воспринимать как?

17:09
Окей, это так, я теперь знаю, как не делать.

17:13
Пока ты её не сделал, ты не понимаешь.

17:15
В внимательной ситуации, когда ты знаешь, как не делать, шансы победить есть.

17:21
Когда ты не ошибался, бесполезно.

17:26
Без проигрыша.

17:27
Да, и тут ещё такой важный фактор, уверенность в себе.

17:32
То есть, вот я понял со временем, что есть, как бы, условия, в которых человеку не подходят абсолютно.

17:39
То есть, одному они подходят, а другому, для другого они неестественны.

17:44
И многие люди говорят так, вот аж ты чего там захотел, сразу туда.

17:49
И как бы людей, как бы, задавливают, да.

17:51
А человек должен искать своё, своё.

17:53
И для этого нужна некая уверенность в себе, чтобы противостоять.

17:57
Противостоять этим людям, которые пытаются, вот, ишь, ты чего захотел, ты вот, значит, ещё вот это не прошёл, а уже вот туда захотел, и как бы они тебя придавливают.

18:07
И надо им как-то уметь противостоять.

18:09
А здесь, как сказать, здесь не то, что противостоять, я бы даже сказал, здесь просто двигаться своей дорогой.

18:17
Вам зачем тратить время на противоэнергию, на противостояние, это когда вы говорите, да так, если нет, ну идёте своей дорогой.

18:27
Ну вот он говорит, нет, не ходи туда, не пробуй, потому что тебе там будет плохо, а кто-то узнает.

18:34
Вы задайте вопрос, вот он предполагает, а другой, вот, допустим, будет нас два человека, вы скажете, вот, слушайте, переехать мне из Селеногорска, условно говоря, там, в Химки, под Москву, круто, только резко, резко, оттуда туда, даже не в Новосибирск.

18:54
Они скажут, ой, да ты что, в Москву, дорогой город, думаешь, там будет...

18:58
А другой скажет, да не знаю, попробуйте, перспектив больше, тем более, там есть школы эти, понимаете.

19:07
Откуда знает один, что там будет, никто не знает, а у вас своя дорога.

19:12
Если у вас есть цель, сказать, ну, здесь я как бы это, а там я могу попробовать.

19:17
Вернуться вы всегда можете.

19:19
Отчего вы, если это неправильно, вы...

19:23
Поэтому не знает не первый, который говорит, вперёд идти, не знает не второй.

19:28
Так.

19:29
Но если у вас, вы знаете, что надо что-то изменить, вам стоит лучше попробовать, чем...

19:35
Ну, опять же, это же всё не то, что там, засунь руку в вагоне, прыгни с двадцать пятого этажа.

19:41
Мы же говорим о таких, если у вас цель.

19:44
Пробуйте.

19:46
В любом случае надо идти.

19:48
Потому что на месте это...

19:50
Цели нет.

19:52
А когда есть цель, лучше попробовать и понять, что это, допустим, не ваша дорога, чем не попробовать и потом сомневаться, что работает.

20:00
Господи, а вы как не поломаете.

20:04
Ищите, ищите, и обрежьте.

20:09
И обретёте.

20:10
Но идёте.

20:12
Пробуйте.

20:16
Ну, ладно, да.

20:17
Даст Бог, у вас ещё и всё-то может получиться.

20:19
С массажем вообще будете довольны.

20:22
И я вот ещё заметил такое, что в жизни бывают какие-то установки у тебя есть, и ты прям идёшь, и пытаешься этого достичь, и у тебя вот ничего не получается.

20:50
Вот не получается всё упорно.

20:52
Ты вроде стараешься.

20:53
А в какой-то момент я задумался.

20:55
А может просто всё это бросить, даже несмотря на то, что я уже какие-то усилия вложил.

21:01
Просто жизнь говорит.

21:03
Это не твой путь, не иди туда.

21:05
Только чтобы это грех.

21:07
Ты попробовал и понимаешь, что это дорога.

21:09
Что-то я думал, там это, она-то ведёт в Гавилону какие-то.

21:13
Там же была другая.

21:15
Да.

21:16
Я упёртый вот шёл, шёл по ней.

21:18
А в какой-то момент потом не счёл.

21:19
Ты почему не смотришь?

21:21
Жизнь тебе даёт подсказки.

21:23
У тебя там плохо.

21:25
Зачем ты туда идёшь?

21:27
Не иди туда.

21:28
Сверни на другую дорогу.

21:30
Ну мы только что про это, про то, что вы видите.

21:34
Жизненный опыт.

21:35
Теоретически это.

21:37
Ладно, приятного аппетита.

21:40
Здравствуйте.

21:41
Спасибо.

21:42
С массажём.

21:44
Опять же, если массаж не здесь, значит он всё же не в головах.

21:49
Вы постоянно это делаете.

21:50
Это вот так относитесь.

21:53
Зато это у вас опыт.

21:55
В следующий раз, если у вас будет это, вы будете знать.

21:58
После Алтая?

22:00
После Алтая.

22:01
Это когда вы уже там.

22:02
Вы говорите, знаете, когда.

22:04
И это для вас уже.

22:06
Вот вы сделали.

22:07
О!

22:08
Хотел.

22:09
Почему?

22:10
Потому что позавтракали.

22:11
Реально всё.

22:12
Опыт.

22:13
Опыт.

22:14
Не получили, да.

22:15
Попалят ещё в деятельности.

22:16
Но это если случится, да.

22:18
Ну, может быть.

22:19
Попал.

22:20
Так же вот на память, так сказать, вот Света сегодня

22:21
вам благоприятствует, на самом деле, помните.

22:22
Света.

22:23
Да.

22:24
Всё.

22:25
Спасибо.

22:26
Будьте здоровы.

22:27
До встречи.

22:28
Будьте здоровы.

22:29
Не забудьте.

22:30
Пока.

22:31
Пока.

22:32
Пока.

22:33
Пока.

22:34
Пока.

22:35
Пока.

22:36
Пока.

22:37
Пока.

22:38
Пока.

22:39
Пока.

22:40
Пока.

22:41
Пока.

22:42
Пока.

22:43
Пока.

22:44
Кто?

22:45
Куда?

22:46
В원лин.

22:47
Вонлун.
